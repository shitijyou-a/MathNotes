\documentclass[a4j]{jarticle}

\usepackage{url}
\usepackage{amssymb}
\usepackage{amsthm}
\theoremstyle{definition}
\newtheorem{theorem}{定理}
\newtheorem{definition}[theorem]{定義}

\def\baselinestretch{0.9} %% 行間を0.9倍に

\title{大きい圏}
\author{shitijyou}

\begin{document}
    \maketitle

    大きい圏は「小さい圏で無い圏」として定義される,
    小さい圏は「その対象が集合となる圏」として定義される.
    だが大きい圏など存在するのだろうか?
    よく使うZFC公理系による集合論は広大で,
    現に現代の数学はこの集合論の上に構築できる.
    対象が集合にならない圏は存在するのか?
    大きい・小さい圏を定義する必要など,あるのか?

    \section{大きい圏の具体例}
    大きい圏は存在する.最も簡単な例は$\mathbf{Sets}$だ.
    これの定義は「対象は全ての集合,射はそれらの間の全ての関数」となっている.
    ラッセルやカントールのパラドックスについての議論から,
    $Ob(\mathbf{Sets})$がZF公理系で集合でないと分かる.

    \section{大きい圏の明確な定義}
        大きい圏の定義を述語論理で記述しよう.

        \subsubsection{クラスの導入}
            まず,述語論理の公理を認める.
            その上でクラス(class)を導入する.

            \begin{definition}[クラス]
                クラスとは述語論理式(=変数項を持つ論理式)~$\phi$を用いて
                次のように表されるものの集まり(collection)である.
                \[ \{x ~ | ~ \phi(x)\} \]
            \end{definition}

            素朴集合論に含まれる内包性公理は「全てのクラスは集合である」を意味する.
            一方でZF公理系に含まれる分出公理は,
            ある集合に含まれるクラスは集合であることを言っている.

        \paragraph{ 真のクラス }
            ZF公理系で集合ではないクラスも存在する.
            ラッセルのパラドックスに現れる$\{x | x \notin x\}$はその簡単な例だ.
            これらを真のクラス(proper class)と呼ぶ.

        \paragraph{ 真のクラスによる定義 }
            クラスの概念を使えば,大きい集合は「対象が真のクラスとなる圏」と表現できる.

        \paragraph{ 集合論の上の圏論 }
            圏論は本質的には土台として集合論を必要としない.
            圏論の公理は全て述語論理だけで記述できる.
            しかし大きい・小さい圏を定義するとき,集合論は不可欠になる.
            集合論無しに圏論を定義すれば大小を定義する必要はない.

    \section{大きい圏を扱う方法}
        すでに見たように,大きい圏を扱うにはZFC公理系では足りない.
        大きい圏はクラスで扱うことができるが,
        これを不用心に用いれば素朴集合論と同じように不完全になってしまう.
        そこでここでは圏論を構築するのに十分で完全な体系を紹介する.

        \subsection{クラスを扱う公理系: NBG}
            クラスを本質的に用い,クラスに制限を加えたものを集合とするNBG公理系では,
            $Ob(\mathbf{Sets})$は集合ではないがクラスとして公理的に扱える.
            この公理系はZFC公理系より真に強い.
            $\mathbf{Sets}$のみならず,クラスが対象となる圏(Grp, Top, Catなど)は全てこの公理系で扱える.
            ただし「全てのクラスのクラス」はNBG公理系でも扱うことはできない.
            クラスを用いる公理系としてはNBG公理系よりも更に強いMK公理系がある.

        \subsection{宇宙を仮定する公理系: ZFCU}
            クラスに使わずに$\mathbf{Sets}$を扱う方法がある.

            \begin{definition}[公理U]
                すべての集合$x$に対して,$x \in  U$となるグロタンディーク宇宙$U$が存在する.
            \end{definition}
            ZFCに上の公理を加えたZFCU公理系がある.これはZFC公理系より真に強い.

            \paragraph{公理Uの意味}
                公理Uは集合全体の集合の存在を保証するものではなく,
                宇宙(集合)の外側にもさらに大きな宇宙が有ることを保証する.
                なので考えている宇宙の内部だけでは議論ができないとわかったならば,
                いつでも使っている宇宙をその外側の宇宙に替えられる.

            \paragraph{もう一つの「小さい」}
                宇宙$\mathfrak{U}$を考え,その内部で議論を行うときはその内部に有る集合を「小さい集合」とし,
                明示するときは``$\mathfrak{U}$-small set"という表現を使う.

            \paragraph{「十分大きい」圏}
                したがって,多くの集合を対象に持つ圏を考えたいのならば,
                その都度適切な宇宙を選んでその中で議論を行えば良い.
                そのような圏は例えば$\mathbf{Sets}_{\mathfrak{U}}$で表される.
                この中で議論すれば事足りるのだから,有る意味でこれは「十分大きい」圏と言える.
                このようにすれば議論の範囲が広くてもZFCU集合論で「小さな圏」のみを考えれば良い.

    \begin{thebibliography}{99}
        \bibitem{guniv} nLab ``Grothendieck universe" \url{https://ncatlab.org/nlab/show/Grothendieck+universe}
        \bibitem{sets} nLab ``category of all sets" \url{https://ncatlab.org/michaelshulman/show/category+of+all+sets}
        \bibitem{univ} Wikipedia ``Universe (mathematics)" \url{http://www.wikiwand.com/en/Universe_(mathematics)}
        \bibitem{st} Wikipedia ``Set theory" \url{https://www.wikiwand.com/en/Set_theory}
        \bibitem{cp} Wikipedia ``Cantor's paradox" \url{https://www.wikiwand.com/en/Cantor's_paradox}
    \end{thebibliography}

\end{document}
