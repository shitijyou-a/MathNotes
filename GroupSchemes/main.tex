\documentclass[a4paper]{jsarticle}
\usepackage[]{../math_note}
\usepackage[]{enumitem}
\usepackage[all]{xy}

\newcommand{\Sch}{\mathbf{Sch}}
\newcommand{\Set}{\mathbf{Set}}
\newcommand{\Ring}{\mathbf{Ring}}
\newcommand{\OpenSubSch}{\mathbf{OpenSubSch}}
\newcommand{\GrpSch}{\mathbf{GrpSch}}

\newcommand{\func}[1]{\underline{#1}}
\newcommand{\ftorM}{\mathcal{M}}

\newcommand{\Ga}{\mathbb{G}_a}
\newcommand{\Gm}{\mathbb{G}_m}
\newcommand{\GL}{GL}

\begin{document}
\title{Group Schemes}
\author{七条 彰紀}
\maketitle

\section{Preface}
    このノートの想定読者は,
    \cite{HarAG}のII, \S 3までを読んだ,
    Geometric Invariant TheoryとModuli Problemに興味がある者である.
    主な参考文献は\cite{Muk1},\cite{AV},\cite{Hos}である.
    大まかな議論の流れは前者の流れを採用し,
    用語などの定義は\cite{Muk1}で述べられているものより
    一般的なものを\cite{AV}と\cite{Hos}から採る.
    \cite{Muk1}で使われる定義は素朴すぎるからである.
    一般的な定義で概念を導入した後,
    特別な場合では\cite{Muk1}での定義と同値に成ることを確かめる,
    という方針を採る.

    このノートでは,次の順に定義していく.
    \begin{enumerate}
        \item $T$-valued point (where $T$ :: scheme),
        \item group scheme,
        \item fine/corse moduli,
        \item categorical/good/geometric/GIT quotient,
        \item representation of group (scheme),
        \item linearly reductive group,
        \item closure equivalence,
        \item unstable/semi-stable/stable.
    \end{enumerate}

    目標とする命題は次のものである.
    \begin{Thm}
        $X$ :: affine scheme,
        $G$ :: linearly reductive group scheme acting on $X$とする.
        affine GIT quotient of $X$ by $G$ :: $X \sslash G$はgood quotientである.
        $X$のstable pointsを$X^s$とすると,
        $X \sslash G$の制限 :: $X^s \slash G$は
        geometric quotient of $X^s$ by $G$である.
%        また,$X \sslash G$はquotient functor :: $\func{X} \slash G$の最良近似である.
    \end{Thm}

\section{$T$-valued Points}
    圏論で言う``generalized point"の概念を,
    名前を変えて用いる.

    \begin{Def}
    \enumfix
    \begin{enumerate}[label=(\roman*),leftmargin=*]
    \item 
    $X, T \in \Sch/S$に対し,
    $\func{X}(T)=\Hom_{\Sch/S}(T,X)$を\textbf{$X$の$T$-valued points}と呼ぶ.

    \item
    field extension :: $k \subseteq K$について
    $S=\Spec k, T=\Spec K$であるときには,
    $\func{X}(T)$を$\func{X}(K)$と書き,$X$の\textbf{$K$-rational points}と呼ぶ.

    \item
    morphism :: $h: G \to H$について
    自然変換$\func{h}: \func{G} \to \func{H}$は
    $\phi \mapsto h \circ \phi$のように射を写す.
    \end{enumerate}
    \end{Def}
    この関手$\func{X}$はfunctor of pointsと呼ばれる.

    $K$-rational pointについては,
    $\func{X}(K)=\{ x \in X \mid k(x) \subseteq K \}$とおく定義もある.
    ここで$k(x)$は$x$でのresidue fieldである.
    しかし\cite{HarAG} Chapter.2 Ex2.7から分かる通り,
    この二つの定義は翻訳が出来る.
    すなわち,
    $k(x) \subseteq K$を満たす$x \in X$と,
    $\Spec k$-morpsihm :: $\Spec K \to X$は一対一に対応する.

    また$X$ :: finite type /$k$であるとき,
    closed point :: $x \in X$について,
    $k(x)$は$k$の有限次代数拡大体である.
    これはZariski's Lemmaの帰結である.
    したがって$\func{X}(\bar{k})$は$X$のclosed point全体に対応する.
    ただし$\bar{k}$は$k$の代数閉包である.

    \begin{Example}
        $\R$上のaffine scheme $X=\Spec \R[x,y]/(x^2+y^2)$の
        $\R$-rational pointと$\C$-rational pointを考えよう.

        $\Spec \R \to X$の射は環準同型 $\R[x,y]/(x^2+y^2) \to \R$と一対一に対応する.
        しかし直ちに分かる通り,
        このような環準同型は
        \[ (\bar{x}, \bar{y}) \mapsto (0, 0) \]
        で定まるものしか存在し得ない.
        ここで$\bar{x}=x \bmod (x^2+y^2), \bar{y}=y \bmod (x^2+y^2)$と置いた.
        よって$\func{X}(\R)$は1元集合.
        また,この環準同型が誘導する$\Spec R \to X$の射は
        1点空間$\Spec \R$を原点へ写す.

        一方,環準同型 $\R[x]/(x^2+1) \to \C$は
        \[ (\bar{x}, \bar{y}) \mapsto (a, \pm ia) \]
        (ここで$i=\sqrt{-1}, a \in \R$)で定まることが分かる.
        すなわち,$\zerosa(x^2+y^2) \subseteq \affine^2_{\C}$の点に対応して,
        $\R[x]/(x^2+1) \to \C$の環準同型が定まる.
        逆の対応も明らか.
        よって$\func{X}(\C)$の元は
        $\zerosa(x^2+y^2) \subseteq \affine^2_{\C}$の点に
        対応している.
    \end{Example}

    \begin{Example}
        体$k$上のaffine variety :: 
        $X \subseteq \affine^n_k$を
        多項式系 :: $F_1,\dots,F_n \in k[x_1,\dots,x_n]$で定まるものとする.
        すると$k$上の環$R$に対して,次の集合が考えられる.
        \[ V_R=\left\{ p=(r_1,\dots,r_n) \in R^{\oplus n} ~\middle|~ F_1(p)=\dots=F_n(p)=0 \right\}. \]
        この集合の元も$R$-value pointと呼ばれる.
        (\cite{Muk1}ではこちらのみを$R$-value pointと呼んでいる.
        実際,こちらのほうが字句``value point"の意味が分かりやすいだろう.)
        $V_R$の点が$\func{X}(R)$の元と一対一に対応することを見よう.

        $X$のaffine coordinate ringを
        $A=k[x_1,\dots,x_n]/(F_1,\dots,F_n)$とし,
        $\bar{x}_i=x_i \bmod (F_1,\dots,F_n) ~(i=1,\dots,n)$とおく.
        $\phi: A \to R$を考えてみると,
        これは次のようにして定まる.
        \[ (\bar{x}_1,\dots,\bar{x}_n) \mapsto (r_1,\dots,r_n) \in V_R. \]
        すなわち,$V_R$の点に対して$\Hom_{\Ring/k}(A,R)$の元が定まる.
        逆の対応は明らか.
        そして,$\Hom_{\Ring/k}(A,R)$が
        $\Hom_{\Sch/\Spec k}(\Spec R, X)=\func{X}(R)$と一対一対応することはよく知られている.
    \end{Example}

\section{Fine/Corse Moduli}
    moduli問題を語るには用語``family"が必要である.

    \begin{Def}
        $\mathcal{P}$を集合のクラス
        \footnote
        {
            集合$X$を変数とする
            述語$X \in \mathcal{C}$の意味を
            「$X$はある条件を満たす対象である」と定義した,
            と考えて良い.
            「属す」の意味は集合と同様に定める.
        }
        とする.
        集合$B$について,
        $B$の構造と整合的な構造を持った集合$\mathcal{F}$と
        写像$\pi: \mathcal{F} \to B$の組が
        $\mathcal{P}$の$B$上の\textbf{family}であるとは,
        各$b \in B$について$\pi^{-1}(s)$は空であるか$\mathcal{P}$に属すということ.
%        $B$はfamily $\pi: \mathcal{F} \to B$のbaseと呼ばれる.
    \end{Def}
    「何らかの上部構造」というのは,
    例えば,
    $S$が位相空間であって
    写像$\ftorM(S) \to S$を連続にするような位相が$\ftorM(S)$に入っている,
    ということである.

    用語``family"を厳密に定義しているものは全くと言っていいほど無いが,
    ここではRenzoのノート
    \footnote{ \url{http://www.math.colostate.edu/~renzo/teaching/Topics10/Notes.pdf} }
    の定義を参考にした.
    ただし,Renzoのノートの定義は一般化されすぎている.
    Renzoのノートでは$\mathcal{P}$を
    ``Let $\mathcal{P}$ define a class of objects in some category $\mathcal{C}$."
    としているが,
    これでは写像$\mathcal{F} \to B$が定義できるか怪しい.
    なので私のこのノートでは$\mathcal{P}$を集合のクラスに限定している.
    結果的に,内容としては``The Encyclopedia of Mathematics"
    \footnote
    {
        ``Moduli theory"のページ.
        \url{ https://www.encyclopediaofmath.org/index.php/Moduli_theory }
    }
    と同様になった.
    ``family"を上のように解釈して不整合が生じたことは,
    私の経験の中ではない.

    \begin{Example}
        $X, B$ :: scheme,
        morphism of schemes :: $f: X \to B$をとる.
        $f$のfibreが成す集合$\{ X_b \mid b \in B\}$は,
        $\pi: X_b \mapsto b$によって$B$上のfamilyを成す.
        fibreが代数幾何学的対象(例えばsmooth curve)であるようなfamilyは
        deformation theoryの対象である.
    \end{Example}

    \begin{Example}
        $k$を適当な体とし,
        $\proj^1_k$の点$O_i~(i=1,2,3)$を順に$(0:1), (1:0), (1:1)$とする.
        この時,$PGL_2(k)$は
        次の全単射で$\proj^1_k$の自己同型写像の$(\proj^1_k)^{\oplus 3}$上のfamilyになる.
        \begin{defmap}
            \pi:& PGL_2(k)& \to& (\proj^1_k)^{\oplus 3} \\
            {}& \phi& \mapsto& (\phi^{-1}(O_i))_{i=1}^3.
        \end{defmap}
    \end{Example}

    以下の定義は\cite{HaMo}など,
    Moduli問題に関する殆どの入門書で述べられている.
    \begin{Def}
        contravariant functor :: $\ftorM : \Sch \to \Set$が
        \textbf{moduli functor}(またはfunctor of families)であるとは,
        各scheme :: $S$に対して,
        $\ftorM(S)$が代数幾何学的対象の$S$上のfamily達を
        familyの間の同値関係で割ったもの
        (``$\{ \text{families over }S \}/\sim_S$" in \cite{Hos})である,
        ということ.
        $\ftorM(S)$には$S$の構造と「整合的」な構造が与えられる.
    \end{Def}
    moduli functorの定義はあえて曖昧に述べられている.
    これは「出来る限り多くのものをmoduli theoryの範疇に取り込みたい」
    という思いがあるからである(\cite{HaMo}).

    \begin{Def}
        scheme :: $M$が
        moduli fuctor :: $\ftorM$に対するfine moduli spaceであるとは,
        $M$が$\ftorM$を表現する(represent)ということである.
        言い換えれば,
        関手$\func{M}=\Hom_{\Sch}(-, M)$が$\ftorM$と自然同型,ということである.
    \end{Def}

%    \begin{Remark}
%        $\Sch$はlocally smallである.
%        2つのscheme :: $X, Y$について,
%        \begin{enumerate}[label=(\arabic*), leftmargin=*]
%            \item 
%                continuous map :: $|X| \to |Y|$の全体が成す集合の濃度は高々$\mathfrak{t}=\#|Y|^{\#|X|}$.

%            \item
%            $Y$の開集合$U$について,
%            写像$\shO_Y(U) \to (f_*\shO_X)(U)$全体の濃度は高々$\mathfrak{s}_U=\#(f_*\shO_X)(U)^{\#\shO_Y(U)}$.

%            \item
%            この写像の集合は開集合で添字付けられているから,
%            結局structure sheafの間の射全体の濃度は
%            高々$\prod_{U \text{ :: open in }Y}\mathfrak{s}_U$.

%            \item
%            $Y$の開集合系の濃度は高々$2^{\#|Y|}$.

%            \item
%            よって
%        \end{enumerate}
%    \end{Remark}

    \begin{Remark}
        $X \in \Sch$をとる.
        moduli functor :: $\ftorM$のfine moduli space :: $M$が存在したとしよう.
        この時,$\ftorM(X) \iso \func{M}(X)$.
        これは
        $X$上の代数幾何学的対象が成す同値類が
        $M$の$X$-value pointと一対一に対応していることを意味する.
        したがって,
        $\ftorM$が指定する代数幾何学的対象の集合の同値類を
        $M$が「パラメトライズ」していると考えられる.
    \end{Remark}

    残念ながら,多くのmoduli functorに対してfine moduli spaceが存在し得ない.
    (このあたりの議論は\cite{HaMo} p.3や\cite{HarDef} p.150にある.)
    そのためMumfordは(おそらくGIT本で)
    fine moduli spaceの代わりとしてcoarse moduli spaceを提唱した.

    \begin{Def}
        moduli functor :: $\ftorM$に対して,
        以下を満たすscheme :: $M$を$\ftorM$のcoarse moduli spaceと呼ぶ.
        \begin{enumerate}[label=(\roman*), leftmargin=*]
            \item
                自然変換$\rho: \ftorM \to \func{M}$が存在する.
            \item
                $\rho$は自然変換$\ftorM \to \func{\tilde{M}}$の中で最も普遍的である:
                \[
                \xymatrix
                {
                    {} & \ar[ld]_-{{}^{\forall} \tau}\ftorM \ar[rd]^-{\rho}& {} \\
                    {}^{\forall}\func{\tilde{M}} \ar[rr]_-{{}^{\exists!} \func{f}}& {} & \func{M}
                }
                \]
                この図式で$\tilde{M}$ :: scheme, $f: M \to \tilde{M}$.
            \item
                任意の代数閉体 :: $k$について
                $\rho_{\Spec k}: \ftorM(\Spec k) \to \func{M}(\Spec k)$は全単射である.
        \end{enumerate}
    \end{Def}

    \begin{Example}
        \cite{HarDef}の\S 26で,
        elliptic curveのmoduliがfine moduliを持たず,
        coarse moduliのみをもつことが述べられている.
        % marrten.pdf Example 3.2
        一般に,
        対象が非自明な自己同型写像をもつときには
        fine moduli spaceが存在し得ない.
        これを述べるにはuniversal familyの定義が
        必要と思われるので省略する.
    \end{Example}

    \begin{Prop}
        moduli functor :: $\ftorM$に対して
        coarse moduli spaceは同型を除いて一意である.
    \end{Prop}

    \begin{Prop}[\cite{HarDef}, Prop23.6]
        scheme :: $M$が
        moduli functor :: $\ftorM$に対する
        fine moduli spaceであるならば,
        $M$は$\ftorM$のcoarse moduli spaceでもある.
    \end{Prop}

    \begin{Prop}[\cite{HarDef}, Prop23.5]
        $S$ :: schemeのopen subschemeと包含写像が成す圏を
        $\OpenSubSch(S)$と書くことにする.
        これは$\Sch/S$のfull subcategoryである.

        moduli functor :: $\ftorM$が
        fine moduli spaceをもつならば,
        任意の$S$ :: schemeについて
        $\ftorM|_{\OpenSubSch(S)}$は$S$上のsheafである.
    \end{Prop}
    \begin{proof}
        $M$ :: fine moduli scheme for $\ftorM$とし,
        $S$ :: schemeを固定する.
        $\shF:=\func{M}|_{\OpenSubSch(S)}$は
        開集合系からのcontravariant functorだから
        presheafであることは定義から従う.
        また$\shF$の元はschemeのmorphismである.
        このことからsheafの公理Identity AxiomとGluability Axiomを
        満たすことも簡単に分かる.
        (一応,\cite{HarAG} II, Thm3.3 Step3を参考に挙げる.)
    \end{proof}

    familyの同値関係は,
    しばしば群作用の軌道分解で与えられる.
    この場合,
    moduli問題は適当なschemeを群作用で「割」ったものを求めることに帰着する.

\section{Definition of Group Schemes}
    $S$ :: scheme上のschemeと$S$-morphismが成す圏を
    \textbf{$\Sch/S$}で表す.
    これはslice categoryの一般的なnotationから来ている.

    group schemeは圏論的に定義される.
    まずは圏論の言葉で述べよう.
    \begin{Def}
        $S$ :: schemeとする.
        $G$ :: scheme over $S$がgroup scheme (over $S$)であるとは,
        $G$が$\Sch/S$におけるgroup objectであるということである.
        group scheme over $S$とhomomorphismsが成す圏を
        $\GrpSch(S)$と書く.
    \end{Def}
    group objectとhomomorphismsの定義を展開すれば次のよう.
    \begin{Def}
    \enumfix
    \begin{enumerate}[label=(\roman*),leftmargin=*]
        \item
        $S$ :: schemeとする.
        $G$ :: scheme over $S$と次の3つの射から成る4つ組が
        \textbf{group scheme (over $S$)}であるとは,
        任意の$T \in \Sch/S$について
        $\func{G}(T)$の群構造が誘導されるということである.
        \begin{alignat*}{2}
            \mu&:       G \times G \to G    && \quad \text{multiplication} \\
            \epsilon&:  S \to G             && \quad \text{identity section}\\
            \iota&:     G \to G             && \quad \text{inverse}
        \end{alignat*}
        $\mu$はgroup lawとも呼ばれる.
        なお,$x, y \in \func{G}(T)$の
        積$x \ast y \in \func{G}(T)$は次のように誘導される.
        \[
            x \ast y:
        \xymatrix
        {
            T \ar[r]^-{\langle x, y \rangle}& G \times G \ar[r]^-{\mu}& G
        }
        \]
        ここで$\langle x, y \rangle$は
        $G \xleftarrow{x} T \xrightarrow{y} G$から
        productの普遍性により誘導される射である.
        単位元は$\epsilon!: T \to S \xrightarrow{\epsilon} G$
        \footnote{ この射は$T \to G \to S \to G$と書いても同じである. },
        $x \in \func{G}(T)$の逆元は$i \circ x$である.

        \item
        group scheme over $S$ :: $G,H$の間の射
        $h: G \to H$が\textbf{homomorphism}であるとは,
        任意の$T \in \Sch/S$について
        $\func{h}(T):\func{G}(T) \to \func{H}(T)$が群準同型であることである.

        \item 
            group schemes over $S$とその間のhomomorphismsが成す圏を\textbf{$\GrpSch(S)$}とする.
    \end{enumerate}
    \end{Def}

    以下の例では$k$を適当な体とし,$k$上のaffine group schemeを定義する.
    \begin{Example}[$\Ga$]
        finitely generated $k$-algebra :: $A=k[x]$と次の3つの$k$-linear mapから,
        $k$上のgroup scheme :: $\Ga$ \& $\mu,\epsilon,\iota$が誘導される.
        \begin{alignat*}{3}
            \tilde{\mu}&:
                A \to A \otimes_k A; &&
                \quad x \mapsto (x \otimes 1)+(1 \otimes x) \\
            \tilde{\epsilon}&:
                A \to k; &&
                \quad x \mapsto 1 \\
            \tilde{\iota}&:
                A \to A; &&
                \quad x \mapsto -x
        \end{alignat*}
        群構造を無視すれば$\Ga=\affine^1_k$.
        この$\Ga$はadditive groupと呼ばれる.

        $x_1=x \otimes 1, x_2=1 \otimes x$とすると,
        $A \otimes A \iso k[x_1,x_2]$となる.
        したがって$f \in A$について$\tilde{\mu}(f)(x_1,x_2) \in k[x_1,x_2]$とみなせる.
        そして$k[x]$のalgebraとしての和は
        $\tilde{\mu}(f)(x_1,x_2)=f(x_1+x_2)$のようにco-algebraに反映されている.
        単位元と逆元は$\tilde{\epsilon}(f)(x)=f(1), \tilde{\iota}(f)(x)=f(-x)$のように
        反映されている.
    
        $\Ga$に備わった群構造はclosed point :: $(a,b)$を$a+b$に写す.
        これを確かめておこう.
        $\affine^1_k \times_k \affine^1_k \iso \affine^2_k$の
        $\bar{k}$-rational point :: $(a,b)$($\bar{k}$は$k$の代数閉体)は
        素イデアル
        \[ \I{p}=(x_1-a, x_2-b)=\{ f \in k[x_1,x_2] \mid f(a,b)=0 \} \]に対応する.
        したがって$\mu(\I{p})=\tilde{\mu}^{-1}(\I{p})$は次のよう.
        \[ \tilde{\mu}^{-1}(\I{p})=\{ g \in A=k[x] \mid \tilde{\mu}(g)(a,b)=g(a+b)=0 \}. \]
        これは$a+b$に対応する素イデアル$(x-(a+b))$に他ならない.
    \end{Example}

    \begin{Example}
        finitely generated $k$-algebra :: $A=k[x,x^{-1}]$と次の3つの$k$-linear mapから,
        $k$上のgroup scheme :: $\Gm$ \& $\mu,\epsilon,\iota$が誘導される.
        \begin{alignat*}{3}
            \tilde{\mu}&:
                A \to A \otimes_k A; &&
                \quad x \mapsto (x \otimes 1) \cdot (1 \otimes x) \\
            \tilde{\epsilon}&:
                A \to k; &&
                \quad x \mapsto 1 \\
            \tilde{\iota}&:
                A \to A; &&
                \quad x \mapsto -x
        \end{alignat*}
        群構造を無視すれば$\Gm=\affine^1_k-\{0\}$.

        こちらも$\tilde{\mu}(f)(x_1,x_2)=f(x_1x_2)$の様に積が入っている.
        $\mu: \Gm \times \Gm \to \Gm$が
        $\bar{k}$-rational point :: $(a,b) \in \affine^2-\{(a,b) \mid ab=0\}$を
        $ab \in \affine^1$に写すことは$\Ga$の場合と同様である.
    \end{Example}
    \begin{Example}
        正整数$n$に対し
        finitely generated $k$-algebra :: $A=k[x_{ij}]_{i,j=1}^n [\det^{-1}]$とおく.
        ここで$\det$は不定元が成す$n$次正方行列$X=\tatev{x_{ij}}_{i,j=1}^n$のdeterminantである.
        $A$と次の3つの$k$-linear mapから,
        $k$上のgroup scheme :: $\GL_n$ \& $\mu,\epsilon,\iota$が誘導される.
        \begin{alignat*}{3}
            \tilde{\mu}&:
                A \to A \otimes_k A; &&
                \quad X \mapsto (X \otimes 1) \cdot (1 \otimes X) \\
            \tilde{\epsilon}&:
                A \to k; &&
                \quad X \mapsto I \\
            \tilde{\iota}&:
                A \to A; &&
                \quad X \mapsto X^{-1}
        \end{alignat*}
        $I$は$n$次単位行列.
        ここで$\tilde{\iota}: X \mapsto I$は
        ($X$の$(i,j)$成分)$\mapsto$($I$の$(i,j)$成分)という意味である.
        $\tilde{\mu}, \tilde{\iota}$の定義も同様である.

        $X_1=X \otimes 1=\tatev{x_{ij} \otimes 1}_{i,j=1}^n,
        X_2=1 \otimes X=\tatev{1 \otimes x_{ij}}_{i,j=1}^n$
        とおけば,
        $f \in k[x_{ij}]$について
        $\tilde{\mu}(f)(X_1, X_2)=f(X_1X_2)$となっている.
        $\mu: \GL_n \times \GL_n \to \GL_n$が
        $\bar{k}$-rational point :: $(M,N) \in \GL_n \times \GL_n$を
        $MN \in \GL_n$へ写すことは$\Ga$での議論と同様である.
        $n=1$の時$\GL_n=\Gm$であることに留意せよ.
    \end{Example}

    3つの例に現れた準同型$\tilde{\mu},\tilde{\epsilon},\tilde{\iota}$は
    それぞれco-multiplication,co-unit, co-inversionと呼ばれる.
    この3つの準同型によってそれぞれの$k$-finitely generatedに
    Hopf algebra
    \footnote
    {
        algebra, co-algebraの構造をもつfinitely generated $k$-moduleであって
        antipodeと呼ばれる自己準同型射を備えるもの.
    }
    の構造が入る.
    一般にaffine group schemeとHopf algbraが
    一対一に対応する(\cite{MilneAGS} II,Thm5.1).


\section{Categorical/Good/Geometric/GIT Quotients}
    % Moduli Theory and Classification Theory of Algebraic Varieties(H_Popp)
    % のp.55からのlecture 5も参考に成る.
    % 特にRemark5.3にcategorialだがgoodでないquotientの例がある.
    % 同様にgoodだがgeometricでないquotientの例もある.
    % graph mapがproperの時,
    % geometric quotientは単に|Q|=|X|/Gなるgood quotientだと言える.

\section{Linearly Reductive Group}

\section{Affine GIT Quotient is Good Quotient.}

\begin{thebibliography}{99}
    \bibitem{Muk1}
        向井茂(2008)『モジュライ理論 I』岩波書店

    \bibitem{AV}
        Gerard van der Geer, Ben Moonen
        ``Abelian Varieties"
        \url{https://www.math.ru.nl/~bmoonen/research.html}
        (Preliminary Version. 2017/12/31参照)

    \bibitem{Hos}
    Victoria Hoskins (2016)
    ``Moduli Problems and Geometric Invariant Theory"
    \url{https://userpage.fu-berlin.de/hoskins/M15_Lecture_notes.pdf}

    \bibitem{HarAG}
    Robin Hartshorne(1977)
    ``Algebraic Geometry"
    Springer

    \bibitem{HarDef}
    Robin Hartshorne
    ``Deformation Theory"
    Springer

    \bibitem{Eisen}
    David Eisenbud(1999)
    ``Commutative Algebra: with a View Toward Algebraic Geometry"
    Springer
    
    \bibitem{MilneAGS}
    J. Milne, ``The basic theory of affine group schemes", 
    \url{www.jmilne.org/math/CourseNotes/AGS.pdf}

    \bibitem{HaMo}
    J. Harris,I. Morrison ``Moduli of Curves"

\end{thebibliography}
\end{document}
