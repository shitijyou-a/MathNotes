\documentclass[a4j, 10pt]{jarticle}
\usepackage{../math_note}

\title{Cantor-Schr\"{o}der-Bernsteinの定理 \\
 \normalsize{行間全部埋めました}}
\author{七条 彰紀}

\begin{document}
    \maketitle
    
    \section{歴史}
        ここで述べる微妙に異なる二つの証明では,ある写像Fの不動点に注目する.
        この証明を私は『天書の証明』で知った.
        そこではこの証明がAndrei. V. Zelevinskyに帰されている.
        
        さらに遡ると,この定理は論文~\cite{tarski}で,
        ``Knaster-Tarski Fixed Point Theorem"の応用として言及される.
        言及はTarskiが脚注で次のように述べているのみだ.
        \begin{quote}
            In 1927 Knaster and the author proved
            a set-theoretical fixpoint theorem (...);
            see [3]\footnote{原文ママ},
            where some applications of this result in set theory
            (a generalization of the Cantor-Bernstein theorem) and
            topology are also mentioned.
            A generalization of this result is the lattice-theoretical
            fixpoint theorem stated above as Theorem 1.
        \end{quote}

        なのでこの証明の初出は恐らくKnasterによる~\cite{knaster}なのだが,
        残念ながら私には確信が持てない.
        この論文は仏語で書かれており,私は仏語が読めない.
        だが一応見た限りでは以下で述べる証明の断片らしきものが
        読み取れるので,間違いないだろうと考えている.

        もっとも,以下に述べる証明は不動点定理と密接に係るが,必要としない.
        なのでこの経路でたどっても初出には辿り着かないかもしれない.
        まあ,自己満足である.

    \section{定理}
        \begin{Them}[表現1]
            集合A, Bについてそれぞれの濃度を$|A|, |B|$のように表す.
            $|A| \leq |B|$かつ$|A| \geq |B|$ならば$|A|=|B|$である.
        \end{Them}

        \begin{Them}[表現2]
            A, Bを集合とし,単射$f:A \to B$と単射$g: B \to A$があったとする.
            この時全単射$h:A \to B$が存在する.
        \end{Them}

        表現1はこの定理を証明するモチベーションの出処がよく分かる.
        表現2からは定理が写像についての基本的な定理であることが分かる.

    \section{証明}
        この定理が述べているのは全単射$h$の存在であるから,証明の方針は二つ有る.
        一つは別の存在定理の帰結として$h$の存在を示す方針,
        もう一つは全単射$h$を実際に構成する方針である.
        このレポートでは後者の方針を取る.
        前者の方針での証明は知らない。可能かどうかもわからない。

        \subsection{全単射hの構成方法}
        $f, g^{-1}; M \to N$とする。
        $M$から$N$への全単射$h$を作る方法として、
        $M$の各元が$M$の部分集合Sに属すか属さないかでMの元を送る先を$f(x)$か$g^{-1}(x)$にする方法がある。
        \[
            h(x)=
            \begin{cases}
                f(x) & (x \in S) \\
                g^{-1}(x) & (x \in X \setminus S) \\
            \end{cases}
        \]
        ただし$S \subset M$は$M \setminus S \subset g(N)$と$f(S) \cup g^{-1}(M \setminus S)=N$を満たす。

        $f$は$M$全体で定義されるが、像が$N$全体とは限らない。
        一方$g^{-1}$は$M$全体で定義されるとは限らないが、像は$N$全体である。
        この二つを組み合わせることで全単射が作れそうだ、というのがこの構成の着想である。
        しかし$f, g^{-1}$が互いをうまく補い合えるかどうかは自明でない。

        \subsection{写像$F$}
            部分集合$S \subset M$に課された二つの条件の内、
            $f(S) \cup g^{-1}(M \setminus S)=N$に注目する。
            この式は$g^{-1}$を用いているが、これを取り除く方針で式変形をする。
            \begin{eqnarray*}
                f(S) \cup g^{-1}(M \setminus S)&=&N \\
                g^{-1}(M \setminus S)&=&N \setminus f(S) \\
                M \setminus S&=&g(N \setminus f(S)) \\
                S&=&M \setminus g(N \setminus f(S)) \\
            \end{eqnarray*}
            そこで、写像$F$を次のように定義する。
            \begin{eqnarray*}
                F : \mathcal{P}(M) &\to& \mathcal{P}(N) \\
                    S &\mapsto& M \setminus g(N \setminus f(S))
            \end{eqnarray*}
            すると、部分集合$S \subset M$の条件のうち一つは$F(S)=S$、
            すなわち「$S$の写像$F$の不動点である」ということになる。

            \subsubsection{もうひとつの条件について}
            もうひとつの条件$M \setminus S \subset g(N)$は
            $S=F(S)$から導かれるので、我々は$S=F(S)$を満たす部分集合$S$を探すだけで良い。
            実際、$S=F(S)$より、$M \setminus S=g(N \setminus f(S))$である。
            そして$N \setminus f(S) \subset N $だから$M \setminus S=g(N \setminus f(S)) \subset g(N)$
            が成立する。

        \subsection{$F$の不動点}
        以上の議論から、我々は写像$F$の不動点さえ求めれば全単射$h$が構成できることがわかった。
        この$F$について,直ちに以下が示される.
        \[ \Forall{I \subseteq M} F(I) \subset I. \]
        つまり,$F$は$\mathcal{P}(M)$上の縮小写像である.
        このことを用いて2つの証明を述べる.

        \subsection{$F$の不動点が存在することの証明(非構成的)}

        \subsection{$F$の不動点が存在することの証明(構成的)}
            以下が不動点である.
            \[ S=\bigcap_{i \geq 0}{F^{i}(M)}=M \cap F(M) \cap F(F(M)) \cap \cdots. \]
            これが不動点であることは以下のように示される.
            \begin{align*}
            F(S)
            =&F \left(\bigcap_{i \geq 0}{F^{i}(M)} \right) \\
            =&M \setminus g \left(N \setminus f \left(\bigcap_{i \geq 0}{F^{i}(M)} \right) \right) \\
            =&M \setminus g \left(N \setminus \bigcap_{i \geq 0}{f(F^{i}(M))} \right)\\
            =&M \setminus g \left(\bigcup_{i \geq 0}{N \setminus f(F^{i}(M))} \right)\\
            =&M \setminus \bigcup_{i \geq 0}{g(N \setminus f(F^{i}(M)))}\\
            =& \bigcap_{i \geq 0}{M \setminus g(N \setminus f(F^{i}(M)))}\\
            =& \bigcap_{i \geq 0}{F(F^{i}(M))} \\
            =& \bigcap_{i \geq 1}{F^{i}(M)} \\
            =& M \cap \bigcap_{i \geq 1}{F^{i}(M)} \\
            =& \bigcap_{i \geq 0}{F^{i}(M)} \\
            =& S
            \end{align*}
            $f$が単射であり,かつ$F$が縮小写像すなわち$I \cap F(I)=F(I)$であることを用いた.

    \begin{thebibliography}{99}
        \bibitem{tarski} Tarski, A.
            ``A Lattice-Theoretical Fixpoint Theorem and Its Applications."
            Pacific J. Math. vol.5, pp.285-309, 1955

        \bibitem{knaster} B. Knaster.
            `Un théorème sur les fonctions d'ensembles"
            ANNALES DE LA SOCI\.{E}T\.{E} POLONAISE DE MATH\.{E}MATIQUE
            vol.6, pp.133-134, 1928
    \end{thebibliography}

\end{document}
