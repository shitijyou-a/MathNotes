\documentclass[a4paper]{jsarticle}
\usepackage[all]{xy}
\usepackage{../math_note, exercise, enumitem}
\renewcommand{\thesection}{Ex5.\arabic{section}}

\newcommand{\shE}{\mathcal{E}}
\newcommand{\shF}{\mathcal{F}}
\newcommand{\shG}{\mathcal{G}}
\newcommand{\shH}{\mathcal{H}}
\newcommand{\shI}{\mathcal{I}}
\newcommand{\shJ}{\mathcal{J}}
\newcommand{\shO}{\mathcal{O}}
\newcommand{\Proj}{\operatorname{Proj}}
\newcommand{\coverU}{\mathfrak{U}}
\newcommand{\OpenIn}{\text{ :: open in }}
\newcommand{\ClosedIn}{\text{ :: open in }}

\usepackage[scr=boondoxo,scrscaled=1.0]{mathalfa}
\newcommand{\shHom}{\mathscr{H\!o\!m}}

\begin{document}

\section{The Dual of Locally Free Module Sheaf.} %% Ex5.1 
    $(X,\shO_X)$をringed spaceとし,
    $\shE$を有限階数のlocally free $\shO_X$-moduleとする.
    また,$\shE$の双対を$\check{\shE}=\shHom_{\shO_X}(\shE, \shO_X)$で定める.
    ($\shHom$はEx1.15で定義されている.)
    $(\check{\shE}) \check{}$も同様である.

    \begin{Lemma}
        $\shF$ :: $\shO_X$-module, $x \in X$とする.
        このとき,$x$に対して$n>0$が存在して
        \[ (\shHom_{\shO_X}(\shE, \shF))_x \cong (\shF_x)^{\oplus n} \cong \Hom_{\shO_{X,x}}(\shE_x, \shF_x). \]
    \end{Lemma}
    \begin{proof}
        Ex5.7の内容は使う.
        $U \OpenIn X$を十分小さく取れば$\shE|_U$はfree moduleになる.
        したがって以下が成り立つ.
        \begin{align*}
            {}&     (\shHom_{\shO_X}(\shE, \shF))(U) \\
            =&      \Hom_{\shO_U}(\shE|_U, \shF|_U) \\
            \cong& \Hom_{\shO_U}(\shO_U^{\oplus n}, \shF|_U) \\
            \cong& \Hom_{\shO_U}(\shO_U, \shF|_U)^{\oplus n} \\
            \cong& (\shF|_U)^{\oplus n} \\
            =& \varinjlim_{W \supseteq U} (\shF(W))^{\oplus n}
        \end{align*}
        最後で$\bigoplus$と$\varinjlim$が可換であることを用いた.
        このことから以下を得る.
        \[
            \varinjlim_{U \ni x} \varinjlim_{W \supseteq U} (\shF(W))^{\oplus n}
            =\varinjlim_{W \ni x}(\shF(W))^{\oplus n}
            =(\shF_x)^{\oplus n}.
        \]
        あとは$\shO_{X,x}$-moduleの同型から最後の同型を得る.
        \[
            (\shF_x)^{\oplus n}
            \cong \Hom_{\shO_{X,x}}((\shO_{X,x})^{\oplus n}, \shF_x)
            \cong \Hom_{\shO_{X,x}}(\shE_x, \shF_x).
        \]
    \end{proof}
    $\shO_X$-homomorphismを構成し,
    それがstalkでmoduleの射としてisomorphismになっていることを確認する.

\subsection{$(\check{\shE}) \check{} \cong \shE$.}
    写像$\Phi: \shE \to (\check{\shE}) \check{}$を以下のように定める.
    \[
        (\Phi_U(s))_V(\phi)=\phi(s|_V)
        ~\mwhere~
        U,V \OpenIn X, V \subseteq U, s \in \shE(U), \phi \in \Hom_{\shO_V}(\shE|_V, \shO_V).
    \]
    これが$\shO_X$-homomorphismであることは明らか.

%    $\Phi$が単射であることは,
%    任意の$\phi \in \Hom_{\shO_V}(\shE|_V, \shO_V)$について
%    $\phi(s)=0$となる$s \in \shE(U)$が$0$しか無い($\ker \Phi=0$)ことから分かる.

%    全射になることを確かめる.
%    $\shE|_U \cong \shO_U^{\oplus n}$となる$U \OpenIn X, n \in \N$をとる.
%    また,$V \subseteq U$を開集合とする.
%    $\{e_i\}_{i=1}^n \subset \shO_U^{\oplus n}$を基底とし,
%    $\{e^i\}_{i=1}^n \subset \Hom_{\shO_V}(\shO_V^{\oplus n}, \shO_V)$を
%    $e^i(e_j|_V)=\delta_{ij}$\footnote{左辺はクロネッカーのデルタである.}で定めると,
%    $\{e^i\}_{i=1}^n$は$\Hom_{\shO_V}(\shO_V^{\oplus n}, \shO_V)$の基底となる.
%    そして$\psi \in ((\check{\shE}) \check{})(U)$をとる.
%    \[
%        ((\check{\shE}) \check{})(U)
%        =\Hom_{\shO_U}(\shHom_{\shO_U}(\shO_U^{\oplus n}, \shO_X)|_U, \shO_U)
%        \cong \Hom_{\shO_U}(\shHom_{\shO_U}(\shO_U^{\oplus n}, \shO_U), \shO_U)
%    \]
%    最後の$\cong$はEx1.18を用いた.
%    そこで$\tilde{\psi} \in \shE|_U=\shO_U^{\oplus n}$を以下のように定める.
%    \[ \tilde{\psi}=\psi(e^1)e_1+\dots+\psi(e^n)e_n. \]
%    すると$\Phi(\tilde{\psi})=\psi$となる.
%    実際,$\alpha=c_1 e^1+\dots+c_n e^n \in \shE|_U$について
%    \[ \Phi(\tilde{\psi})(\alpha)=\alpha(\tilde{\psi})=c_1 \psi(e^1)+\dots+c_n \psi(e^n)=\psi(\alpha). \]
%    $U,V$を動かせば,$X$の任意の開集合$U,V$で$(\Phi_U(-))_V$が全射になる事が分かる(?).

\subsection{For any $\shO_X$-module $\shF$,
    $\shHom_{\shO_X}(\shE,\shF) \cong \check{\shE} \otimes_{\shO_X} \shF.$}
%    $\shE|_U \cong \shO_U^{\oplus n}$となる$U \OpenIn X, n \in \N$をとる.
%    また,$V \subseteq U$を開集合とする.
%    \[
%        \shHom(\shE,\shF)(U)=\Hom(\shO_U^{\oplus n},\shF|_U),~~
%        \check{\shE} \otimes \shF=\Hom(\shO_U^{\oplus n},\shO_U) \otimes \shF(U)
%    \]
%    $\Hom, \shHom, \otimes$は$\shO_X(U)$-moduleとしてのものである.
    

\subsection{For any $\shO_X$-module $\shF,\shG$,
    $\Hom_{\shO_X}(\shE \otimes \shF,\shG) \cong \Hom_{\shO_X}(\shF, \shHom_{\shO_X}(\shE,\shG))$}
    

\subsection{Projection Formula.}


\section{Module Sheaves over the $\Spec$ of a valuation ring.} %% Ex5.2 

\section{$\tilde{\square}$ and $\Gamma$ are Adjoint Pair.} %% Ex5.3 

\section{The Original Definition of (Quasi-)Coherent Sheaves.} %% Ex5.4 

\section{Is $f_* \shF$ Coherent?} %% Ex5.5 

\section{Support.} %% Ex5.6 

\section{The Stalks of Locally Free Sheaves are Free.} %% Ex5.7 

\section{$\phi(x)=\dim_{k(x)} \shF_x \otimes_{\shO_x} k(x)$.} %% Ex5.8 

\section{Quasi-Finitely Generated Graded $S$-Modules.} %% Ex5.9 

\section{Saturated Ideals and Closed Sub-Schemes.} %% Ex5.10 

\section{The Segre Embedding.} %% Ex5.11 

\section{Very Ample Invertible Sheaves.} %% Ex5.12 

\section{The $d$-uple Embedding.} %% Ex5.13 

\section{The $d$-uple Embedding is Projectively Normal.} %% Ex5.14 
これはch I, Ex3.17bで私が考察したことのSchemeにおける一般化である.

\section{Extension of Coherent Sheaves.} %% Ex5.15 

\section{Tensor Operations on Sheaves.} %% Ex5.16 

\section{Affine Morphisms.} %% Ex5.17 

\section{Vector Bundles.} %% Ex5.18 

\end{document}
