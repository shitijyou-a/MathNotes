\documentclass[a4paper]{jsarticle}
\usepackage[all]{xy}
\usepackage{../math_note, braket}

\newcommand{\shO}{\mathcal{O}}
\newcommand{\Sch}{\mathbf{Sch}}
\newcommand{\Var}{\mathbf{Var}}
\newcommand{\Rings}{\mathbf{Rings}}
\newcommand{\red}[1]{#1_{\text{red}}}
\newcommand{\basesp}{\operatorname{sp}}
\newcommand{\res}{\operatorname{res}}
\newcommand{\Rat}{\operatorname{Rat}} %% rational points
\newcommand{\Proj}{\operatorname{Proj}}
\newcommand{\CoverU}{\mathfrak{U}}
\newcommand{\OpenIn}{\text{ :: open in }}
\newcommand{\ClosedIn}{\text{ :: open in }}

\begin{document}
    schemeやscheme morphismの性質の定義は
    \url{section3_text.pdf}にまとめたので参照すること.
    同じPDFで$B$-fin.gen. schemeなどの独自の用語を定義している.
    \url{http://stacks.math.columbia.edu/tag/01T0}も参照すると良い.

\section{Definition(s) of Locally of Finite Type Morphism.} %% Ex3.1 
以下の命題を示す.
\begin{align*}
    {}&
    \Exists{\{B_i\}_{i \in I}}
    \lbra{Y=\bigcup_{i \in I} \Spec B_i} \land \lbra{\Forall{i \in I} f^{-1}(\Spec B_i)\text{ :: locally $B_i$-fin.gen. scheme}} \\
    \iff&
    \Forall{\Spec A \subseteq X} f^{-1}(\Spec A)\text{ :: locally $A$-fin.gen. scheme}
\end{align*}

下から上は自明である.上から下を示そう.

まず$U=\Spec A \subset X$をとり,$U_i=\Spec A \cap \Spec B_i$を考える.
$U_i \subseteq \Spec A$なので開基$D(a_{ij}) ~~(a_{ij} \in A)$によって$U_i$を被覆できる.
また$D(a_{ij})=D(b_{ij}) ~~(b_{ij} \in B_i)$とする.
この時$f^{-1} U=\bigcup_{i,j} f^{-1}D(b_{ij})$

最終的に$\Spec A \subseteq \Spec B$かつ$f^{-1}\Spec B\text{ :: locally $B$-fin.gen. scheme}$の場合に帰着させられると思う.

\section{ } %% Ex3.2 

\section{ } %% Ex3.3 

\section{ } %% Ex3.4 

\section{ } %% Ex3.5 

\section{ } %% Ex3.6 

\section{ } %% Ex3.7 

\section{ } %% Ex3.8 

\section{ } %% Ex3.9 

\section{ } %% Ex3.10 

\section{ } %% Ex3.11 

\section{ } %% Ex3.12 

\section{ } %% Ex3.13 

\section{ } %% Ex3.14 

\section{ } %% Ex3.15 

\section{ } %% Ex3.16 

\section{ } %% Ex3.17 

\section{ } %% Ex3.18 

\section{ } %% Ex3.19 

\section{ } %% Ex3.20 

\section{ } %% Ex3.21 

\section{ } %% Ex3.22 

\section{ } %% Ex3.23 

\end{document}
