\documentclass[a4paper]{jsarticle}
\usepackage[all]{xy}
\usepackage{../math_note, braket}

\newcommand{\shO}{\mathcal{O}}
\newcommand{\Sch}{\mathbf{Sch}}
\newcommand{\Var}{\mathbf{Var}}
\newcommand{\Rings}{\mathbf{Rings}}
\newcommand{\red}[1]{#1_{\text{red}}}
\newcommand{\basesp}{\operatorname{sp}}
\newcommand{\res}{\operatorname{res}}
\newcommand{\Rat}{\operatorname{Rat}} %% rational points
\newcommand{\Proj}{\operatorname{Proj}}
\newcommand{\CoverU}{\mathfrak{U}}
\newcommand{\OpenIn}{\text{ :: open in }}
\newcommand{\ClosedIn}{\text{ :: open in }}

\begin{document}
    schemeやscheme morphismの性質の定義は
    \url{section3_text.pdf}にまとめたので参照すること.
    同じPDFで$B$-fin.gen. schemeなどの独自の用語を定義している.
    \url{http://stacks.math.columbia.edu/tag/01T0}も参照すると良い.

    記法について.$\Spec A_f=D_A(f)$と書く.

\section{Definition(s) of Locally of Finite Type Morphism.} %% Ex3.1 

\begin{Lemma}[Nike's Lemma]
    $X$ :: scheme, $U,V \subseteq X, U=\Spec A, V=\Spec B$かつ
    $U \cap V \neq \emptyset$とする.
    この時,任意の点$P \in U \cap V$に対し,$a \in A, b \in B$であって
    \[ P \in D_A(a)=D_B(b) \subset U \cap V \]となるものがある.
    系としてProp2.2より$A_a \cong B_b$が得られる.
\end{Lemma}
\begin{proof}
    適当に$a \in A, b \in B$をとり,
     \[ P \in D_B(b) \subseteq D_A(a) \subseteq U \cap V \]
    としよう.
    $X=\Spec B, X_f=D_B(b), \bar{b}=b|_{D_A(a)} \in A_a$
    としてEx2.16aを用いると,
    \[ D_B(b)=D_A(a) \cap D_B(b)=\Spec (A_a)_{\bar{b}}. \]
    なので,あとは$(A_a)_{\bar{b}}$を調べれば良い.

    $(A_a)_{\bar{b}}$の元は以下のように書ける.
    \[ \frac{u/a^m}{\bar{b}^n}=\frac{u}{a^m\bar{b}^n} ~~(m,n \in \N; u \in A). \]
    $\bar{b} \in A_a$なので$a^N\bar{b}=a' \in A$となる$N \in \N$が存在する.
    \[ \frac{u a^{nN}}{a^m a^{nN}\bar{b}^n}=\frac{u a^{nN}}{a^m a'^{n}}. \]
    仮に$m \geq n$とすると
    \[ \frac{u a^{nN}}{a^m a'^{n}}=\frac{ua^{m-n+nN}}{(aa')^m} \]
    $m \leq n$でも同様に分子分母に$a'^{n-m}$をかければ,
    $(A_a)_{\bar{b}}$の元は$A_{aa'}$の元として書ける.
    逆に$A_{aa'}$の元を$(A_a)_{\bar{b}}$の元として書くことは直ちに出来る.
    よって$(A_a)_{\bar{b}}=A_{aa'}$.

    以上より,$\alpha=aa' \in A, b \in B$について$D_B(b)=D_A(\alpha)$.
\end{proof}

\begin{Lemma}[Preimage of POS\footnote{Principle Open Set} is POS.]
    $f: X \to Y$ :: scheme morphism.
    $\Spec B \subseteq Y, f^{-1}\Spec B=\bigcup_{i \in I} \Spec C_i$とする.
    この時,以下が成立する.
    \[
        \Forall{b \in B} \Exists{\{c_i(\in C_i)\}}
        f^{-1}D_B(b)=\bigcup_{i \in I}D_{C_i}(c_i).
    \]
\end{Lemma}
\begin{proof}
    $U=\Spec B, V_i=\Spec C_i$とする.
    すると$f$の制限によりscheme morphism $f|_{V_i}: V_i \to U$が得られる.
    これは$V_i \hookrightarrow X \xrightarrow{f} Y$という写像で,
    したがって逆写像は$(f|_{V_i})(S)=f^{-1}(S) \cap V_i$であることに注意.
    structure sheafの間の射を考えると,以下が得られる.
    \[ \phi_i=\left((f|_{V_i})^{\#}\right)_{U}: B=\shO_U(U) \to (f|_{V_i})_* \shO_{V_i}(U)=C_i. \]
    ここでProp2.2を用いた.
    Prop2.3から,$\phi_i$は$f|_{V_i}: V_i \to U$に1-1対応し,
    特にtopological spaceとして
    \[ f|_{V_i}(\I{p})=\phi_i^{-1}(\I{p}) ~~(\I{p} \in \Spec C_i) \]
    が成り立つ.このことから以下が得られる.
    \[ f^{-1}(D_B(b)) \cap V_i=(f|_{V_i})^{-1}D_B(b)=D_{C_i}(\phi_i(b)). \]
    最左辺と最右辺を$\bigcup_{i \in I}$すれば主張が示せる.
\end{proof}

\begin{Lemma}
    $f \in A$とする.
    有限生成$A_{f}$代数は有限生成$A$代数でもある.
\end{Lemma}
\begin{proof}
    変数の数は問題にならないので1変数で証明する.
    (つまり以下で$A_{f}[x]$を多変数にしても構わない.)
    有限生成$A_{f}$代数$B$には$A_{f}[x]$からの全射が存在する.
    $A_{f}[x]$には$A[x,y]$から次のような全射が存在する.
    \[ y \mapsto 1/f \]
    これが全射であることは,
    \[ ay^nx^m \mapsto (a/f^n)x^m \in A_{f}[x] \]
    のように分かる.
    あとはこの写像が$A$準同型(代入写像)であることに注意すれば良い.
    よって$A[x,y] \to A_{f}[x] \to B$という全射が存在する.
\end{proof}

以下の命題を示す.
\begin{align*}
    {}&
    \Exists{\{B_i\}_{i \in I}}
    \lbra{Y=\bigcup_{i \in I} \Spec B_i} \land \lbra{\Forall{i \in I} f^{-1}(\Spec B_i)\text{ :: locally $B_i$-fin.gen. scheme}} \\
    \iff&
    \Forall{\Spec A \subseteq X} f^{-1}(\Spec A)\text{ :: locally $A$-fin.gen. scheme}
\end{align*}

下から上は自明である.上から下を示そう.

$U=\Spec A \subset X, V_i=\Spec B_i$とする.
$U \cap V_i$の各点$P$に対し,
\[ P \in D_{B_i}(b_{ij})=D_A(a_{ij}) \subseteq U \cap V_i \]
であるような$b_{ij} \in B_i, a_{ij} \in A$が取れる.
$P$を動かせば,このようにして$U$が被覆できる.
\[ U=\bigcup_{i,j} D_{B_i}(b_{ij})=\bigcup_{i,j} D_A(a_{ij}). \]
仮定より,各$V_i$は$\{ \Spec C_{ik} \}_{i,k}$で被覆され,
これらの$C_{ik}$は有限生成$B_i$代数
\footnote{$\phi_{ik}=\left((f|_{\Spec C_{ik}})^{\#}\right)_{\Spec B_i}$で代数とみなす.}
であるようにとれる.

Lemma (Preimage of POS is POS)より,$c_{ijk} \in C_{ik}$が存在し,
以下のようになる.
\[ f^{-1}U=\bigcup_{i,j} f^{-1}D_{B_i}(b_{ij})=\bigcup_{i,j} \bigcup_{k} D_{C_{ik}}(c_{ijk}). \]
$D_{C_{ik}}(c_{ijk})=\Spec (C_{ik})_{c_{ijk}}$であり,
$(C_{ik})_{c_{ijk}}$は有限生成$(B_i)_{b_{ij}}$代数.
これは有限生成代数の定義から存在する全射$B[x_1,\dots,x_n] \to C_{ik}$の両辺を局所化
\footnote{$C_{ik}$が$\phi_{ik}$による$B_i$代数であることと$c_{ijk}=\phi_{ik}(b_{ij})$を用いて計算する.}
すれば分かる.
$(B_i)_{b_{ij}} \cong A_{a_{ij}}$(Nike's Lemmaの最後の文)と最後のLemmaより,
$(C_{ik})_{c_{ijk}}$は有限生成$A$代数.

以上より,$f^{-1}\Spec A$は$\Spec (C_{ik})_{c_{ijk}}$で被覆され,
各$(C_{ik})_{c_{ijk}}$は有限生成$A$代数である.

\section{Definition(s) of Quasi-Compact Morphism.} %% Ex3.2 
以下を示す.
\begin{align*}
    {}&
        \Exists{\{B_i\}_{i \in I}}
        \lbra{Y=\bigcup_{i \in I} \Spec B_i} \land \lbra{\Forall{i \in I} f^{-1}(\Spec B_i)\text{ :: quasi-compact.}} \\
    \iff&
        \Forall{\Spec A \subseteq Y} f^{-1}(\Spec A)\text{ :: quasi-compact.}
\end{align*}

まず$\Spec A=\bigcup_{i,j} D_{B_i}(b_{ij})$となるように$b_{ij}$をとる.
Ex2.13bより$\Spec A$はquasi-compactだから$b_{ij}$は有限個でよい.
$f^{-1}\Spec B_i$はopen subschemeだから,
$f^{-1}\Spec B_i=\bigcup_{i,k} \Spec C_{ik}$なる$C_{ik}$がある.
仮定より$f^{-1}\Spec B_i$はquasi-compactであるから$C_{ik}$は有限個.
これにEx3.1の中で示したLemma (Preimage of POS is POS)を用いると以下のようになる.
\[ f^{-1}\Spec A=\bigcup_{i,j} f^{-1}D_{B_i}(b_{ij})=\bigcup_{i,j} \bigcup_{k} D_{C_{ik}}(c_{ijk}). \]
確認したとおり組$(i,j,k)$は高々有限の組み合わせしか無い.
Ex2.13の証明にあるとおり,$D_{C_{ik}}(c_{ijk})$はquasi-compactだから,
$f^{-1}\Spec A$はquasi-compactな開集合の有限和.
よって$f^{-1}\Spec A$もquasi-compact.

\section{ } %% Ex3.3 

\section{ } %% Ex3.4 

\section{ } %% Ex3.5 

\section{ } %% Ex3.6 

\section{ } %% Ex3.7 

\section{ } %% Ex3.8 

\section{ } %% Ex3.9 

\section{ } %% Ex3.10 

\section{ } %% Ex3.11 

\section{ } %% Ex3.12 

\section{ } %% Ex3.13 

\section{ } %% Ex3.14 

\section{ } %% Ex3.15 

\section{ } %% Ex3.16 

\section{ } %% Ex3.17 

\section{ } %% Ex3.18 

\section{ } %% Ex3.19 

\section{ } %% Ex3.20 

\section{ } %% Ex3.21 

\section{ } %% Ex3.22 

\section{ } %% Ex3.23 

\end{document}
