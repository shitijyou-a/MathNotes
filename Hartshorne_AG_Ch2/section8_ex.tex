\documentclass[a4paper]{jsarticle}
\usepackage{../math_note, exercise}
\usepackage[all]{xy}
\renewcommand{\thesection}{Ex8.\arabic{section}}

\newcommand{\der}[1][\!]{\mathrm{d}_{#1}\,}
\newcommand{\Der}{\Omega}
\newcommand{\shCano}{\omega}
\newcommand{\pbundle}{\mathbb{P}}

\begin{document}
\section{Strengthen Some Results in the Text.} %% Ex8.1 

\section{$0 \to \shO_X \to \shE \to \shE' \to 0$.} %% Ex8.2 
    $X$ :: variety of dimension $n$ over $k$,
    $\shE$ :: locally free sheaf of rank $>n$,
    $V^{\#} \subset \Gamma(X,\shE)$ ::
    $k$-vector space of global sections which generate $\shE$
    とする.
    $X$ :: varietyより$X$ :: connectedなので$\shE$のrankは$X$全体で一定である.
    $\rank \shE=r (>n)$としておこう.

    \begin{Claim}
        ある$s \in V$について次が成立する.
        \[ \Forall{x \in X} s_x \not \in \I{m}_x \shE_x. \]
    \end{Claim}

    \paragraph{Convensions and Notations.}
    $X$のclosed point全体を$X^+$と書く.
    Ex3.14より,これはdense in $X$.
    また,$d=\dim_k V^{\#}, V=\proj_k^{d-1}$とし,
    $V^+=(V^{\#}-\{0\})/k^*$を$V$のclosed pointsと同一視する.
    この同一視の仕方はProp7.7やdual projective spaceと同じである.
    $\dim_k V^{\#}-1=\dim V$に注意.
    $V^{\#}$のsubspaceも同様に$V$のsubspaceとみなす.

    \paragraph{Definition of $B, B^+$.}
    $B \subseteq X \times_k V$を次のように置く.
    \[ B=\bigcap_{s \in V^{\#}} \pr_1^{-1}(\{ x \in X \mid s_x \in \I{m}_x\shE_x \}). \]
    $B$は$X \times V$のclosed subschemeである.
    ($\{\}$部分がclosedであることはEx2.16を参照.)
    $B$にはreduced structureを与えておく.
    $\pr_1|_{B}: B \to X$を$p_1$と略す.
    $B$のclosed points :: $B^+$は次のような集合である.
    \[ B^+=\{ (x, s) \in X^+ \oplus V^+ \mid s_x \in \I{m}_x\shE_x \}. \]

    \paragraph{Plot.}
    主張は,$\pr_2(B) \not \supseteq V^+$と言い換えられる.
    (詳細は後ほど.)
    これには$B$の次元が$V$の次元より小さいことを言えば良い.
    $B$の次元はEx3.22の結果を用いればそのfiber :: $B_x$から計算できる.
    全ての$x \in X$について$\dim B_x$を計算することは難しい.
    しかし少し妥協して,$x \in X^+$についての$\dim B_x$を計算することは出来る.
    この場合でもEx3.22cの結果を用いて$\dim B_x$が計算できる.

    \paragraph{Definition of $\phi_x$.}
    $x \in X$について次の写像を考える.
    \begin{defmap}
        \phi_x:& V^{\#}& \to& \shE_x \otimes_k k(x) \\
        {}& s& \mapsto& s_x \otimes 1
    \end{defmap}
    これが$k$-linear mapであることは明らか.
    $k(x):=\shO_x/\I{m}_x$より$\shE_x \otimes_k k(x) \iso \shE_x/\I{m}_x \shE_x$.
    このことと$\phi_x$の定義の仕方から,
    $\ker \phi_x=\{ s \in V^{\#} \mid s_x \in \I{m}_x\shE_x \}.$
    
    \paragraph{$\phi_x$ for $x \in X^+$.}
    この段落では$x \in X^+$とする.
    すると$k(x)=k$
    \footnote
    {
        $X$ :: varietyより,
        $k$ :: algebraically closed fieldかつ$X$ :: finite type / $k$.
        $A=k[x_1,\dots,x_n], \I{a} \subseteq A$とし,
        $\I{m}/\I{a} \in \Spec A/\I{a} \subseteq X$が$x$に対応する極大イデアルだとする.
        ここで$\I{m}$は$A$の極大イデアル.
        $S=A-\I{m}$とすると
        \[
            k(x)
            =\frac{S^{-1}(A/\I{a})}{S^{-1}(\I{m}/\I{a})}
            \iso S^{-1}\left( \frac{A/\I{a}}{\I{m}/\I{a}} \right)
            \iso S^{-1}(A/\I{m}).
        \]
        $A/\I{m} \iso k$は体だから,これは$k(x) \iso k$.
    }
    なので$\shE_x \otimes_k k(x) \iso \shE_x$.
    また$\shE_x \otimes_k k(x) \iso \shE_x/\I{m}_x\shE_x$.
    さらに$V^{\#}$ :: global generators of $\shE$であるから,
    $\phi_x$はsurjective.
    なので$x \in X^+$について$\dim \ker \phi_x$が分かる.
    \[
        \dim_k \ker \phi_x
        =\dim_k V^{\#} \otimes_k k(x)-\dim_k \shE_x
        =\dim_k V^{\#}-r.
    \]

    \paragraph{Dimension of fiber :: $\dim B_x$.}
    $p_1$についての$x \in X^+$のfiber :: $B_x$のbase spaceは,
    Ex3.10 より,$\basesp B_x \homeo p_1^{-1}(x)$.
    したがって次が分かる.
    \[ \basesp B_x \cap \basesp B^+ \homeo p_1^{-1}(x) \cap \basesp B^+=\{x\} \times \ker \phi_x. \]
    ここで$\times$は集合としての直積を表す.
    よって$B_x$の次元が分かる
    \footnote
    {
        closed subscheme of $B$ :: $C$について$\dim C=\dim C \cap B^+$を示す.
        $C \cap B^+ \subset C$より$\dim C \geq \dim C \cap B^+$は明らか.
        $d=\dim C$とし,
        $C$のirreducible closed subsetが成す真の極大上昇鎖をとる:
        $Z_0 \subsetneq \dots \subsetneq Z_d.$
        closed immersion $\implies$ finite typeに注意すると,
        $Z_i$ :: finite type/$k$.
        なのでEx3.14より$Z_i \cap B^+$ :: dense in $Z_i$.
        したがって$Z_i \cap B^+=Z_j \cap B^+ \implies Z_i=Z_j$となり,
        $Z_0 \cap B^+ \subsetneq \dots \subsetneq Z_d \cap B^+$
        は$B^+$のirreducible closed subsetが成す真の上昇鎖.
        以上から$\dim C \leq \dim C \cap B^+$も成り立つ.
    }.
    \[ \dim B_x=\dim_k \ker \phi_x-1=\dim_k V^{\#}-r-1=\dim V-r. \]

    \paragraph{$p_1$ :: closed map.}
    $V \to \Spec k$はprojectiveであり,
    $V, \Spec k$共にnoetherianであるからこの射はproper.
    よってuniversally closedである.
    \[
    \xymatrix
    {
        X \times_k V \ar[r]\ar[d]_-{\pr_1}& V
            \ar[d]^-{\substack{\text{universally} \\ \text{closed}}}\\
        X \ar[r]& \Spec k
    }
    \]
    $B$ :: closedなので$B$のclosed subsetは$X$でもclosed.
    したがって
    $p_1=\pr_1|_B$ :: closed map.

    \paragraph{$p_1(B)=X$ or $B=\emptyset$.}
    $p_1(B) \supseteq X^+$とする.
    すると$p_1(B)$ :: closedより$p_1(B) \supseteq \cl_X(X^+)=X$.
    次に$p_1(B) \not \supseteq X^+$とする.
    すると上で述べたこと
    (全ての$x \in X^+$について$\dim p_1^{-1}(x)$が等しいこと)
    から,結局$p_1(B) \cap X^+=\emptyset$が分かる.
    $p_1(B)$が空でないと仮定しよう.
    すると$p_1$ :: closed mapより,
    $x \in p_1(B)$なら$\cl_X(\{x\}) \subseteq p_1(B)$.
    $\cl_X(\{x\})$はclosed pointを含むので矛盾が生じる.
    よって$p_1(B) \not \supseteq X^+$ならば$p_1(B)=\emptyset$.
    これは$B=\emptyset$を意味し,
    さらにこれは$0$を除く全ての$V^{\#}$の元がclaimの条件を満たすことを意味する.
    \underline{以下,$B \neq \emptyset$と仮定する.}

    \paragraph{$p_1^{-1}(x)$ :: irreducible.}
    任意のclosed point :: $x \in X^+$について$p_1^{-1}=(\ker \phi_x-\{0\})/k^*$.
    これはprojective linear spaceだからirreducible.

    \paragraph{$B$ :: irreducible.}
    以上から$B$ :: irreducibleが分かる.
    $B$が二つの閉集合$C_1, C_2$の和であったとすると,
    $x \in X^+$について$p_1^{-1}(x)$は次のように書ける.
    \[ p_1^{-1}(x)=(C_1 \cap \pr_1^{-1}(x)) \cup (C_2 \cap \pr_1^{-1}(x)). \]
    これはirreducibleだから,
    $C_1 \cap \pr_1^{-1}(x)$か$C_2 \cap \pr_1^{-1}(x)$に一致する.
    $x_1, x_2 \in X^+$について次のようになっていたと仮定しよう.
    \[
        p_1^{-1}(x_1)=C_1 \cap \pr_1^{-1}(x),~~~
        p_1^{-1}(x_2)=C_2 \cap \pr_1^{-1}(x).
    \]
    すると,$x_1 \in, x_2 \not \in p_1(C_1)$となる.
    $p_1(C_2)$も同様.
    すなわち$p_1(C_1), p_1(C_2)$は$p_1(B)(=X)$空でないの真の閉集合である.
    しかし$X=p_1(B)=p_1(C_1) \cup p_1(C_2)$であり$X$ :: irreducibleであるから,
    これはありえない.
    よって任意の$x \in X^+$について
    $p_1^{-1}(x)=C_1 \cap \pr_1^{-1}(x)$(あるいは$=C_2 \cap$...)となる.
    両辺で$\bigcup_{x \in X^+}$として
    \[ p_1^{-1}(X^+)=C_1 \cap p_1^{-1}(X^+). \]
    $p_1^{-1}(X^+)=(X^+ \times V) \cap B \supset B^+$であり,
    $B^+$ :: dense in $B$.
    $B^+ \cap C_1$ :: dense in $C_1$もEx3.14から得られるので,
    両辺の$B$での閉包を取って$B=C_1$.
    したがって$B$ :: irreducible.

    \paragraph{Dimension of $B$.}
    $B$ :: integral \& finite type/$k$ ($\implies$ variety/$k$)なので,
    Ex3.22cから次が成り立つ:
    $x \in U$ならば$\dim B_x=\dim B-\dim X$,
    となる$U$ :: open dense subset in $X$が存在する.
    $U$ :: non-empty open subsetと$X^+$ :: denseから,$U \cap X^+ \neq \emptyset$.
    $x \in X^+$であるときの及び開集合$\dim B_x$が既に分かっているから,
    $\dim B$も分かる.
    \[ \dim B=\dim B_x+\dim X=\dim V-r+n. \]
    $r>n$なので,$\dim B<\dim V$.

    \paragraph{$\pr_2(B) \supseteq V^+ \implies \dim B \geq \dim V$.}
    $\pr_2(B) \supseteq V^+$としよう.
    $B^+$の場合と同様に$\dim V^+=\dim V$.
    ch I, Ex1.10より,$\dim U=\dim V$を満たす
    affine open subset of $V$ :: $U$がとれる.
    適当に$\pr_1(B)$からもaffine open subset ::  $U'$をとると,
    $X, V$共にfinite type /$k$だから,
    ch I, Ex3.15 (Products of Affine Varieties)が使える.
    よって$\dim U \times U'=\dim U+\dim U' \geq \dim U=\dim V$.
    $U \times_k U' \subset B$だから$\dim B \geq \dim V$
    
    \paragraph{Complete proof of the claim.}
    今はこれの対偶が成立する.
    すなわち,$s \in V^+-\pr_2(B)$が存在する.
    この$s$と任意の$x \in X$について$s_x \not\in \I{m}_x\shE_x$が成り立つ.

    \paragraph{An exact sequence.}
    $\Phi$を以下で定める.
    \begin{defmap}
        \Phi:& \shO_X& \to& \shE  \\
        {}& \sect{U}{\sigma}& \mapsto& \sect{U}{(s|_U) \cdot \sigma}
    \end{defmap}
    これの$x \in X$におけるstalkを見ると,
    $\Phi_x: \sigma_x \mapsto s_x \cdot \sigma_x$と成っている.
    $\shE_x \iso \shO_x^{\oplus r}$かつ$\shO_x$ :: domainより,
    $\Ann(\shE_x)=0$.
    そして$s_x \not\in \I{m}_x\shE_x$から,$s_x \neq 0$.
    なので$\Phi_x$は,したがって$\Phi$はinjective.
    よって$\shE'=\coker \Phi$とおくと以下はexact sequence.
    \[ 0 \to \shO_X \to \shE \to \shE' \to 0. \]
    
    \paragraph{$\shE'$ :: locally free.}
    $\shE'$がlocally freeであることを示そう.
    Ex5.7bから,任意の点におけるstalkがfreeであることを示せば十分.
    以下,$\shE_x=\shO_x^{\oplus r}$($\iso$でなく$=$)とする.
    点$x \in X$について
    \[ s_x=(s_x^{(i)})_i \in \shO_x^{\oplus r}=\shE_x \]とする.
    $s_x \not\in \I{m}_x\shE_x=\I{m}_x^{\oplus r}$から,
    ある$i$について$s_x^{(i)} \not \in \I{m}_x$.
    すなわち$s_x^{(i)}$ :: unit.
    ここでは$i=0$とし,
    \[
        u
        =(s_x^{(0)})^{-1}s_x
        =\left( 1, s_x^{(2)}(s_x^{(0)})^{-1}, \dots, s_x^{(r)}(s_x^{(0)})^{-1} \right) \in s_x\shO_x
    \]
    と置く.
    すると$\shE'_x \iso \shE_x/\im \Phi_x=\shO_x^{\oplus r}/s_x\shO_x$は
    次の写像で$\shO_x^{\oplus r-1}$と同型.
    \begin{defmap}
        {}& \shO_x^{\oplus r}/s_x\shO_x& \to& 0 \oplus \shO_x^{\oplus r-1} \\
        {}& (t^{(j)})_j \bmod s_x\shO_x& \mapsto& (t^{(j)})_j-t^{(0)}u
    \end{defmap}
    well-definedであることは明らか.
    逆写像は次のもの.
    \begin{defmap}
        {}& \shO_x^{\oplus r-1}& \to& \shO_x^{\oplus r}/s_x\shO_x \\
        {}& t & \mapsto& (0 \oplus t) \bmod s_x\shO_x
    \end{defmap}

    \subsubsection{$B$の別構成.}
    $d+1=\dim_k V^{\#}$とし,$\shV=(V^{\#})\sidetilde$とする.
    $V^{\#} \iso k^{\oplus d+1}$から
    $\shV$は$\rank \shV=d+1$のlocally free sheafとなる.
    そして全射$\shV \otimes_k \shO_X \to \shE$が
    $\sect{U}{s} \otimes \sect{U}{a} \mapsto \sect{U}{sa}$の様に構成できる
    \footnote
    {
        $\shO_X$が$k$-moduleであることは次のように分かる.
        今,$f: X \to \Spec k$が存在するので
        $\shO_{\Spec k} \to f^*\shO_X$が存在する.
        これのadjoint :: $f^{-1}\shO_{\Spec k} \to \shO_X$を考えれば,
        開集合$U \subseteq X$について$\shO_X(U)$が$k$-moduleであることが分かる.
        また,ここで書いた$\shV \otimes_k \shO_X \to \shE$の定義は
        presheaf :: $U \mapsto \shV(U) \otimes_k \shO_X(U)$からの
        morphismなのでsheafificationが必要である.
    }.
    これの$\ker$を$\shB$とおく.
    \[ \xymatrix{ 0 \ar[r]& \shB \ar[r]& \shV \otimes \shO_X \ar[r]& \shE \ar[r]& 0 } \]
    構成から$\shB$ :: locally freeと$\rank \shB=d+1-r$が分かる(?).
    双対をとる.(すなわち$\shHom(-, \shO_X)$で写す.)
    \[ \xymatrix{ 0 \ar[r]& \check{\shE} \ar[r]& \check{\shV} \otimes \shO_X \ar[r]& \check{\shB} \ar[r]& 0 } \]
    全射$\check{\shV} \otimes \shO_X \to \check{\shB}$から,
    injective $X$-morphism :: $\pbundle(\check{\shB}) \to \proj_k^{d} \times X$が誘導される(?).
    ここでの$\pbundle(\check{\shB})$が$B$である(?).
    構成の仕方から,$\dim B=\rank \check{\shB}-1$.

    \subsubsection{$\shE'$ :: locally freeの別証明.}
    任意の点$x \in X$におけるstalkを考える.
    \[ \xymatrix{ 0 \ar[r]& \shO_x \ar[r]^-{\times s_x}& \shE_x \ar[r]& \shE'_x \ar[r]& 0 } \]
    これを$\otimes_{\shO_x} k(x)$で写し,
    $k(x)$-moduleのexact sequenceにする.
    \[ \xymatrix{ \shO_x \otimes k(x) \ar[r]^-{\times (s_x \otimes 1)}& \shE_x \otimes k(x) \ar[r]& \shE'_x \otimes k(x) \ar[r]& 0 } \]
    同型で書き換える.
    \[ \xymatrix{ k(x) \ar[r]^-{\times (s_x)\sidebar}& \shE_x/\I{m}_x\shE_x \ar[r]& \shE'_x \otimes k(x) \ar[r]& 0 } \]
    ただし$(s_x)\sidebar=s_x \bmod \I{m}_x\shE_x$.
    これは$s_x \not \in \I{m}_x\shE_x$から,$0$でない.
    したがって左の写像は$1 \in k(x)$を非ゼロ元に写す.
    このexact sequenceは$k(x)$-moduleのものだったから,
    左の写像はinjective.
    よって次が分かる.
    \[ \dim_{k(x)} \shE'_x \otimes k(x)=\dim_{k(x)} \shE_x \otimes k(x)-\dim_{k(x)} k(x)=r-1. \]
    すなわち$\dim_{k(x)} \shE'_x \otimes k(x)$は$x \in X$について定数関数.
    Ex5.8より,$\shE'$ :: locally freeと分かる.

\section{Product Schemes.} %% Ex8.3 
    \subsection{$\Der_{X \times_S Y/S} \iso \pr_X^* \Der_{X/S} \oplus \pr_Y^* \Der_{Y/S}$.}
    $S$ :: scheme, $X,Y$ :: scheme /$S$とする.
    Thm8.10より,$\Der_{X \times Y/Y} \iso \pr_X^*\Der_{X/S}$が分かる.
    これとThm8.11を合わせて次の完全列が得られる.
    \[
    \xymatrix
    {
        \pr_Y^* \Der_{Y/S} \ar[r]&
        \Der_{X \times Y/S} \ar[r]&
        \pr_X^*\Der_{X/S} \ar[r]&
        0.
    }
    \eqno{(*)}
    \]
    $X$と$Y$を交換したものと合わせて次の図式を得る.
    これは$\shO_{X \times Y}$-moduleの図式である.
    \[
    \xymatrix
    {
        {} &
        \pr_X^* \Der_{X/S} \ar[r] \ar[d]^-{\bar{\gamma}}&
        \Der_{X \times Y/S} \ar[r] \ar@{=}[d]^-{\id{}}&
        \pr_Y^*\Der_{Y/S} \ar[r]&
        0 \\
        0 &
        \pr_X^* \Der_{X/S} \ar[l]&
        \Der_{X \times Y/S} \ar[l]&
        \pr_Y^*\Der_{Y/S} \ar[l]&
        {}
    }
    \]
    この図式において$\bar{\gamma}$は
    $\Der_{X \times Y/S}$を経由する射の合成である.
    $\gamma=\id{\pr_Y^* \Der_{Y/S}}$が示せれば,
    $\alpha$ :: inj \& splitが得られる.
    これは$X \times Y$のopen affine coverをとって
    localに調べれば良い.
    $\Spec R \subseteq S, \Spec A \subseteq X, \Spec B \subseteq Y$を任意にとり,
    $C=A \otimes_R B$とする.
    図式全体を$\Gamma(\Spec C,-)$で写す.
    $\Der$の構成から,これは次のように成る.
    これは$C$-moduleの図式である.
    \[
    \xymatrix
    {
        {} &
        \Der_{A/S} \otimes_A C \ar[r]^-{\alpha} \ar[d]^-{\gamma}&
        \Der_{C/S} \ar[r] \ar@{=}[d]^-{\id{}}&
        \Der_{B/S} \otimes_B C \ar[r]&
        0 \\
        0 &
        \Der_{A/S} \otimes_A C \ar[l]&
        \Der_{C/S} \ar[ld]^-{\beta}&
        \Der_{B/S} \otimes_B C \ar[l]&
        {} \\
        {} & \Der_{C/B} \ar[u]^-{\iso}
    }
    \]
    それぞれの写像は次のように定義される
    (Matsumura, p.193 \& Eisenbud, Prop16.4).
    \begin{defmap}
        \alpha:& [\der[A/S] a] \otimes c& \mapsto& [\der[C/S] (a \otimes 1_B)] \cdot c \\
        \beta:& \der[C/S] c& \mapsto& \der[C/B] c \\
        \iso:& \der[C/B] (a \otimes b)& \mapsto& [\der[A/S] a] \otimes (1_A \otimes b)  \\
    \end{defmap}
    よって$\gamma$は次のように成る.
    \[
        [\der[A/S] a] \otimes c
        \mapsto
        [\der[C/S] (a \otimes 1_B)] \cdot c
        \mapsto
        [\der[C/B] (a \otimes 1_B)] \cdot c
        \mapsto
        ([\der[B/S] a] \otimes 1_C) \cdot c
        =
        [\der[A/S] a] \otimes c.
    \]
    以上より$\gamma=\id{}$が示された.

    \subsection{$\shCano_{X \times Y} \iso \pr_X^*\shCano_{X} \otimes \pr_Y^*\shCano_{Y}$.}
    $X, Y$ :: nonsingular varieties over a field $k$とする.
    $d_X=\dim X, d_Y=\dim Y$とする.
    この時Thm8.15より,
    $\Der_{X/k}, \Der_{Y/k}$は
    それぞれ$\rank=d_X, d_Y$のlocally free sheafである.
    また(a)の完全列$(*)$より,
    $\rank \Der_{X \times_k Y/k}=d_X+d_Y$
    \footnote
    {
        各点でのstalkをとって$\rank$がadditiveであることを使えば分かる.
    }.

    Ex5.16dを(a)の完全列$(*)$に用いれば,
    \[
        \shCano_{X \times Y}
        =\bigwedge^{d_X+d_Y}\Der_{X \times Y/k}
        \iso \left( \bigwedge^{d_X}\pr_X^* \Der_{X/k} \right)
            \otimes \left( \bigwedge^{d_X}\pr_Y^* \Der_{Y/k} \right).
    \]
    Ex5.16eより$\pr_X^*, \pr_Y^*$はそれぞれ$\bigwedge$と交換できる.
    よって$\shCano_{X \times Y} \iso \pr_X^*\shCano_{X} \otimes \pr_Y^*\shCano_{Y}$.

    \subsection{An Example that Gives $p_g \neq p_a$.}
    $Y \subset \proj^2_k$をnon-singluar cubic curveとする.
    さらに$Y \times_k Y$をSegre embeddingで
    $\proj^8$に埋め込んだものを$X$とする.

    Example8.20.3より,$\shCano_Y \iso \shO_Y(0)=\shO_Y$.
    したがって(b)より$p_g(X)$が計算できる.
    \[
        p_g(X)
        =\dim_k \Gamma(X, \pr_1^*\shO_Y \otimes \pr_2^* \shO_Y)
        =\dim_k \Gamma(X, \shO_X)
        =\dim_k k
        =1.
    \]
    ここでEx5.11:
    $\shO_X(1) \iso \pr_1^*\shO_Y(1) \otimes \pr_2^* \shO_Y(1)$
    (両辺に逆元をテンソルすれば利用した同型が得られる)と
    Ex4.5dを順に用いた.

    I, Ex7.2bより$p_a(Y)=\frac{1}{2}(3-1)(3-2)=1$.
    同じくI, Ex7.2eより$p_a(X)$が計算できる.
    \[ p_a(X)=(p_a(Y))^2-2p_a(Y)=-1. \]

%    \subsubsection{Direct Calc of $\shCano_Y$.}
%    $Y \cap \zerosp(x)^c$のcoordinate ringを$R=k[x,y]/(f(x,y))$とする.
%    加群$\Der_{R/k}$を求めよう.
%    $\bar{x}=x \bmod (f(x,y)), \bar{y}=y \bmod (f(x,y))$とおくと,
%    $\Der_{R/k}=R \der\bar{x}+R \der{}\bar{y}$となる(Matsumura, p.192).
%    $f(\bar{x}, \bar{y})=0$から,
%    $\Der_{R/k}$が持つ唯一の関係式が定まり,
%    次のように成る.
%    \[
%        \Der_{R/k} \iso
%        \frac{ k[x,y]\der x+k[x,y] \der y}
%        {\left( \partial_x f(x,y) \cdot \der x+\partial_y f(x,y) \cdot \der y \right)}.
%    \]
%    これはヤコビ行列で定まる写像の$\coker$である.
%    $Y=\Spec k[x,y]/(f)$はnon-singularであったからヤコビ行列の$\rank$は$1$.
%    よって$\rank \Der_{R/k}=1$.
%    $\dim Y=1$なので$\shCano_Y=\Der_{X/k}=(\Der_{R/k})\sidetilde$.

\section{Complete Intersections in $\proj^n$.} %% Ex8.4 

\section{Blowing Up a Nonsingular Subvariety.} %% Ex8.5 

\section{The Infinitesimal Lifting Property.} %% Ex8.6 

\section{Classifying Infinitesimal Extension: One Case.} %% Ex8.7 

\section{Plurigenera and Hodge Numbers are Birational Invariants.} %% Ex8.8 

\end{document}
