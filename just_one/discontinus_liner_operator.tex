\documentclass[a4j]{jarticle}
\usepackage{../math_note}

\title{不連続線形作用素の構成}

\begin{document}
\maketitle
Wikipediaの「不連続線形作用素」のページに
「完備でない空間においては不連続線型写像の例を構成するのは容易である。」
と書いてあるのを見つけた.
けれど,それに続く具体例の構成はすぐさま他の完備でない空間で
使えるものではない.
そこで,このノートではその構成方法をもう少し一般的に与える.
ヒントは基底変換である.

ノルム空間$X$を考える.
まず,$X$が加算濃度の正規化された基底$\{ \phi_n \}_{n \in \N}$を持つとする.
すると
\[ B:=\{b_n\}_{n \in \N}=\left\{\frac{1}{n} \phi_n \right\}_{n \in \N} \]
は0に収束するCauchy列でありしかも$X$の基底である.
これに対し,
0に収束しないCauchy列$A:=\{ a_n \}_{n \in \N}$をとる
\footnote{例えば$\{ \phi_n \}_{n \in \N}$とは異なる正規直交基底$\{ \psi_n \}_{n \in \N}$をとり,$a_n=e^{1/n} \psi_n$とする.}.
これは明らかに$B$と線形独立である.

さて,線形作用素$T:X \to X$を
\[ Tb_n=a_n \]
となるように取る.
$B$は基底だったから,これは正しく定義できる.
そうすれば$\lim_{n \to \infty} Tb_n=\lim_{n \to \infty} a_n \neq 0$が成立する.
$\lim_{n \to \infty} b_n=0$であったから,$T$は不連続線形作用素.

以上で述べた不連続線形作用素の構成はノルム空間$X$が加算濃度の基底を持つことを仮定している.
これには$X$が以下の条件を満たすことが必要である.
\begin{enumerate}[i)]
    \item 完備でない
    \item 有限次元でない
\end{enumerate}
完備であればBarelのカテゴリー定理から$X$は加算基底を持たない.
有限次元であれば非完備にならない.

Wikipediaの「不連続線形作用素」のページにある具体例は
滑らかな実数値函数全体の成す空間を考えている.
上の構成法になぞらえてこの具体例を読むと,
基底$B$を正規直交関数系$\{ 1, \sin x, \sin 2x, \dots \}$から作り,
それをもうひとつの正規直交関数系$\{ 1, \cos x, \cos 2x, \dots \}$から作る点列$A$へ写している.
この時に身近な線形作用素である微分作用素を用いるために
少々不自然な$B,A$を構成している.
\end{document}
