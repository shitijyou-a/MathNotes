\documentclass[a4paper]{jsarticle}
\usepackage{../math_note, exercise}
\usepackage[all]{xy}

\renewcommand{\thesection}{Ex4.\arabic{section}}
\setcounter{section}{8}

\newcommand{\cond}{\mathcal{C}}

\begin{document}
\section{Suitable Stereographic Projection Gives Birational Map.}
    $X$ :: projective variety in $\proj^n_k$とし,
    $r=\dim X \leq n-2$とする.
    また$H$ :: hyperplane in $\proj^n$とし,
    適宜$\proj^{n-1}$と同一視する.
    適切に点$P \not \in X$をとれば,
    $P$から$H$へのstereographic projection :: $\pi : X \to \proj^{n-1}$が
    $X$と$\pi(X)$の間のbirational mapになることを示す.

    $I=\defsp(X)$とする.
    $\bar{x}_i=x_i \bmod I, \bar{y}_i=\frac{\bar{x}_i}{\bar{x}_0}$とすると,
    $K:=K(X)=k(\bar{y}_0,\dots,\bar{y}_n)$.
    Thm4.8より拡大$K/k$はfinitely and separably generated.
    Thm4.7より,$\{\bar{y}_i\}_{i=1}^n$は
    separating transcendence baseを部分集合として含む.
    そこで番号を付け替えて,
    $\{\bar{y}_i\}_{i=1}^n$に含まれる
    separating transcendence baseを$\{\bar{y}_i\}_{i=1}^{r}$としよう.
    baseの濃度が$r(=\dim X)$であることはThm3.2による.
    そして以下の拡大はfinite generated extensionである.
    \[ k(\{\bar{y}_i\}_{i=1}^{n})/k(\{\bar{y}_i\}_{i=1}^{r}) \]
    Thm4.6から,この拡大は以下のような元$\bar{\eta}$で生成することが出来る.
    \[
        \bar{\eta}=\eta \bmod I
        \mwhere
        d \geq 0,~
        \eta_{r+1},\dots,\eta_{n} \in k[\{x_i\}_{i=1}^{r}]^d,~
        \eta=\frac{1}{x_0^d}\sum_{i=r+1}^{n} \eta_i x_i.
    \]

    $\pi(X) \subseteq H$のfunction fieldを$L$とする.
    $\pi$から誘導される準同型(TODO)$\pi^*$を次で定める.
    \begin{defmap}
        \pi^*:& L& \to& K \\
        {}& f& \mapsto& f \circ \pi
    \end{defmap}
    $\pi$は$Q \in X$を直線 :: $tP+Q$と$H$の交点へ写す写像であった.
    ($P \not \in H$なので$P=1 \cdot P+0 \cdot Q$は予め除いている.)
    したがって$R \in \pi(X)$をとると$(\pi^* f)(tP+R)$は$t \in k$について定数.
    この値は$f(R)$であるから$\pi^*$は単射である.
    逆に$g \in K$から得られる関数$g(tP+Q)$が$t$について定数ならば,
    $f(R) ~(R \in \pi(X))$を$g(\pi^{-1}(R))$
    \footnote
    {
        これは$\{ g(tP+R) \mid t \in k, tP+R \in X \}$に等しい.
        単元集合なので関数$f$を定めることが出来る.
    }と置くことで
    $g=\pi^*f$となる$f \in L$が取れる.
    以上から,$K$の任意の元$g$について次の条件$\cond(g)$が成立すれば$\pi^*$は同型写像と成る:
    任意の$Q \in X$に対し$g(tP+Q)$は$t \in k$について定数である.

    さて,既に分かっている通り
    $K=k(\bar{y}_1,\dots,\bar{y}_r, \bar{\eta})$であった.
    なので$\cond(\bar{y}_1),\dots,\cond(\bar{y}_r), \cond(\bar{\eta})$の
    全てが成立すれば良い.

    引き続き$Q \in X$とする.
    $P=(p_0: \dots: p_n), Q=(q_0: \dots: q_n)$とすると
    \[ tP+Q=(tp_0+q_0, \dots, tp_n+q_n). \]
    なので$p_0=\dots=p_r=0$すなわち$P \in \zerosp(x_0,\dots,x_r)$であれば
    $\cond(\bar{y}_1),\dots,\cond(\bar{y}_r)$は成立する.
    以下,$P$はこのようにとる.

    $tP+Q \in X$であるような$t$について$\bar{\eta}(tP+Q)$は次のように成る.
    (分母を払って考えれば$Q \in X \cap \zerosp(x_0)^c$に限る必要はない.)
    \[
        q_0^d \cdot \eta(tP+Q)
        =\sum_{i=r+1}^{n} \eta_i(q_0,\dots,q_r) (tp_i+q_i)
        =\left( \sum_{i=r+1}^{n} \eta_i(q_0,\dots,q_r)p_i \right)t+\left( \sum_{i=r+1}^{n} \eta_i(q_0,\dots,q_r)q_i \right)
    \]
    この$t$の係数が任意の$Q \in X$について$0$であるような$P$が目標の点である.
    
    $X$から$\zerosp(x_{r+1},\dots,x_{n})$へ射影した像を$Z$とする.
    また$B \subseteq \affine^{r} \times \affine^{n-r}$を次のように置く.
    \[ B=\zerosa(\eta)^c \cap \pr_1^{-1}(Z). \]
    これは次のようにも書ける.
    \[ B=\left\{ (Q,P) \in \affine^{r} \times \affine^{n-r} \middle| \sum_{i=r+1}^{n} \eta_i(q_1,\dots,q_r)p_i \neq 0 \right\} \]
    したがって$\pr_2(B)$に含まれない点が我々が求める点$P$である.
\end{document}
