\section{収束に関して}
    Banach空間とその上の作用素が成す空間の点列には,
    いくつかの収束,すなわちいくつかの位相が定められる.
    ここではそれらの定義を整理し,違いを明確にしたい.
    \subsection{定義}
    弱いものから並べていく.
    以下で$\to$と書いているものは$\C$での収束である.
    \newpage
    \subsubsection{Banach空間の点列の収束}
    この節では$\spX$をノルム空間とする.
    さらに$u_n (n=1,2,\dots),~ u \in \spX$とする.
    \begin{description}
        \item[弱収束] $\Forall{f \in \spX^*} f(u_n) \to f(u_n) ~~ (n \to \infty)$.
        \item[強収束] $\|u_n-u\|_{\spX} \to 0 ~~ (n \to \infty)$.
    \end{description}
    それぞれ収束先のことを弱極限,強極限と呼び,これらは存在すれば一意である.
    (これは以下の2節でも同様.)
    弱収束は$\spX^*=B(\spX,\C)$の全ての元が連続であるような位相として最弱のものを誘導する.
    一方,強収束はノルムが定める位相での収束であり,
    この位相について$\spX^*$の元が連続であることはp.28で述べられている.

    \subsubsection{共役空間の点列の収束}
    この節では$\spX$をノルム空間とする.
    さらに$f_n (n=1,2,\dots),~ f \in \spX^*$とする.
    \begin{description}
        \item[汎弱収束] $\Forall{u \in \spX} f_n(u) \to f(u) ~~ (n \to \infty)$.
        \item[弱収束]\hspace{0.6em} $\Forall{\psi \in \spX^{**}} \psi(f_n) \to \psi(f_n) ~~ (n \to \infty)$.
        \item[強収束]\hspace{0.6em} $\|f_n-f\|_{\spX^*} \to 0 ~~ (n \to \infty)$.
    \end{description}
    $\spX$から$\spX^{**}$への自然な写像を$\kappa$としよう.
    \footnote{$\kappa$は$u$をとって$f \mapsto f(u)$という写像を返す.}
    すると,汎弱収束は以下と同値である.
    \[ \Forall{\psi \in \ran \kappa \subset \spX^{**}} \psi(f_n) \to \psi(f_n) ~~ (n \to \infty). \]
    汎弱収束が弱収束より弱いことは明らかである.
    $\spX$が反射的ならば$\ran \kappa=\spX^{**}$なので弱収束と汎弱収束は同値である.
        
    \subsubsection{作用素の点列の収束}
    $\spX$をノルム空間とし,$T_n (n=1,2,\dots),~ T \in \spB{X}{Y}$とする.
    \begin{description}
        \item[弱収束]\hspace{0.6em} $\Forall{u \in \spX} \Forall{f \in \spX^*} f(T_n u) \to f(T u) ~~ (n \to \infty)$.
        \item[強収束]\hspace{0.6em} $\|T_n u-T u\|_{\spY} \to 0 ~~ (n \to \infty)$.
        \item[一様収束] $\|T_n-T\|_{\spB{X}{Y}} \to 0 ~~ (n \to \infty)$.
    \end{description}
    $\{T_n\}_{n=1}^{\infty}$が$T$へ強収束することは,次と同値である.
    \[ \Forall{u \in \spX} \Forall{\varepsilon>0} \Exists{N \in \N}~ n>N \implies \| T_n u-T u \| < \varepsilon. \]
    $\{T_n\}_{n=1}^{\infty}$が$T$へ一様収束(ノルム収束)することは,次と同値である.
    \[ \Forall{\varepsilon>0} \Exists{N \in \N} \Forall{u \in \mathrm{UB}} n>N \implies \| T_n u-T u \| \leq \varepsilon. \]
    ただし$\mathrm{UB}$は$\spX$の閉単位球$\{u \in \spX ~|~ \|u\| \leq 1\}$である.
    つまり,$\{T_n\}_{n=1}^{\infty}$が$T$へ一様収束(ノルム収束)することは,
    $\{T_n\}_{n=1}^{\infty}$が$\mathrm{UB}$上$T$へ一様収束することと同値である.

    \subsection{例}
    \subsection{定理・命題・補題・系}
