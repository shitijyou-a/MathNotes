\documentclass[a4j]{jsarticle}
\usepackage{../../math_note}

\newcommand{\ran}{\operatorname{ran}}
\newcommand{\ind}{\operatorname{ind}}

\newcommand{\spX}{\mathscr{X}}
\newcommand{\spY}{\mathscr{Y}}
\newcommand{\spZ}{\mathscr{Z}}
\newcommand{\spM}{\mathscr{M}}
\newcommand{\spN}{\mathscr{N}}
\newcommand{\spB}[2]{B(\mathscr{#1}, \mathscr{#2})}
\newcommand{\spBX}{B(\spX)}

\begin{document}
    これは黒田成俊『関数解析』(以下,教科書)の内容を
    抜き出し,一部補完して流れを明確にしたノート/cheat-sheetである.

        \section{第1,2章/空間}
    \subsection{定義}
        \begin{Def}
            空でない集合$V$,体$k$,
            2つの演算子$+:V \times V \to V$と$\ast: k \times V \to V$の組$(V,k,+,\ast)$
            であって以下の条件を満たすもの\textbf{線形空間}と呼ぶ.
            ただし$\alpha, \beta \in k$と$u,v \in V$とする.
            \begin{description}
                \setlength{\leftskip}{0.5truecm}
                \item[V1] $V$は加法$+$について群を成す
                \item[V2] $\alpha(u+v)=\alpha u+\alpha v$
                \item[V3] $(\alpha+\beta)u=\alpha u+\beta u$
                \item[V4] $(\alpha \beta)u=\alpha (\beta u)$
                \item[V5] $1 u=u$~(ただし左辺の1は$k$の乗法単位)
            \end{description}
            スカラー倍の演算子$\ast$は略記した.
            また,加法の演算子+には
            $+_{V}:V \times V \to V$と$+_{k}: k \times k \to k$の2つがあるが,
            混同のおそれが無いためどちらも+で表した.
        \end{Def}

        \begin{Def}
            $\C$上の線形空間$V$に対し,以下の条件を満たす
            対応$\| \cdot \|: V \to \R $を\textbf{ノルム}と呼ぶ.
            ただし$\alpha \in \C$と$x, y \in V$とする.
            \begin{description}
                \setlength{\leftskip}{0.5truecm}
                \item[N1] $\|x\|=0 \iff x=0$
                \item[N2] $\|\alpha x\| =|\alpha| \|x\|$
                \item[N3] $\|x+y\| \leq \|x\|+\|y\|$
            \end{description}
            $\|x\| \geq 0$を仮定することも多いが,これは上の(N3)で$y=-x$とすれば導出できる.
            ノルムを備えた線形空間を\textbf{ノルム空間}と呼ぶ.
        \end{Def}
        これは大きさの概念を抽象化したものである.
        これを利用し,差の大きさとして標準的な距離を定めることが出来る.
        すなわち,任意の2点$x,y$の距離を$d(x,y)=\|x-y\|$とすると
        これは距離の公理を満たす.

        \begin{Def}
            $\C$上の線形空間$V$に対し,以下の条件を満たす
            対応$(\cdot, \cdot): V \times V \to \C $を\textbf{内積}と呼ぶ.
            ただし$\alpha \in \C$と$u, v, w \in V$が取れるものとする.
            \begin{description}
                \setlength{\leftskip}{0.5truecm}
                \item[I1] $(u,v)=\overline{(v,u)}$
                \item[I2] $(\alpha u, v) =\alpha (u,v)$
                \item[I4] $(u+v,w)=(u,w)+(v,w)$
                \item[I5] $(u,u) \geq 0$
                \item[I6] $(u,u)=0 \iff u=0$
            \end{description}
            内積を備えた線形空間を\textbf{内積空間}(あるいは前Hilbert空間)と呼ぶ.
            また,\[ \|x\|=(x,x)^{1/2} \]と置くと,これはノルムの定義を満たす.
            したがって内積空間はノルム空間とすることが出来る.
            このノルムは\textbf{内積が定めるノルム}を呼ばれる.
        \end{Def}
        ノルムが大きさの概念を抽象化したものであるのに対し,
        内積は大きさと角度の概念を抽象化したものである.
        別の言い方をすれば,これはより一般的な「近さ」を抽象化したものである.
        % グラフにノルムを入れることは出来るか?

        \begin{Def}
            ノルム空間$(V, \| \cdot \|)$について,
            点列$\{ u_n \}_{n \in \N}$が以下の論理式を満たすとき,
            点列$\{ u_n \}$を\textbf{Cauchy列}と呼ぶ.
            \[ \Forall{\varepsilon>0} \Exists{N \in \N} [ \Forall{m,n>N} \| u_m-u_n \|<\varepsilon ]. \]
        \end{Def}

        \begin{Def}
            空間$X$に含まれる任意のCauchy列が$X$の元に収束するとき,空間$X$は\textbf{完備}であると呼ばれる.
        \end{Def}

        \begin{Def}
            完備なノルム空間を\textbf{Banach空間},
            内積が定めるノルムについて完備な内積空間を\textbf{Hilbert空間}と呼ぶ.
        \end{Def}

        \begin{Def}
            線形空間$\spX$の空でない部分集合$\mathscr{M}$が
            \[ u, v \in \mathscr{M},~ \alpha, \beta \in \C,~ \alpha u+ \beta v \in \mathscr{M} \]
            を満たすとき,$\mathscr{M}$は$\spX$の(線形)\textbf{部分空間}と呼ぶ.
            特にノルム空間の部分空間であって閉集合であるものを\textbf{閉部分空間}と呼ぶ.
        \end{Def}
        これに関しては後の定理\ref{them1:28}, 系\ref{cor1:29}が重要である.

        \begin{Def}
            $\C$上の線形空間として有限次元を持つノルム空間を\textbf{有限次元ノルム空間}と呼ぶ.
        \end{Def}
        これに関しては定理\ref{them1:37}が重要である.
        すなわち,有限次元ノルム空間は完備である.

        \begin{Def}
            Hilbert空間$\mathscr{H}$の空でない\kenten{閉部分集合}$\mathscr{M}$に対して,
            \[ \mathscr{M}^{\perp}=\{ u \in \mathscr{H} \mid \Forall{v \in \mathscr{M}} (u,v)=0\} \]
            を$\mathscr{M}$の\textbf{直交補空間}と呼ぶ.
        \end{Def}

        \begin{Def}
            任意の有界閉部分集合がコンパクトであるようなノルム空間を\textbf{局所コンパクト}であると言う.
        \end{Def}

    \subsection{例}
        \begin{Example}[例1.17, p.8]
            閉区間$[a,b]( \subset \R)$から$\C$への連続な関数全体を$C[a,b]$と表す.
            ノルムを\[ \|u\|=\displaystyle{\sup_{x \in [a,b]}{|u(x)|}} \]として入れると,
            $C[a,b]$はBanach空間である.
        \end{Example}
        上のことを証明する際は教科書の例1.9(p.5), 1.13(p.6)も参照のこと.

        \begin{Example}[例1.18, p.9]
            点列$\{u_n\}_{n \in \N}$で
            $\sum_{n \in \N}{|u_n|}<\infty$であるもの全体の集合を$l^1$と書く.
            スカラー集合は$\C$,線形演算は項毎のものとし,ノルムを\[ \|u\|=\sum_{n \in \N}{|u_n|} \]で定義する.
            この時,$l^1$はBanach空間である.
        \end{Example}
        \begin{Example}[例1.19, p.10]
            点列$\{u_n\}_{n \in \N} \subset \C$で
            $\sup_{n \in \N}{|u_n|}<\infty$であるもの全体の集合を$l^{\infty}$と書く.
            $l^{\infty}$に於けるノルムを\[ \|u\|=\sup_{n \in \N}{|u_n|} \]で定義する.
            この時,$l^{\infty}$はBanach空間である.
        \end{Example}
        \begin{Example}[例1.33, p.19]
            閉区間$[a,b] \subset \R$について,
            $[a,b]$から$\C$への$C^m$級関数全体を$C^m[a,b]$と表す.
            これは$C[a,b]$の部分空間であり,$C[a,b]$の中で稠密.
            $C^m[a,b]$にノルムを
            \[ \|u\|=\sum_{j=0}^{m}{ \sup_{x \in [a,b]}{\left| \frac{d^j u}{dx^j}(x) \right|} } \]
            として入れると,
            $C^m[a,b]$はBanach空間である.
        \end{Example}

        次の二つは特に関数解析学で重要である.
        \begin{Example}[\S 2.3, p.37-40]
            $\Omega$を$\R^n$の可測集合とし,$|\Omega|>0$とする.
            $\Omega$上の可積分関数全体を,
            $u=v (a.e.)$な元を同一視するという同値関係で割ったものを$L^1(\Omega)$と呼ぶ.
            これはノルム\[ \|u\|_{L^1}:=\int_{\Omega}{|u(x)|dx} \]によってBanach空間となる.
        \end{Example}
        \begin{Example}[\S 2.4, p.40-45]
            $\Omega$を前と同じものとする.また,$p$は$1 \leq p < \infty$を満たす実数とする.
            $\Omega$上の可測関数で,
            \[ \|u\|_{L^p}:=\left( \int_{\Omega}{|u(x)|^p dx} \right)^{\frac{1}{p}} < \infty \]
            を満たすもの全体を$\mathscr{L}^p(\Omega)$と書く.
            このノルム$\|u\|_{L^p}$について
            Minkowshiの不等式(三角不等式)及び
            H\"olderの不等式
            \footnote{$p,q \in [1,\infty]$が$1/p+1/q=1$を満たすとき
                $f \in \mathscr{L}^p, g \in \mathscr{L}^q$について$\|fg\|_{L^1} \leq \|f\|_{L^p} \|g\|_{L^q}$}
            が成立する.
            $\mathscr{L}^p(\Omega)$を$u=v (a.e.)$な元を同一視するという同値関係で割ったものを$L^p(\Omega)$と呼ぶ.
            これはノルム$\|u\|_{L^p}$によってBanach空間となる.
        \end{Example}

    \subsection{定理・命題・補題・系}
    \subsubsection{ノルム・内積の基本的性質}
    \begin{Them}
        ノルム空間$(V, \| \cdot \|)$について,ノルム$\|u\|: V \to \C$は連続関数である.
    \end{Them}
    \begin{Them}[定理1.24, p.14]
        内積空間$(V, (\cdot, \cdot))$について,
        内積$(u, v): V \times V \to \C$は$u,v$両方について連続関数である.
    \end{Them}
    \begin{Them}[Schwarzの不等式, 定理1.21, p.13]
        内積空間$(V, (\cdot, \cdot))$に対して,
        \[ \Forall{u, v \in V} |(u,v)| \leq \|u\|\|v\|. \]
        ただしここに有る$\| \cdot \|$は内積が定めるノルムである.
    \end{Them}

    \subsubsection{閉部分空間}
    \begin{Them}[定理1.28, p17] \label{them1:28}
        Banach空間$\spX$の部分空間$\mathscr{M}$を考える.
        $\mathscr{M}$のノルムを$\spX$のノルムを$\mathscr{M}$に制限したものとした時に
        $\mathscr{M}$もBanach空間になるための必要十分条件は,$\mathscr{M}$が$\spX$の閉部分空間であること.
    \end{Them}
    \begin{Cor}[系1.29, p.18] \label{cor1:29}
        Hilbert空間$\spX$の部分空間$\mathscr{M}$を考える.
        $\mathscr{M}$の内積を$\spX$の内積を$\mathscr{M}$に制限したものとした時に
        $\mathscr{M}$もHilbert空間になるための必要十分条件は,$\mathscr{M}$が$\spX$の閉部分空間であること.
    \end{Cor}

    \subsubsection{有限次元ノルム空間}
    \begin{Them}[補題1.38, p.22] \label{them1:38}
        ノルム空間$X$の元$u$を適当な$X$の基底$\{ \phi_1, \dots, \phi_n \}$を用いて
        $u=\alpha_1 \phi_1+\dots+\alpha_n \phi_n (\alpha_k \in \C)$と表したとする.
        $||| u |||:=\sup_{k=1,\dots,n}{|\alpha_k|}$とおくと,これはノルムであり,
        しかも任意の有限次元ノルム空間に備えられた任意のノルムと$||| \cdot |||$は同値である.
    \end{Them}
    このノルム$||| \cdot |||$について任意の有限次元ノルム空間が完備であることを示すことが出来る.
    \begin{Them}[定理1.37, p.22] \label{them1:37}
        有限次元ノルム空間は完備である.
    \end{Them}

    \subsubsection{局所コンパクト性}
    \begin{Them}
        ノルム空間$\spX$の
        単位球$\mathcal{S}=\{u \in \spX : \|u\|=1\}$がコンパクト集合ならば,
        $\spX$は有限次元である.
    \end{Them}


    \section{第11章 / コンパクト作用素,Fredholm作用素}
    \subsection{定義}
    \subsubsection{直和分解と補空間}
    \begin{Def}
        $\spX$をBanach空間,$\spM, \spN$を$\spX$の\kenten{閉部分空間}とする.
        $\spM \cap \spN=\{0\}$であるとき,
        \[ \spM \oplus \spN=\set{m+n}{m \in \spM, n \in \spN}. \]
        とし,これを$\spM$と$\spN$の\textbf{直和}と呼ぶ.
        Banach空間$\spX$が2つの閉部分空間$\spM, \spN$の直和であるとき,
        $\spX$は$\spM$と$\spN$に\textbf{直和分解}されると言い,
        $\spM, \spN$は互いに\textbf{補空間}であると言う.
    \end{Def}
    $\spM \oplus \spN$の元を一つとって$\spM, \spN$の元の和に分解するとき,
    その分解の仕方は一意である.
    $\spX$はHilbert空間ならば,
    任意の閉部分空間$\spM$は補空間$\spM^{\perp}$を持つ.
    $\spX$がBanach空間である時の補空間の存在については定理\ref{them1103}で,
    一意性については定理\ref{them1102}で述べられる.
    以上の定義は2個以上の空間の直和へ一般化される.
    \begin{Def}
        $\spX$をBanach空間,$\spM_1,\dots,\spM_n$をその部分空間とする.
        これらが以下の条件を満たすとする.
        \[ u_k \in \spM_k,~~ u_1+\dots+u_n=0 \implies u_1=\dots=u_n=0. \]
        このとき$\spM_1,\dots,\spM_n$の直和を以下で定める.
        \[
            \spM_1 \oplus \dots \oplus \spM_n
            =\set{u_1+\dots+u_n}{\Forall{k} u_k \in \spM_k}.
        \]
    \end{Def}
    前提条件から,
    $\spM \oplus \spN$の元一つとって$\spM, \spN$の元の和に分解するとき,
    その分解の仕方は一意である.

    次の集合は教科書p.197で一度定義されたもので,Fredholm作用素の定義にも現れる.
    \begin{Def}
        Banach空間$\spX$の部分集合$\spM$に対して$\spM^{\perp}$を以下で定める.
        \[ \spM^{\perp}=\set{f \in \spX^*}{\Forall{u \in \spM} f(u)=0}. \]
    \end{Def}

    \subsubsection{コンパクト作用素}
    \begin{Def}
        $\spX, \spY$をBanach空間,
        $K \in B(\spX, \spY), \dom(K)=\spX$とする.
        $\spX$の任意の有界点列$\{u_n\}$に対して,
        $\{Ku_n\}$が$\spY$で収束する部分列を持つとき,
        $K$を\textbf{コンパクト作用素}(または完全連続作用素)と呼ぶ.
    \end{Def}
    教科書では最初コンパクト作用素に有界であることを求めないが,
    教科書の定理11.10より,そのように定義したコンパクト作用素も有界である.
    次の定理はコンパクト作用素の別の定義として使える.
    \begin{Def}[定理11.9, p.257]
        $\spX, \spY$をBanach空間,
        $K \in B(\spX, \spY), \dom(K)=\spX$とする.
        $\spX$の任意の有界集合$\spM$について$\overline{K \spM}$が
        $\spY$のコンパクト集合であるとき,
        $K$を\textbf{コンパクト作用素}と呼ぶ.
    \end{Def}
    $\spX$から$\spY$へのコンパクト作用素全体を$B_0(\spX,\spY)$と書く.
    \begin{Def}
        $F \in B(\spX, \spY)$について値域$\ran F$が有限次元であるとき,
        $F$を\textbf{有限次元作用素}と呼ぶ.
    \end{Def}
    有界な部分空間$\spM$について$\overline{F \spM}$は
    $\dim \ran F<\infty$ならば有界閉集合になる.
    よって有限次元作用素はコンパクト作用素である.

    \subsubsection{Fredholm作用素}
    \begin{Def}
        $\spX, \spY$をBanach空間,$T \in B(\spX, \spY)$とする.
        $T$が以下の3条件を満たすとき,$T$を\textbf{Fredholm作用素}と呼ぶ.
        \begin{enumerate}
            \item $\dim \ker T < \infty$.
            \item $\dim \ker T^* < \infty$.
            \item $\ran{T} \subset \spY$は閉部分空間.
        \end{enumerate}
    \end{Def}
    $\spX$から$\spY$へのFredholm作用素全体[2]を$F(\spX,\spY)$で表す.
    \begin{Def}
        $T \in F(\spX,\spY)$に対して\textbf{指数}$\ind T$を,
        \[ \ind T=\dim \ker T-\dim \ker T^* \]
        で定義する.
    \end{Def}
    教科書の定理8.43 i) (p.197)より,
    $\dim \ker T^*=\dim (\ran T)^{\perp}:=\codim \ran T$が成り立つ.

    \subsubsection{自己共役なコンパクト作用素}
%    \begin{Def}
%        Hilbert空間$\spX$の有界線形作用素$T \in B(\spX)$に対して,
%        Hilbert共役作用素を$T^{\star}$ \footnote{$(Tu,v)=(u,T^{\star}v)$なるもの.}とする.
%        $T=T^{\star}$であるとき,$T$は\textbf{自己共役}であるという.
%    \end{Def}

    \subsection{例(前半)}
    \subsubsection{コンパクト作用素}
    \begin{Example}[問, p.258]
        $\spX=l^p, 1 \leq p<\infty, a_n \in C, a_n \to 0$とする.
        $K \in B(l^p)$を$K(u_1,u_2,\dots)=(a_1u_1,a_2u_2,\dots)$で定義する.
        $\|Ku\| \leq \sup_i|a_i| \cdot \|u\|$なので確かにこれは有界作用素.
        この時,$\{u^{(i)}\}$を$M$を上限とする有界点列とすると,
        $\|K u^{(i)}\| \leq \|K\|M$なので像も有界点列.
        よって$K$はコンパクト作用素である.
        後にこれがコンパクトであることの別証明を与える.
    \end{Example}

    \newpage
    \subsection{定理・命題・補題・系}
    \subsubsection{直和分解と補空間}
    この節では$\spX$をBanach空間,$\spM,\spN$をその\kenten{閉部分空間}とする.
    \begin{Them}[定理11.2, p.252] \label{them1102}
        $\spM$が補空間を持つならば,
        それらはBanach空間としての同型を除いて一意.
    \end{Them}
    \begin{Them}[定理11.3, p.253] \label{them1103}
        $\spM$は有限次元ならば補空間を持つ.
    \end{Them}
    \begin{Them}[定理11.4, p.253] \label{them1104}
        $\spM^{\perp}$が有限次元であることと,$\spM$は有限次元な補空間$\spN$を持つことは同値.
        しかもその時$\dim \spM^{\perp}=\dim \spN$.
    \end{Them}
    \begin{Them}[定理11.7, p.255] \label{them1107}
        $\spM,\spN$が閉部分空間であり,$\spN$は有限次元であるとする.
        この時,$\spM \cap \spN=\{0\}$ならば
        \footnote{つまり$\spM \oplus \spN$が存在すれば.}
        $\spM \oplus \spN$も閉部分空間である.
    \end{Them}

    \subsubsection{コンパクト作用素}
    この節では$\spX, \spY$をBanach空間とする.
    \begin{Prop}[p.257,p.261] \label{prop-cmp}
        任意の$K \in B(\spX,\spY)$は$\ran K$または$\spY$が有限次元ならばコンパクトである.
        一方,$\spX$が無限次元ならば恒等作用素$I$はコンパクトでない.
    \end{Prop}
    \begin{Them}[定理11.12, p.257] \label{them1112}
        $B_0(\spX,\spY)$は$B(\spX,\spY)$の閉部分空間である.
    \end{Them}
    \begin{Them}[定理11.13, p.258] \label{them1113}
        $S \in B(\spX, \spY), T \in B(\spY,\spZ)$とする.
        $S,T$のどちらか一方でもコンパクトであれば$ST$もコンパクトである.
    \end{Them}
    \begin{Them}[Schauderの定理, 定理11.15, p.258] \label{them1115}
        $K \in B(\spX,\spY)$について,以下が成り立つ.
        \[ K \in B_0(\spX,\spY) \iff K^* \in B_0(\spX^*,\spY^*). \]
    \end{Them}

    \subsubsection{コンパクト作用素のスペクトル理論}
%    $T \in F(\spX), \zeta \neq 0$について
%    $\zeta I-T \in F(\spX), \ind \zeta I-T=0$が成り立つことに注意しておく.
%    これらは教科書のp.268で述べられている.
    \begin{Them}[定理11.29, p.269] \label{them1129}
        $K \in B_0(\spX)$のスペクトル$\sigma(K)$について以下が成り立つ.
        \begin{enumerate}[i)]
            \item $\sigma(K)=\sigma_p(K) \mor \sigma_p(K) \cup \{0\}$.\footnote{$0 \in \sigma_p(K)$はあり得る.}
            \item $\dim \spX=\infty$ならば$\sigma(K)=\sigma_p(K) \cup \{0\}$.
            \item $\sigma_p(K)$は高々加算な集合$\{\zeta_k\}$.
            \item $\sigma_p(K)=\{\zeta_k\}$が加算集合ならば$\lim_{k \to \infty}\zeta_k=0$.
            \item 各$\zeta_k$の多重度は有限.
            \item 各$\zeta_k$は$K^*$の固有値でもある.
            \item 各$\zeta_k$の$K$の固有値としての多重度は$K^*$の固有値としての多重度に等しい.
        \end{enumerate}
    \end{Them}

    \subsubsection{Fredholm作用素}
    \begin{Them}[定理11.20, p.262] \label{them1120}
        $\spX$をBanach空間とし,$K \in B_0(\spX), T=I-K$とする.
        この時$T$はFredholm作用素である.
    \end{Them}
    \begin{Them}[定理11.24, p.264] \label{them1124}
        $T \in B(\spX,\spY)$がFredholm作用素であるための必要十分条件は,
        以下が成り立つこと.
        \begin{align*}
            &\Big[\Exists{A_1 \in B(\spY,\spX), K_1 \in B_0(\spX)} A_1 T+K_1=I \Big] \\
            &\land \\
            &\Big[\Exists{A_2 \in B(\spY,\spX), K_2 \in B_0(\spY)} T A_2+K_2=I \Big]
            .
        \end{align*}
    \end{Them}
    この定理は大雑把に言えば「Fredholm作用素とはコンパクト作用素の違いを無視すれば可逆なもの」ということを言っている.
%    実際,Masamichi Takesaki ``Theory of OPerator Algebra I''の
%    p.55で述べられているFredholm作用素の定義は次のようになっている.
%    「
%    $C^*$代数として$B(\spX), B_0(\spX)$を見て,標準的全射$\pi: B(\spX) \to B(\spX)/B_0(\spX)$を定める.
%    $\spX$のFredholm作用素とは,$\pi(K)$が可逆であるような$K \in B(\spX)$.
%    」

    \begin{Them}[定理11.25, p.265] \label{them1125}
        $S \in F(\spX, \spY), T \in F(\spY,\spZ)$とする.
        この時$ST \in F(\spX,\spZ)$であり,指数について$\ind ST=\ind S+\ind T$となる.
    \end{Them}

%    \begin{Them}[定理11., p.2] \label{them11}
%    \end{Them}

    \subsubsection{自己共役なコンパクト作用素}

    \subsection{例(後半)}
    \subsubsection{コンパクト作用素}
    \begin{Example}[問, p.258]
        (前半で与えた主張の別証明.)
        $\spX=l^p, 1 \leq p<\infty, a_n \in C, a_n \to 0$とする.
        $K \in B(l^p)$を$K(u_1,u_2,\dots)=(a_1u_1,a_2u_2,\dots)$で定義する.
        $K_n(u_1,u_2,\dots)=(a_1u_1,a_2u_2,\dots,a_nu_n,0,0,\dots)$と定めると,
        明らかに$\dim \ran K_n=n<\infty$で,しかもノルム収束の意味で$K_n \to K$.
        したがって命題\ref{prop-cmp}と定理\ref{them1112}より,
        $K$はコンパクト作用素である.
    \end{Example}

    \section{収束に関して}
    Banach空間とその上の作用素が成す空間の点列には,
    いくつかの収束,すなわちいくつかの位相が定められる.
    ここではそれらの定義を整理し,違いを明確にしたい.
    \subsection{定義}
    弱いものから並べていく.
    以下で$\to$と書いているものは$\C$での収束である.
    \newpage
    \subsubsection{Banach空間の点列の収束}
    この節では$\spX$をノルム空間とする.
    さらに$u_n (n=1,2,\dots),~ u \in \spX$とする.
    \begin{description}
        \item[弱収束] $\Forall{f \in \spX^*} f(u_n) \to f(u_n) ~~ (n \to \infty)$.
        \item[強収束] $\|u_n-u\|_{\spX} \to 0 ~~ (n \to \infty)$.
    \end{description}
    それぞれ収束先のことを弱極限,強極限と呼び,これらは存在すれば一意である.
    (これは以下の2節でも同様.)
    弱収束は$\spX^*=B(\spX,\C)$の全ての元が連続であるような位相として最弱のものを誘導する.
    一方,強収束はノルムが定める位相での収束であり,
    この位相について$\spX^*$の元が連続であることはp.28で述べられている.

    \subsubsection{共役空間の点列の収束}
    この節では$\spX$をノルム空間とする.
    さらに$f_n (n=1,2,\dots),~ f \in \spX^*$とする.
    \begin{description}
        \item[汎弱収束] $\Forall{u \in \spX} f_n(u) \to f(u) ~~ (n \to \infty)$.
        \item[弱収束]\hspace{0.6em} $\Forall{\psi \in \spX^{**}} \psi(f_n) \to \psi(f_n) ~~ (n \to \infty)$.
        \item[強収束]\hspace{0.6em} $\|f_n-f\|_{\spX^*} \to 0 ~~ (n \to \infty)$.
    \end{description}
    $\spX$から$\spX^{**}$への自然な写像を$\kappa$としよう.
    \footnote{$\kappa$は$u$をとって$f \mapsto f(u)$という写像を返す.}
    すると,汎弱収束は以下と同値である.
    \[ \Forall{\psi \in \ran \kappa \subset \spX^{**}} \psi(f_n) \to \psi(f_n) ~~ (n \to \infty). \]
    汎弱収束が弱収束より弱いことは明らかである.
    $\spX$が反射的ならば$\ran \kappa=\spX^{**}$なので弱収束と汎弱収束は同値である.
        
    \subsubsection{作用素の点列の収束}
    $\spX$をノルム空間とし,$T_n (n=1,2,\dots),~ T \in \spB{X}{Y}$とする.
    \begin{description}
        \item[弱収束]\hspace{0.6em} $\Forall{u \in \spX} \Forall{f \in \spX^*} f(T_n u) \to f(T u) ~~ (n \to \infty)$.
        \item[強収束]\hspace{0.6em} $\|T_n u-T u\|_{\spY} \to 0 ~~ (n \to \infty)$.
        \item[一様収束] $\|T_n-T\|_{\spB{X}{Y}} \to 0 ~~ (n \to \infty)$.
    \end{description}
    $\{T_n\}_{n=1}^{\infty}$が$T$へ強収束することは,次と同値である.
    \[ \Forall{u \in \spX} \Forall{\varepsilon>0} \Exists{N \in \N}~ n>N \implies \| T_n u-T u \| < \varepsilon. \]
    $\{T_n\}_{n=1}^{\infty}$が$T$へ一様収束(ノルム収束)することは,次と同値である.
    \[ \Forall{\varepsilon>0} \Exists{N \in \N} \Forall{u \in \mathrm{UB}} n>N \implies \| T_n u-T u \| \leq \varepsilon. \]
    ただし$\mathrm{UB}$は$\spX$の閉単位球$\{u \in \spX ~|~ \|u\| \leq 1\}$である.
    つまり,$\{T_n\}_{n=1}^{\infty}$が$T$へ一様収束(ノルム収束)することは,
    $\{T_n\}_{n=1}^{\infty}$が$\mathrm{UB}$上$T$へ一様収束することと同値である.

    \subsection{例}
    \subsection{定理・命題・補題・系}

\end{document}
