\documentclass[a4paper]{jsarticle}
\usepackage[]{macros}

\newcommand{\Isom}{\operatorname{Isom}}
\newcommand{\ftorIsom}{\mathcal{I}\!som}
\newcommand{\hilb}{\mathcal{H}}
\newcommand{\dualnum}{\mathbb{I}}
\newcommand{\famX}{\mathcal{X}}
\newcommand{\famY}{\mathcal{Y}}
\newcommand{\cvU}{\mathfrak{U}}
\newcommand{\der}{\mathrm{d}}
\newcommand{\Der}{\mathrm{Der}}

\begin{document}
\title{ゼミノート \#4 \\ Deformation Theory}
\author{七条彰紀}
\maketitle

\section{Automorphism Group of Stable Curve}
    \cite{HaMo} 3.A, \cite{IrrOfMg} \S 1を参照する.

    $C, D$ :: stable curves of genus $g$ over a scheme $S$の間の
    isomorphism groupのschemeとしての構造を与える.
    このschemeを$\Isom(C, D)$と書く.
    そして$\Aut(C)=\Isom(C,C)$と定義し,
    これのschemeとしての特徴を調べる.
    
    $\Isom(C, D)$の特徴付けをするため,次の関手を考える.
    \begin{defmap}
        \ftorIsom_S(C, D):& \text{(Scheme over $\C$)}& \to& \text{(Set)} \\
        {}& S'& \mapsto& \{ \ C \times_{\C} S' \to D \times_{\C} S' \text{ :: $S'$-isomorphism} \}
    \end{defmap}
    $\iota \in \ftorIsom(C, D)(S')$から得られる$\iota^*$は
    $\shDual_{C \times S'/S'}=\iota^*(\shDual_{D \times S'/S'})$を満たす.
    また$\otimes$と交換する
    (すなわちPicard群の間の準同型である.\cite{HarAG} Ex II.6.8).
    このことから$\Isom(C, D)$が適当な$r$をとると
    $PGL(r+1)$の部分群として書けることが分かる.

    もう少し詳しく$\Isom(C, D)$を書く.
    $n \geq 3$を整数とする.次のように$r,d$をとる.
    \[
        r+1=h^0((\shDual_{C/\C})^{\otimes n})=(2n-1)(g-1),
        \qquad
        d=\deg (\shDual_{C/\C})^{\otimes n}=2n(g-1).
    \]
    すると\cite{HarAG} II.7より,
    $C, D$は$\proj_{\C}^r$の次数$d$, arithmetic genus $g$のclosed curveとみなせる
    ($\proj^r$に埋め込める).
    なのでHildert scheme :: $\hilb=\hilb_{d,g,r}$の点として扱うことが出来る.
    ここで次のように射を定める.
    \begin{defmap}
        \mu:& PGL(r+1)& \to& \hilb \times \hilb \\
        {}& \alpha& \mapsto& (\alpha \cdot [C], [D])
    \end{defmap}
    すると,$\ftorIsom(C, D)$は$\mu^{-1}(\Delta)$によって表現される
    \footnote
    {
        $\Delta$は$\hilb \times \hilb$のdiagonal set.
        $\mu^{-1}(\Delta)$は
        \[ \Delta \cap \im \mu=\{ (\alpha \cdot [C], [D]) \mid \alpha \cdot [C]=[D] \} \]
        の$PGL(r+1)$への逆像なので,
        この点と$C, D$の間の同型と対応することが分かるだろう.
    }.
    これをgroup scheme over $\C$ :: $\Isom(C, D)$とする.

    scheme over $\C$ :: $X$について少々一般の理論を述べる.
    $\dualnum=\Spec \C[\epsilon]/(\epsilon^2)$とおく(ref. \cite{HaMo} 1).
    \cite{HarAG} Ex II.2.8より,
    $t \in \ftor{X}(\dualnum)$は$X$の$\C$-rational point :: $x$と
    $T_x(X)=\I{m}_x/\I{m}_x^2=\shT_x$の元に対応する.
    ここで$\shT$はtangent sheaf :: $\shT=\shHom_{\shO_X}(\shDer_X, \shO_X)$のことである.
    \cite{HaMo}でいうregular vector fieldとは$\shT$のsectionのこと(と思われる).

    \begin{Thm}
        $C$ :: stable curve of genus $g \geq 2$について,
        \[ \operatorname{Ext}^0(\shDer_C, \shO_C)=\shT_C(C)=0. \]
    \end{Thm}
    \begin{proof}
        \cite{IrrOfMg} \S1.

        $\pi: \tilde{C} \to C$をnormalization of $C$とする.
        また$\tilde{C}$のconnected componentの
        個数を$\nu$,それぞれのgenusを$g_i \ (i=1,\dots,\nu)$とする.
        
        今,
        $D \in \shT_C(C)$は
        pullback :: $\pi^*: \shT_{\tilde{C}} \to \shT_{C}$によって
        \footnote
        {
            $R$ :: ring, $A, B$ :: ring over $R$とする.
            一般に,$k$-homomorphism :: $\phi: A \to B$があるとき,
            $D \in \Der_{R}(B)$は$\phi^*: D \mapsto D \circ \phi$によって
            $\Der_R(A)$へ写すことが出来る.
        }.
        $\tilde{D} \in \shT_{\tilde{C}}(\tilde{C})$
        $C$のdouble pointに$\pi$で対応する点
        (point laying over double point, plodp)で$0$になるような
        regular vector field :: $\tilde{D} \in \shT_{\tilde{C}}(\tilde{C})$
        に対応する(TODO).
        このような$\tilde{D}$は$0$しかないことを確かめれば,
        $\shT_C(C)=0$がわかる.

        \begin{Claim}
            $1$点$P \in \tilde{C}$で$\tilde{D}_P=0$ならば,
            $\tilde{D}=0$である.
        \end{Claim}
        \begin{proof}
            $C$ :: reduced connected schemeに注意する.
            $P \in C$において$\tilde{D} \in \shT_C(C)$が$\tilde{D}_P=0$を満たすとしよう.
            $C$のirreducible affine open cover :: $\cvU$をとり,
            $P \in U$なる$U=\Spec A \in \cvU$をとって固定する.
            すると$C$ :: reducedより$A$ :: integral domain.
            $\tilde{D}|_U \in \shT_C(U)$が$P \in U$で$0$になるのだから,
            次が成立する.
            \[ \Exists{u \in A-\I{p}_P} u \cdot (\tilde{D}|_U)=0. \]
            $A$ :: integralより,これは$\tilde{D}|_U=0$を意味する.
            $U$と交わるirreducible affine open subset of $C$ :: $V \in \cvU$についても,
            $\tilde{D}|_{U \cap V}=0$なので$\tilde{D}|_{V}=0$.
            $C$ :: connectedなので,このように$V$を取り続けることで,
            全ての$V \in \cvU$について$\tilde{D}|_V=0$であることがわかる.
            sheafのIdentity Axiomから,$C$全体で$t=0$.
        \end{proof}

        したがって我々は
        $\tilde{C}$の各componentは少なくとも一つずつ
        plodpをもつこと示せば良い.

        $\shT_{\tilde{C}}=\shHom(\shDer_{\tilde{C}/\C}, \shO_{\tilde{C}})$なので,
        $\shT_{\tilde{C}}$に対応するdivisorは$K_{\tilde{C}}$.
        $\deg K_{\tilde{C}}=2\tilde{g}-2$なので,
        $\tilde{g}>1$ならば$\deg (-K_{\tilde{C}})<0$.
        したがって\cite{HarAG} Lemma IV.1.2から$\dim_{\C} H^0(\tilde{C}, \shT_{\tilde{C}})=0$.
        すなわち$\shT_{\tilde{C}}(\tilde{C})=0$.
        なので以下では$\tilde{g}_i=0,1$とする.

        $\tilde{g}_i=0,1$であるとき,
        $\tilde{C}$の各connected componentは必ずplodpをもつ.
        実際,genus formulaで$\delta=0$とすると
        \[ g=\sum_{i}(\tilde{g}_i-1)+1 \geq 2 \]
        したがって$\sum_{i}(\tilde{g}_i-1)>0$ということになる.
        しかし仮定から$\tilde{g}_i-1 \leq 0$なので,$\delta>0$.
        すなわち$C$は必ずnodeをもつ.
        $\tilde{C}$の各componentはsmoothであることと
        $C$がconnectedであることも踏まえて考えると,
        $\tilde{C}$の各componentは少なくとも一つずつ
        plodpをもつことが分かる.
        (この辺りは\cite{IrrOfMg} Lemma1.4で詳しく述べられている).
    \end{proof}

    \begin{Prop}
        $\Aut(C)$ :: reduced scheme.
    \end{Prop}
    \begin{proof}
        $X=\Aut(C)$はgroup schemeであるから,
        $X$のある点でのlocalな性質は
        transitionを用いて単位元$I$での性質と言い換えられる.
        reducedはlocalな性質であるから,
        $X$がreducedであることを示すには$\shO_{X, I}$ :: reduced ringを示せば良い.

        $\shO_{X, I}$がnon-reducedであると仮定する.
        この時,非自明な(nonzero)射$\Spec D \to X=\Aut(C)$がとれる(TODO).
        $\Aut(C)$の定義から,この射は$C \times D$の自己同型に対応する(後述).
        これは$\shT_C(C)$の非自明な元($\neq 0$)に対応する.
        よって定理から$X$ :: reducedが分かる.
    \end{proof}

\section{Definitions of Deformations and Versal Deformation.}
    \begin{Def}[Deformation of Scheme \cite{HaMo} 3.B, \cite{DefLCI} 6.1]
        $X$ :: scheme over $k$,
        $(Y, y_0)$ :: pointed scheme over $k$とする.
        deformation of $X$とは,次の二つから成る組 $(\famX, Y)$.
        \begin{itemize}
            \item flat family :: $\phi: \famX \to (Y, y_0)$.
            \item morphism :: $\psi: X \to \famX$.
        \end{itemize}
        これらは次の条件を満たす:
        以下の図式がfiber productであり,
        $\psi$から$X \iso \famX \times_{Y} \Spec k(y_0)$が誘導される.
        \[\vcenter{\xymatrix{
            X \ar[r]^-{\psi} \ar[d]& \famX \ar[d]^-{\phi} \\
            k(y_0) \ar[r]& Y
            \ar@{}[lu]|{\ulcorner}
        }}\]
        ここで$\Spec (\shO_{Y, y_0}/\I{m}_{Y, y_0})$を$k(y_0)$と書いている.

        deformations of $X$ :: $(\famX, Y), (\famX', Y')$が
        equivalentであるとは,
        可換図式が成立する同型$\eta: \famX' \to \famX$が存在すること.
        \[\xymatrix{
            X \ar@{=}[d]\ar[r]^-{\psi'}& \famX' \ar[d]^-{\eta}\\
            X \ar[r]_-{\psi}& \famX
        }\]
    \end{Def}

    \begin{Def}[Universal Deformation, \cite{HaMo} 3.B, \cite{HarDef} \S15]
        universal deformation for $X$とは,
        次の性質を満たすdeformation of $X$ ::
        $\phi: \famX \to (S, s_0) \& \psi: X \to \famX$のこと:
        任意のdeformation of $X$ ::
        $\phi': \famX' \to (S', s'_0) \& \psi': X \to \famX'$に対して
        射$f: S' \to S$が一意に存在し,
        $f$と$\phi, \psi$のpullbackで生じるdeformationが
        誘導されるdeformationが$\phi' \& \psi'$とequivalent.
    \end{Def}

    \begin{Def}[Versal Deformation, \cite{GACII} \S 11 p.188, p.192]
        $f: X \to Y$がanalitically isomorphismであるとは,
        任意の$x \in X$について,
        $f$のstalkの完備化により得られる準同型
        \[ (f^{\#}_x)\sidehat: (\shO_{Y, f(x)})\sidehat \to (\shO_{X, x})\sidehat \]
        が同型であるということ(\cite{HarAG} \S I.5).

        deformation of $X$ :: $\phi: \famX \to (S, s_0)$がversal deformationである
        (versality propertyをもつ)とは,
        次の性質を持つということである:
        任意のdeformation of $X$ :: $\xi: \famY \to (T, t_0)$と
        任意の点$t \in T$に対して,
        $t$の開近傍$U \subseteq T$と射$f: U \to S$が存在し,
        射影$(\famY \times_T U=)\xi^{-1}(U)$と$\famX \times_S U$の間に
        analitically isomorphismが存在するということ.
        \[\vcenter{\xymatrix{
            \xi^{-1}(U) \ar[d]\ar@{~>}[r]& \famX \times_S U \ar[r]\ar[d]& \famX \ar[d]^-{\phi}\\
            U \ar@{=}[r]& U \ar[r]_-{f}& S
            \ar@{}[lu]|{\ulcorner}
        }}\]
    \end{Def}

    \begin{Remark}
        開集合$U$を``any sufficiently small connected neighborhood of $t_0$"に限定することも
        あるようである(\cite{GACII} \S11, p.188).
    \end{Remark}

    \begin{Def}[First Order Deformation]
        $D=\C[x]/(x^2), \epsilon=x \bmod (x^2)$とする.
        $\dualnum=\Spec D$の唯一の閉点を$0$で表す.
        $(\dualnum, 0)$上のdeformationを,
        first order deformation (or infinitesimal deformation)と呼ぶ.
    \end{Def}
    これは``formal deformation"の最も簡単な場合である.
    $D$ :: local atrinian ringであり,
    したがってcomplete ringでもある.

    \begin{Def}[Homomorphism of Deformations for First Order Deformation, \cite{DefLCI} 7.2]
        homomorphism of deformations of $X$ ::
        $(\alpha, f): (\famX, \dualnum) \to (\famY, \dualnum)$とは
        homomohpsim $f: \dualnum \to \dualnum$と,
        $\alpha: f_*\famX \iso \famY$の組のことである.
        ここで$f_* \famX$は$f$と$\famX \to \dualnum$のfiber productで得られるdeformationである.
    \end{Def}

    \begin{Def}[Versal Deformation for First Order Deformation, \cite{DefLCI} 7.2]
        deformation of $X$ :: $\phi: \famX \to (S, s_0)$がversal deformationである
        (versality propertyをもつ)とは,
        次の性質を持つということである:
        \[\xymatrix{
            {} & (X, \C) \ar@{->>}[d]^-{\forall} \\
            (\famX, S) \ar@{-->}[ru]^-{\exists} \ar[r]^-{\forall}& (\famY, T)
        }\]
    \end{Def}

    通常の場合,versal deformationはほとんど存在しない.


\section{First Order Deformation of Nonsingular Scheme Of Finite Type.}
    \begin{Lemma}[\cite{Eisen} Cor6.2]
        $D$-module :: $M$がflatであることは,
        $M/\epsilon M \xrightarrow{\times \epsilon} \epsilon M$
        が同型であることと同値.
    \end{Lemma}

    \begin{Lemma}[\cite{GlimpseDefTh} Prop5.1]
        $X$ :: affine, nonsingular, finite type scheme over a field $k$とする.
        この時,$X$のfirst order deformationは
        自明なdeformation :: $X \times_k \Spec D$しか存在しない.
    \end{Lemma}
    \begin{proof}
        $\phi: \famX \to \dualnum, \psi: X \to \famX$を
        $X$のfirst order deformationとする.
        $\phi$ :: flatと上の補題を用いると,
        $X \isomap \famX \times_{D} \Spec \C$から
        $\psi^{\#}: \shO_{\famX}/\epsilon \shO_{\famX} \to \shO_X$が
        同型であることが得られる.
        逆にこの同型が存在する時$\famX \to \dualnum$がflatであることが言える.
        したがって$X$のfirst order deformationは
        infinitesimal extension of $X$ by $\shO_X$ (\cite{HarAG} Ex II.8.7)に対応する.
        しかし\cite{HarAG} Ex II.8.7より,これは自明なものしか存在しない.
    \end{proof}

    \begin{Lemma}[\cite{GlimpseDefTh} Prop5.2]
        $X$ :: nonsingular scheme of finite type over $k$とする.
        この時,次のsheafを考える.
        \[ X \supseteq U \mapsto \{ \text{$\dualnum$-automorphisms of $D \times_k \dualnum$} \}. \]
        するとこのsheafはtangent sheaf of $X$ :: $\shT_X$と同型である.
    \end{Lemma}

    \begin{Thm}[\cite{GlimpseDefTh} p.7]
        $X$ :: separated nonsingular scheme of finite type over $k$とする.
        特に,$X$ :: nonsingular (abstruct) variety over $K$であればよい.
        この時,first oder deformation of $X$の同値類は
        $H^1(X, \shT_X)$の元と一対一に対応する.
    \end{Thm}
    \begin{proof}
        
    \end{proof}

\section{Several Other Deformation Theory}
    ここではfirst order deformation of a nonsingular varietyの変種として,
    様々な``deformation of something"の問題と
    その``space of first order deformation"を列挙する.

    ここでの``curve"は私のノート``session2\_ApproachesToConstructionOfMg"同様に
    smooth complete (abstruct) variety of dimention $1$ over $\C$を意味する.

    \subsection{Deformation of a nonsingular curve}
    \subsection{Deformation of a nonsingular pointed curve}
    \subsection{Deformation of a curve with line bundle.}
    \subsection{Deformation of a map \tp{$f: X \to Y$}{f: X to Y} with \tp{$X, Y$}{X, Y} both fixed.}
    \subsection{Deformation of a map \tp{$f: X \to Y$}{F: X to Y} with only \tp{$Y$}{Y} fixed.}
    \subsection{Deformation of a singular point of plane curve.}
    \subsection{Deformation of a singular variety.}

\section{Formal Deformation}
    \begin{Def}
        
    \end{Def}

%    \cite{HarAG} Example III.9.13.1
%    Lectures on Moduli of Curves(D_Gieseker).pdf

\bibliographystyle{jplain}
\bibliography{reference}
\end{document}
