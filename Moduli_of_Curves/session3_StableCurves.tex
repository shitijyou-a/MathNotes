\documentclass[a4paper]{jsarticle}
\usepackage[]{../math_note}
\usepackage[all]{xy}

\newcommand{\tp}[2]{\texorpdfstring{#1}{#2}}

%% moduli spaces
\newcommand{\modA}{\mathcal{A}}
\newcommand{\modC}{\mathcal{C}}
\newcommand{\modH}{\mathcal{H}}
\newcommand{\modP}{\mathcal{P}}
\newcommand{\modT}{\mathcal{T}}

\newcommand{\an}[1]{{#1}^{\,\mathrm{an}}}
\newcommand{\Pic}{\operatorname{Pic}}

\newcommand{\M}{\mathcal{M}}
\newcommand{\K}{\mathcal{K}}
\newcommand{\barM}{\overline{\mathcal{M}}}

\newcommand{\Sing}{\operatorname{Sing}}
\newcommand{\Sm}{\operatorname{Sm}}

\begin{document}
\title{ゼミノート \#3 \\ Stable Curves}
\author{七条彰紀}
\maketitle

\cite{HaMo} 2.C,Dを中心にstable curveについて記述する.
以下で曲線は全て\textbf{arithmetic genusが$2$以上}であるものとする.
これは特異曲線を扱うためにarithmetic genusを用い,
自己同型群が有限であるために$\geq 2$とする.

\section{Motivation: To Get Modular Compactification of $\M_g$.}
    G.I.T.によって$\M_g$を得る方法では,
    Hilbert scheme $\modH_{2(g-1)n, g, N}$\footnote{$N=(2n-1)(g-1)-1$.}の開集合$K$を用いて
    $\M_g=K/PGL(N+1, \C)$
    として$\M_g$を得た.

    そこでcompactification of $\M_g$
    (ここでは$\M_g$を開集合として含むprojetive scheme over $\Z$)を得る方法として,
    $K$の$\modH_{2(g-1)n, g, N}$での閉包を取って$PGL_{N+1}(\C)$で割る,
    ということが思いつく.
    しかしこれではmoduli spaceが得られない.
    moduli spaceを得るためには,
    $K$を含む集合$\tilde{K}$の商$\tilde{K}/PGL_{N+1}(\C)$をとらなくてはならない.
    これらの包含関係は$K \subset \tilde{K} \subset \cl_{\modH}(K)$となる.

    $\bar{K}$でなく$\tilde{K}$,という制限が必要な理由は次のように説明される:
    次のような$(s:t) \in \proj_{\C}^1-\{(0:1), (1:0)\}=:B$で
    パラメトライズされるfamily of smooth curvesを考える.
    \[
        C_{(a,b)}: s^3y^2z=s^3 x^3-st^2axz-t^3bz^3
        \mwhere
        a,b \in \C, (s:t) \neq (0:1), (1:0)
    \]
    $j$-invariantを計算すると,
    これはfiberwise trivial family(session1A2A参照)になっている.
    また,この曲線族$C_{(a,b)}$は$a,b$の値を変えることで
    任意の楕円曲線を含むものに成る.
    今family :: $C_{(a,b)} \to B$があるから,
    coarse moduli spaceの定義より,morphism :: $\phi \colon B \to \barM_g$が存在する.
    今,$\barM_g$はprojective (over $\Z$)であるから,proper (\cite{HarAG} Thm II.4.9).
    なので$B=\proj_{\C}^1-\{(0:1), (1:0)\} \to \barM_g$は
    $\bar{\phi}: \proj^1 \to \barM_g$へ拡張される
    \footnote
    {
        証明の概略:
        criterion of propernessを用いて
        $\Spec \shO_{\affine^1, \zeta} \to \barM_g$を
        $\Spec \shO_{\affine^1, t} \to \barM_g$に拡張し,
        これらが$\phi$と貼り合わせられることを射の一意性から述べる.
        \url{ https://math.stackexchange.com/questions/1540201 },
        \url{ http://lovelylittlelemmas.rjprojects.net/properness-and-completeness-of-curves/ }
        を参照のこと.
    }.
    そこで$\bar{\phi}$の$t=0$におけるfiberを考えると,
    明らかにこれはcuspidal curve :: $y^2z=x^3$である.
    これはrational curveであり,他のfiberと同型でない.
    他のfiberは全て同型であったから,
    このfamilyではjamp phenomenonが発生している.
    したがってmoduli spaceを得るためには,
    cuspidal curveに対応する点を$K$に(そして$\M_g$に)付け加えてはならない.
    この意味でcuspidal curveは楕円曲線の``bad degeneration"と呼べる.

    ではjump phenomenonが発生しないような``good degeneration"は何か,
    というと,これがstable curveである.
    DeligneとMumfordがstable curveを定義し,研究した.

    3A, 4A

\section{Definition.}
    \begin{Def}[Stable Curve]
        stable curveとは,以下を満たす曲線(scheme of dimension $1$ over $\C$)である.
        \begin{enumerate}
            \item 完備 ($=$proper),
            \item 連結,
            \item (存在すれば)特異点は2重点(node)のみ,
            \item 自己同型群が有限位数.
        \end{enumerate}
    \end{Def}
    \begin{Remark}
    Hurwitz's automorphisms theoremから,
    connected proper smooth curve of genus $g \geq 2$は
    全てstable curveである.
    \end{Remark}

    まったく同様にstable $n$-pointed curveも定義できる.
    \begin{Def}
        stable $n$-pointed curveとは,
        以下を満たす曲線$C$(scheme of dimension $1$ over $\C$)
        \begin{enumerate}
            \item 完備 (=proper),
            \item 連結,
            \item (存在すれば)特異点は2重点(node)のみ,
        \end{enumerate}
        と,$n$個の互いに異なる$C$の点$p_1,\dots,p_n$の組$(C, p_1,\dots,p_n)$であって,
        $\sigma(p_i)=p_i$を満たすような自己同型$\sigma: C \to C$が
        成す群が有限群であるものである.
    \end{Def}
    関連してsemi-stable (pointed) curveとunstable (pointed) curveの概念がある.
    これは「自己同型群が有限群」であるという条件をゆるめたもので,
    「自己同型群が簡約群(reductive group)」とする.
    reductive groupはG.I.T.の文脈で現れる概念である.

    %% smooth rational (irreducible) componentについての条件で言い換えることも出来る.

\section{Example}
    \begin{Example}
        次の$\affine^1$上のfamilyを考える.
        \[ C_t: y^2z=x(x-1)(x-t) \mwhere t \in \affine^1. \]
        $t \neq 0,1$の時,$C_t$は楕円曲線である.
        また,任意の$\C$上の楕円曲線はこのfamilyのいずれかのfiberと同型である.
        すなわち,fiberwise trivial familyではない.
        そして$t=0,1$の時$C_t$はstable curveとなっている.

        実際,(TODO: 証明)
    \end{Example}

\section{$\Delta=\barM_g-\M_g$}
    \begin{Thm}
        coarse moduli space of stable curves (resp. stable $n$-pointed curves)
        :: $\barM_g$ (resp. $\barM_{g,n}$)が存在し,
        これはprojective varietyである.
    \end{Thm}
    さらに,
    Nodeを$\delta$個以上もつstable curveに対応する点の集合は,
    pure codimention $\delta$であることが知られている.
    特にこのことから,stable curveは多くとも$3g-3$個のnodeしか持てないことが分かる.

\section{(Semi-)Stable Reduction.}

\bibliographystyle{jplain}
\bibliography{reference}
\end{document}
