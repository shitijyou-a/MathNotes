\documentclass[a4]{article}
\usepackage{../math_note}
\usepackage{amsmath}
\newcommand{\affine}{\mathbb{A}}
\newcommand{\proj}{\mathbb{P}}

\begin{document}
    \section{射影曲線}
        \subsection{射影空間}
        $\affine^{n+1} \setminus \{ 0 \}$において、
        $\mathbf{a}=(a_0, \dots, a_n)$, $\mathbf{b}=(b_0, \dots, b_n)$に対し、
        以下のように同値関係を入れる(同値関係であることは自明)。
        \[
        \mathbf{a} \sim \mathbf{b} \iff \exists \lambda \in k^{\times} ~s.t.~ \lambda \mathbf{a}=\mathbf{b}
        \]
        そこで、$n$次元射影空間を以下で定める。
        \[
        \mathbb{P}^n := (\affine^{n+1} \setminus \{ 0 \}) / \sim
        \]
        
        点$A \in \proj^n$に対し、
        その代表元として$\mathbf{a}=(a_0, \dots, a_n)$をとる。
        このとき、$A=(a_0 : \dots : a_n)$と表し、
        $a_i$達を$A$の斉次座標と呼ぶ。
        全ての$a_i$が0になる点は無い。
        
        各$i=0, 1, \dots, n$に対し、
        \[
            \sqcup_i:=\{(a_0: a_1: \dots: a_n) | a_i \neq 0\} \subset \proj^n
        \]
        このとき、$\proj^n = \bigcup_{i=0}^{n}{\sqcup_i}$となる。
        この$\{\sqcup_i \}_{i=0}^{n}$をアフィン開被覆と呼ぶ。
        
        \begin{Lemma}
            各$i$に対し、$\phi_i$を
            \begin{eqnarray*}
                \phi_{i} :
                \sqcup_i &\to& \affine^n \\
                (a_0: \dots: a_i: \dots: a_n) &\mapsto& (a_0/a_i, \dots, a_n/a_i)
            \end{eqnarray*}
            とおく。これは全単射で、
            \begin{eqnarray*}
                \psi_{i} :
                \affine^n &\to& \sqcup_i \\
                (a_0, \dots, a_n) &\mapsto& (a_0: \dots: \underset{\text{$i$番目の要素}}{1}: \dots: a_n)
            \end{eqnarray*}
            がその逆写像である。
        \end{Lemma}
        
        \begin{Proof}
            全単射の定義にしたがって調べれば良い。 \QED
        \end{Proof}
        
        零点集合$Z(X_i)=\{(a_0: a_1: \dots: a_n) | a_i=0\}$を$\sqcup_i$の
        無限遠超平面と言う。これはそれぞれ$\sqcup_i$の補集合で、
        $\affine^n \simeq \sqcup_i = \proj^n \setminus Z(X_i)$が成り立つ。
        零点集合はぞれぞれ$\sqcup_i$と無限遠で交わる。

        \subsection{射影曲線}
        \begin{Def}[射影曲線]
            体$k$上の斉次多項式$F \in k[X, Y,Z]$について、
            以下で定まる集合を射影曲線と呼ぶ。
            \[ C:=Z(F)=\{p \in \proj^2 : F(p)=0 \}\]
            $\deg C:=\deg F=d$を$C$の次数と呼び、
            また、$C$を$d$次曲線と呼ぶ。
        \end{Def}

        これはwell-definedである。
        なぜなら任意の点$p \in \proj^2$と、任意の$\lambda \in k^{\times}$について、
        $F(\lambda p)=(\lambda^{\deg F}) F(p)$だからである。
        したがって$F(p)=0$の解集合は点$p$の斉次座標のとり方によらない。
        これは$F$が斉次多項式であることから成り立つ。
        逆に、このように射影曲線$C$がwell-definedであるためには
        $F$は斉次多項式でなくてはならない。

        \begin{Prop}
            2点$\mathbf{a}, \mathbf{b} \in \proj^2$
            (ただし$\mathbf{a} \neq \mathbf{b}$)を通る直線はただ一つであり、
            \footnote{$\lambda \mathbf{a}$,$\mu \mathbf{b}$も同じ2点を表すということを考えれば、これは自明ではない。}
            その定義多項式$\bar{F}$は以下で与えられる。
            \[
            \bar{F}=
              \left|
                  \begin{array}{ccc}
                      X & Y & Z \\
                      a_0 & a_1 & a_2 \\
                      b_0 & b_1 & b_2
                  \end{array}
              \right|
            \]
            ここで$\mathbf{a}=(a_0 : a_1 : a_2), \mathbf{b}=(b_0 : b_1 : b_2)$とした。
        \end{Prop}
        \begin{Proof}
            主張に有る$\bar{F}$は$\mathbf{a}$か$\mathbf{b}$を代入すると
            同じ行を2つもつ行列式になるから、0になる。
            また、$\bar{F}$は一次斉次多項式である。
            したがって$\bar{F}$は$\mathbf{a}, \mathbf{b}$を通る直線の定義多項式の一つである。
            以下、このような直線がただ一つであることを示す。

            直線の定義多項式$F$は1次斉次多項式だから、
            $F(X, Y, Z)=\alpha X + \beta Y + \gamma Z$のように表される。
            写像$\varepsilon$を以下で定義する。
            \begin{eqnarray*}
                \varepsilon : \{ \mbox{k上の1次斉次多項式全体} \} &\to& k^2 \\
                F &\mapsto& 
                        \left[
                          \begin{array}{c}
                              F(\mathbf{a})\\
                              F(\mathbf{b})
                          \end{array}
                        \right]
            \end{eqnarray*}
            $\varepsilon$で$F$を送った先が$\mathbf{0}$であれば
            $F$は2点$\mathbf{a}, \mathbf{b}$を通る。
            すなわち、定義多項式は$\ker \varepsilon$の元である。
            すでに述べたように、$\bar{F} \in \ker \varepsilon$となっている。

            $\varepsilon$で$F$を送った先をもう少し考えると、次のようになる。
            \[
                F \mapsto 
                        \left[
                          \begin{array}{c}
                              F(\mathbf{a}) \\
                              F(\mathbf{b})
                          \end{array}
                        \right]
                        =
                        \left[
                          \begin{array}{ccc}
                              a_0 & a_1 & a_2 \\
                              b_0 & b_1 & b_2
                          \end{array}
                        \right]
                        \left[
                          \begin{array}{c}
                              \alpha \\ \beta \\ \gamma
                          \end{array}
                        \right]
            \]
            
            $\mathbf{a} \neq \mathbf{b}$から、
            $\operatorname{rank} \left[ \begin{array}{ccc} a_0 & a_1 & a_2 \\ b_0 & b_1 & b_2 \end{array} \right]=2$である。
            したがって$\dim \ker \varepsilon=3-2=1$となる。
            つまり、$\ker \varepsilon$は$\bar{F}$を有る一つのパラメータで変化させたもの全体。
            実際、$\bar{F}$を$k^{\times}$倍したものも$\ker \varepsilon$の元である
            \footnote{$\bar{F}$の$X$の係数だけ変化させても$\ker \varepsilon$の元になる、といった可能性を排除するための議論だった。}。
            以上の議論から、$\ker \varepsilon$の元は$k^{\times}$倍を除いて一意。
            したがって、2点$\mathbf{a}, \mathbf{b}$を結ぶ直線$Z(F)$は一意。
            \QED
        \end{Proof}

        \begin{Def}
            $\proj^2$内の直線全体のなす集合を$\check{\proj^2}$と書く。
            \[ \check{\proj}^2 := \{Z(\alpha X+ \beta Y+ \gamma Z) : (\alpha, \beta, \gamma) \in k^3 \setminus\{0\} \}\]
            これを双対射影空間と呼ぶ。
        \end{Def}
        実際、
        $\check{\proj}^2 \ni Z(\alpha X+ \beta Y+ \gamma Z) \mapsto (\alpha : \beta : \gamma) \in \proj^2$
        は全単射。
        
        問。素数$p$について$\proj^2_{\mathbb{F}_p}$に含まれる直線は何本か。

        \subsection{多項式の斉次化・非斉次化}
        以下では$k$を体、
        $\mathbb{X}=(X_0, \dots, X_n)$\footnote{$(n+1)$個の不定元。},
        $\mathbb{Y}=(Y_1, \dots, Y_n)$\footnote{$n$個の不定元。}とおく。

        \begin{Def}[非斉次化]
        \begin{eqnarray*}
            \alpha :
                k[\mathbb{X}] &\to& k[\mathbb{Y}]\\
                F(\mathbb{X}) &\mapsto& F(1, Y_1, \dots, Y_n)
        \end{eqnarray*}
        これを$X_0$に関する非斉次化と呼ぶ。
        \end{Def}
        これは代入なので環の準同型写像である。

        \begin{Def}[斉次化]
        \begin{eqnarray*}
            \beta :
                k[\mathbb{Y}] &\to& k[\mathbb{X}] \\
                f(\mathbb{Y}) &\mapsto& X_0^{\deg f} f \left( \frac{X_1}{X_0}, \dots, \frac{X_n}{X_0} \right)
        \end{eqnarray*}
        これを$X_0$に関する斉次化と呼ぶ。
        \end{Def}
        これは次に示すように準同型写像でない。

        \begin{Prop}
            $f \in k[\mathbb{Y}]$に対して$\beta (f)$は斉次多項式。
            さらに、$f$の斉次分解を$f=\sum^{d}_{k=0}{f_k(\mathbb{X})}$とした時
            \[ \beta(f)(\mathbb{X})=\sum^{d}_{k=0}{X_0^{d-k} f_k(\mathbb{X}')}\]
            となる。ただし$d:=\deg f$, $\mathbb{X}'=(X_1, \dots, X_n)$\footnote{$X_0$を$\mathbb{X}$から消した。}とした。
        \end{Prop}
        \begin{Proof}
            「さらに、」以降の主張から前半の主張は明らか。
            $f(\mathbb{Y})=\sum_{0 \leq k \leq d}{ \sum_{|\mathbb{I}|=k}{c_\mathbb{I} \mathbb{Y}^{\mathbb{I}}} }$
            とする。$\beta(f)$は次のようになる。
            \begin{eqnarray*}
                \beta(f)(\mathbb{X})
                &=& X_0^d \sum^{d}_{k=0}{
                    \left( \sum_{|\mathbb{I}|=k}{c_\mathbb{I}
                    \left(\frac{X_1}{X_0}\right)^{i_1} \cdots \left(\frac{X_n}{X_0}\right)^{i_n}} \right)} \\
                &=& \sum^{d}_{k=0}{
                    \left( \sum_{|\mathbb{I}|=k}{X_0^{d-|\mathbb{I}|} \cdot c_\mathbb{I} {\mathbb{X}'}^{\mathbb{I}}} \right)} \\
                &=& \sum^{d}_{k=0}{X_0^{d-k} f_k(\mathbb{X}')}
            \end{eqnarray*}
            \QED
        \end{Proof}

        \subsubsection{$\alpha, \beta$の関係}
        証明は略すが、$\alpha(\beta(f))=f$が成り立つ。
        しかし$\beta(\alpha(F))=F$は一般に成立しない。

        \begin{Lemma}
            斉次多項式$F \in k[\mathbb{X}]$に対し、
            ある$e \geq 0$が存在して次式が成り立つ。
            \[ F(\mathbb{X}) = X_0^e \cdot \beta(\alpha(F(\mathbb{X})))\]
        \end{Lemma}
        \begin{Proof}
            $d:=\deg F$として、
            \[ F(\mathbb{X}) = \sum_{|\mathbb{I}|=d}{c_{\mathbb{I}} X_0^{i_0} \cdots X_n^{i_n}}\]
            と表せる。この時、
            \[ \alpha(F)(\mathbb{X}) = \sum_{|\mathbb{I}|=d}{c_{\mathbb{I}} 1^{i_0} X_1^{i_1} \cdots X_n^{i_n}}\]
            明らかに$\deg \alpha(F) \leq d$なので、この差を$e$と置く。
            つまり$\deg \alpha(F)=d-e$とする。
            すると$d-\sum_{1 \leq j \leq n}{i_j}=i_0$より、以下のようになる。
            \begin{eqnarray*}
                X_0^e \cdot \beta(\alpha(F(\mathbb{X}))) \\
                &=& X_0^e
                \left(
                    X_0^{d-e}
                    \sum_{|\mathbb{I}|=d}{c_{\mathbb{I}} \left(\frac{X_1}{X_0}\right)^{i_1} \cdots \left(\frac{X_n}{X_0}\right)^{i_n}}
                \right) \\
                &=&
                \left(
                    \sum_{|\mathbb{I}|=d}{c_{\mathbb{I}} X_0^{i_0} X_1^{i_1} \cdots X_n^{i_n}}
                \right) \\
                &=& F(\mathbb{X})
            \end{eqnarray*}
            等号が成立するのは$i_0=0 \implies c_{\mathbb{I}}=0$の時。
            \QED
        \end{Proof}

        \subsection{アフィン曲線の射影化}
        \begin{Def}
            多項式$f \in k[x,y]$によって定まるアフィン曲線$C:=Z_a(f) \subset \affine^2$に対し、
            $Z_p(\beta(f)) \subset \proj^2$を$C$の射影化と呼ぶ。
            ただし、$\beta$は$Z$に関する斉次化、
            すなわち$\beta : f(x, y) \mapsto Z^{\deg f}f \left( \frac{X}{Z},\frac{Y}{Z}\right)$である。
        \end{Def}

        \begin{Def}
            斉次多項式$F \in k[X, Y, Z]$により定まる射影曲線$C:=Z_p(F)$に対し、
            $Z_a(\alpha(F)) \subset \affine^2$をその$Z \neq 0$のアフィン部分と呼ぶ。
            ただし$\alpha$は$Z$に関する非斉次化、
            すなわち$\alpha : F(X, Y, Z) \mapsto F(x, y, 1)$である。
        \end{Def}
        アフィン部分には他に$X$に関するもの、$Y$に関するものがある。

        \begin{Prop}
            斉次多項式$F \in k[X, Y, Z]$に対して、
            \[ Z_p(F) \cap \sqcup_c = \psi_c(Z_a(\alpha(F)))\]
            ただし$\sqcup_c$と$\psi_c$は補題1で定義したものである。
        \end{Prop}
        \begin{Proof}
            $\sqcup_c \ni p=(a : b : 1)$をとる。
            \begin{eqnarray*}
                &{}&    p \in Z_p(F) \\
                &\iff&  F(a,b,1)=0 \\
                &\iff&  \alpha(F)(a,b)=0 \\
                &\iff&  \phi_c(p) \in Z_a(\alpha(F)) \\
                &\iff&  p \in \psi_c(Z_a(\alpha(F))) \\
            \end{eqnarray*}
            \QED
        \end{Proof}

        \begin{Lemma}
            $f \in k[x, y]$に対して、
            \[ \overline{\psi(Z_a(f))}=Z_p(\beta(f)) \]
            ただし、左辺はZariski位相での閉包である。
        \end{Lemma}
        この補題は利用しないので証明もしない。

    \subsection{特異点}
        \begin{Def}[射影曲線の特異点]
            斉次多項式$F$により定まる射影曲線$C:=Z_p(F)$において、
            $p \in C$が$C$の特異点であるとは、
            $p$を含むアフィン開被覆における$C$のアフィン部分が$p$に於いて特異点を持つこと
            と定める。

            したがって、$p \in C$が$C$の特異点であるとは、
            $p \in \sqcup_i$のとき、
            アフィン部分$Z_a(\alpha(F))$が$\phi_i(p)$に於いて特異点を持つことである。
        \end{Def}

        \begin{Lemma}
            $p$が$Z_p(F)$の特異点である。
            $\iff$ $F_X(p)=F_Y(p)=F_Z(p)=0$
            \footnote{$F_X$は斉次多項式$F$を$X$について偏微分したものである。$F_Y$なども同様。}
        \end{Lemma}
        \begin{Proof}
            $\sqcup_c \ni p=(a:b:1)$をとり、$f:=\alpha(F)=F(x, y, 1)$とおく。
            $C:=Z_a(f)$が特異点$p$を持つとは、$f$の斉次分解$\{ f_k \}$について
            $m_p(C)=\min\{k : f_k(p)=0\}>1$ということ。
            したがって、
            \[ f_x(a,b)=f_y(a,b)=0 \]
            ここで$F_X(x, y, 1)=f_x(x,y)$, $F_Y(x, y, 1)=f_y(x,y)$だったから、
            \[ F_X(a, b, 1)=F_Y(a, b, 1)=0 \]
            が成り立つ。これは特異点の定義と同値。

            さらにここでオイラーの公式\[ XF_X+YF_Y+ZF_Z=(\deg F)F \]を用いると、
            \[ a \cdot F_X(p)+b \cdot F_Y(p)+1 \cdot F_Z(p)=(\deg F)F(p)=0 \]
            だから、$F_Z(a, b, 1)=0$も出る。
            逆に、$F_X(p)=F_Y(p)=F_Z(p)=0$は明らかに$F_X(p)=F_Y(p)=0$を含む。
            \QED
        \end{Proof}

        \subsection{接線}
        \begin{Def}[射影曲線の接線]
            射影曲線$C$の点$p \in C$における接線を、
            $p$を含むアフィン開被覆の$\phi(p)$における接線の射影化として定める。
        \end{Def}

        \begin{Lemma}
            斉次多項式$F$について、$p \in C:=Z_p(F)$がCの非特異点(単純点)であるとき、
            $p$における$C$の接線は次式で定まる。
            \[ F_X(p)X+F_Y(p)Y+F_Z(p)Z=0 \]
        \end{Lemma}
        \begin{Proof}
            $\sqcup_c \ni p=(a:b:1)$をとり、$f:=\alpha(F)$とする。
            $C$の$\sqcup_c$におけるアフィン部分$Z_a(f)$への$\phi_c(p)$に於ける接線は
            次で定まる。
            \[ f_x(a,b)(x-a)+f_y(a,b)(y-b)=0 \]
            fの定義より、
            \[ F_X(a,b,1)(x-a)+F_Y(a,b,1)(y-b)=0 \]
            これを斉次化すれば
            \[ F_X(a,b,1)(X-aZ)+F_Y(a,b,1)(Y-bZ)=0 \]
            オイラーの公式を用いれば、結論が得られる。
        \end{Proof}

        例として$F=XZ-Y^2$を取ると、
        これの点$p=(a:b:c)$における接線は$cX-2bY+aZ=0$となる。
        標数2の体に於いては、$cX+aZ=0$となり、これは点$(0:1:0)$を常に通る。
        接線が定点を通る曲線をstrange曲線と呼ぶが、
        これは以下の定理の通り、かなり限られた状況のものしか無い。
        \begin{Them}[Samuel]
            非特異射影曲線でstrangeのものは、
            直線(自明な場合)か標数2の2次曲線に限る。
        \end{Them}
        証明はHartshorn, IV, Theorem 3.9にある。

        \subsection{直線との交点数}
        $A=(a_0:a_1:a_2), B=(b_0:b_1:b_2) \in \proj^2$とおく。
        $A, B$を通る直線$L$のパラメータ表示として、
        \[ L: (X:Y:Z)=sA+tB=(s a_0+t b_0:s a_1+t b_1:s a_2+t b_2) \]
        をとる。
        斉次多項式$F$に$L$のパラメータ表示を代入して得られる多項式を
        \[ \Phi(s,t)=F(s a_0+t b_0, s a_1+t b_1, s a_2+t b_2) \]
        と置く。

        $L$上の点$P$は$(s_0, t_0) \neq (0, 0)$によって$P=s_0 A+t_0 B$と表される。
        このとき、$C \cap L$に於ける$C$と$L$の交点数を
        \[ I(C, L; P)=\max \{ m : (s_0 t-t_0 s)^m | \Phi(s,t) \} \]
        で定義する。これはwell-definedである。

        \paragraph{問}
        射影曲線$C=Z(F)$と直線$L$に対して、$L \not \subset C$とする。
        体$k$が代数的閉包ならば、
        \[ \sum_{P \in C \cap L}{I(C, L; P)}=\deg F \]
        となる。これを示せ。ヒントはテイラー展開。

        \begin{Prop}
            $P \in \proj^2$を含むアフィン開被覆での、$C$と$L$のアフィン部分を$C_0, L_0$とすれば
            \[ I(C, L; P)=i(C_0, L_0; \phi(P)) \]
            が成立する。
        \end{Prop}
        \begin{Proof}
            \textbf{定義の確認}~~
            適当に座標変換して$L=Z_p(Y), P=(0:0:1)$とする。
            $f(x, y)=\alpha(F)=F(x, y, 1)$と置けば、
            $C:=Z_p(F)$と$L$のアフィン部分は
            \[ C_0:=Z_a(f), L_0:=Z_a(y) \]
            である。
            $L_0$のパラメータ表示は$(x, y)=(t, 0)$とすれば、
            $P=(0:0:1)$に対応する点は$t=0$で与えられる。
            アフィン曲線の交点数の定義より、
            $i(C_0, L_0; \phi(P))=\max \{ k : t^k|f(t,0) \}$
            
            \paragraph{$F$の分解}
            ここで、$F$を$Y$の多項式として整理する。
            つまり、$F$を多項式環$k[X, Z][Y]$の元として見る。
            $d:=\deg F$とおき、$F_i \in k[X, Z]$をi次斉次多項式とする。
            \[ F=F_d(X, Z)+F_{d-1}(X, Z)Y + \dots + F_0(X, Z) Y^d \]
            すると、
            \[ f(t, 0)=F(t, 0, 1)=F_d(t, 1) \]
            となるから、
            \[ i(C_0, L_0; \phi(P))=\max \{ k : t^k|F_d(t,1) \} \]
            となる。

            \paragraph{$\Phi$の表示を見る}
            一方$L$のパラメータ表示として$(X, Y, Z)=(t:0:s)$をとれば、
            $P(=(0:0:1))$に対応するのは$(s_0,t_0)=(1,0)$が与える点。
            したがって$\Phi(s, t)=F(t, 0, s)=F_d(t, s)$となり、
            あとは単なる計算で結論が得られる。
            \begin{eqnarray*}
                &{}&    I(C, L; P) \\
                &=&     \max \{ m : (s_0 t-t_0 s)^m | \Phi(s,t) \} \\
                &=&     \max \{ k : t^k|F_d(t,s) \} \\
                &=&     \max \{ k : t^k|F_d(t,1) \} \\
                &=&     i(C_0, L_0; \phi(P)) \\
            \end{eqnarray*}
            \QED
        \end{Proof}

    \subsection{射影変換}
    正則行列$A \in GL(3, k)$による線形写像
    \begin{eqnarray*}
        A : \affine^3 &\to& \affine^3 \\
        \begin{bmatrix}
            x \\ y \\ z
        \end{bmatrix}
        &\mapsto&
        A
        \begin{bmatrix}
            x \\ y \\ z
        \end{bmatrix}
    \end{eqnarray*}
    が定まる。
    任意の$\lambda  \in k$に対し、
    \[
        \lambda
        \begin{bmatrix}
            x \\ y \\ z
        \end{bmatrix}
        \mapsto
        A
        \begin{bmatrix}
            \lambda x \\ \lambda y \\ \lambda z
        \end{bmatrix}
    \]
    となるので、正則行列$A$によって
    \begin{eqnarray*}
        \phi_{A} : \proj^2 &\to& \proj^2 \\
        (a:b:c) &\mapsto& \psi_{Z}(A \cdot {}^t[a~b~c])
    \end{eqnarray*}
    が定まる。これはwell-definedである。

    明らかに以下が成り立つ。
    \begin{eqnarray*}
        \phi_{E} &=& id_{\proj^2} \\
        \phi_{AB} &=& \phi_{A} \circ \phi_{B} ~(\forall A, B \in GL(3,k))
    \end{eqnarray*}
    下の式から正則行列Aについて$\phi_A$は全単射となり、
    \[ (\phi_A)^{-1}=\phi_{A^{-1}} \]となる。
    特に射影変換全体
    \[ PGL(2, k):=\{ \phi_A : A \in GL(3,k) \} \]
    は群を成す。これを射影変換群と呼ぶ。

    \begin{Lemma}
        正則行列$A$が定める射影変換$\phi_A$を考える。
        3点$P_1, P_2, P_3 \in \proj^2$に対して、
        $P_1, P_2, P_3$が同一直線上に有ることと
        $\phi_A(P_1), \phi_A(P_2), \phi_A(P_2)$が同一直線上に有ることは同値。
    \end{Lemma}
    \begin{Proof}
        $P_i=(p_{i0}:p_{i1}:p_{i2}) \in \proj^2$に対して
        $\mathbf{p}_i={}^t[ p_{i0}, p_{i1}, p_{i2} ]$とおく。
        この時、アフィン空間に於いて2点を通る直線は行列式で書ける、
        という命題から、以下のように証明が出来る。
        \begin{eqnarray*}
            &{}&    \mbox{$P_1, P_2, P_3$が同一直線上に有る} \\
            &\iff&  \det[\mathbf{p}_1~\mathbf{p}_2~\mathbf{p}_3]=0 \\
            &\iff&  (\det A) (\det[\mathbf{p}_1~\mathbf{p}_2~\mathbf{p}_3])=0 \\
            &\iff&  \det[A\mathbf{p}_1~A\mathbf{p}_2~A\mathbf{p}_3]=0 \\
            &\iff&  \mbox{$\phi_A(P_1), \phi_A(P_2), \phi_A(P_2)$が同一直線上に有る} \\
        \end{eqnarray*}
        \QED
    \end{Proof}

    \begin{Prop}[Four Points Lemma]
        4点$P_1, P_2, P_3, P_4 \in \proj^2$はどの3点も同一直線上にないとする。
        $O_1=(1:0:0), O_2=(0:1:0), O_3=(0:0:1), O_4=(1:1:1)$
        とするとき、
        \[ \phi(P_i)=O_i ~(i=1,2,3,4)\]
        となる射影変換はただ一つ存在する。
    \end{Prop}
    \begin{Proof}
        $P_i$と$\mathbf{p}_i$と前のように定める。
        $B'=[\mathbf{p}_1~\mathbf{p}_2~\mathbf{p}_3]$
        と置けば、$P_i$はどの3つも同一直線上にないので$B'$は正則。
        $B=(B')^{-1}$と置くと、
        \[ B[\mathbf{p}_1~\mathbf{p}_2~\mathbf{p}_3]=B B'=E \]
        なので、$i=1,2,3$について$\phi_B(P_i)=O_i$となる。

        $\phi(P_4)$を考える。そのために
        \begin{gather}
        B \mathbf{p}_4=
        \begin{bmatrix}
            \lambda_1 \\ \lambda_2 \\ \lambda_3
        \end{bmatrix}
        =\sum_{i=1}^{4}{(\lambda_i \cdot B \mathbf{p}_i)} \label{astarisk}
        \end{gather}
        とする。
        この時、$\lambda_i \neq 0$である。
        実際、例えば$\lambda_1$とすると
        \[ \phi_B(P_2)=O_2,~ \phi_B(P_3)=O_3,~ \phi_B(P_4)=(0:\lambda_2:\lambda_3) \]となり、
        これらは直線$X=0$上にある。
        補題よりこれは3点$P_2, P_3,P_4$が同一直線上に有ることと同値であり、
        したがって仮定に反する。
        そこで正則行列Aを
        \[
        A=
        \begin{bmatrix}
            1/\lambda_1& {}& {} \\
            {}& 1/\lambda_2& {} \\
            {}& {}& 1/\lambda_3 \\
        \end{bmatrix}
        B
        \]
        と置けば、$\phi_A$が求める射影変換。
        実際に計算してみると、
        \begin{eqnarray*}
            &\phi_A(P_1)=\psi_{c}(A \mathbf{p}_1)=(\lambda_1:0:0)&=O_1 \\
            &\phi_A(P_2)=\psi_{c}(A \mathbf{p}_2)=(0:\lambda_2:0)&=O_2 \\
            &\phi_A(P_3)=\psi_{c}(A \mathbf{p}_3)=(0:0:\lambda_3)&=O_3 \\
            &\phi_A(P_4)=\psi_{c}(A \mathbf{p}_4)=(1 : 1: 1)    ~&=O_4
        \end{eqnarray*}

        もしも$A' \in GL(3,k)$によって$\phi_{A'}(P_i)=O_i$が成立したとする。
        この時ある定数$\alpha \in k^{\times}$によって$A=\alpha A'$となることを示す。
        この時、0でない定数$\mu_i$によって、
        \[
            A'[\mathbf{p}_1~\mathbf{p}_2~\mathbf{p}_3~\mathbf{p}_4]
            =
            \begin{bmatrix}
                \mu_1& 0& 0& \mu_4 \\
                0& \mu_2& 0& \mu_4 \\
                0& 0& \mu_3& \mu_4
            \end{bmatrix}
        \]
        と書ける。
        \[ \frac{1}{\mu_4}A'\mathbf{p}_4=\frac{1}{\mu_1}A'\mathbf{p}_1+\frac{1}{\mu_2}A'\mathbf{p}_2+\frac{1}{\mu_3}A'\mathbf{p}_3 \]
        また、式(\ref{astarisk})の左に$\frac{1}{\mu_4} A'B'$を掛けると、
        \[ \frac{1}{\mu_4}A'\mathbf{p}_4=\frac{\lambda_1}{\mu_4}A'\mathbf{p}_1+\frac{\lambda_2}{\mu_4}A'\mathbf{p}_2+\frac{\lambda_3}{\mu_4}A'\mathbf{p}_3 \]
        仮定より、$A'\mathbf{p}_i$は基底になっているから、係数が一致して
        \[ \frac{\lambda_i}{\mu_4}=\frac{1}{\mu_i} ~~(i=1,2,3,4) \]
        が成立する。したがって、
        \[ \mu_4 A [\mathbf{p}_1~\mathbf{p}_2~\mathbf{p}_3]=A'[\mathbf{p}_1~\mathbf{p}_2~\mathbf{p}_3] \]
        と、$B'=[\mathbf{p}_1~\mathbf{p}_2~\mathbf{p}_3]$が正則であることから、
        \[ \mu_4 A=A' \]
        が成立する。よって$\phi_A=\phi_{A'}$である。
        これで一意性が言えた。
        \QED
    \end{Proof}

    \begin{Lemma}
        斉次多項式$F \in k[X, Y, Z]$により定まる射影曲線$C:=Z_p(F)$を$\phi_A$で写した像は
        \[\phi_A(C)=Z(F \circ A^{-1}) \]
        さらに$\deg(\phi_A(C))=\deg C$である。
    \end{Lemma}
    \begin{Proof}
        任意の$P \in \proj^2$に対して、以下のようになる。
        \[ P \in \phi_A(C) \iff \phi^{-1}_A(P) \in C \iff F(A^{-1} P)=0 \iff P \in Z(F \circ A^{-1}) \]
        さらに、一般に行列$M$について$\deg F \geq \deg (F \circ M)$
        であることを用いて後半を証明する。
        \[
            \underbrace{\deg (F \circ A^{-1}) \leq \deg F}_{M=A^{-1}}
            =
            \underbrace{\deg (F \circ A^{-1} \circ A) \leq \deg(F \circ A^{-1})}_{M=A}
        \]
        \QED
    \end{Proof}

    \begin{Def}
        $F, G$を斉次多項式とし、$C:=Z(F), D:=Z(G)$とおく。$C,D$が射影同値であるとは、
        ある$A \in GL(3,k), \lambda \in k^{\times}$によって
        \[ G=\lambda F \circ A^{-1} \]
        となることである。
    \end{Def}
    \paragraph{注意}
    $k$が代数的閉包であるときは$\lambda'=\lambda^{1/\deg F} \in k$となるので、
    $F \circ (\lambda' A)^{-1}=\lambda F \circ A^{-1}$が成り立つ。
    つまり、定数$\lambda$を行列$A$に纏めることが出来る。

    \paragraph{例: 平行移動}
    アフィン空間における平行移動$(x,y) \mapsto (x-a, y-b)$を、
    射影化$\psi_{Z}$によって射影変換にする。
    \[ \phi_{A} : (X:Y:Z) \mapsto (X-aZ:Y-bZ:Z)\]
    このような射影変換$\phi_{A}$を与える正則行列$A$を求めよう。

    Four Points Lemmaより、射影変換は4点の写った先が決まれば一意に定まる。
    4点として$(0,0),(0,1),(1,0),(1,1)$をとり、これを射影化してから$\phi_A$で写す。
    するとその値から、
    \[
        A=
        \begin{bmatrix}
            1& {}& -a \\
            {}& 1& -b \\
            {}& {}& 1
        \end{bmatrix}
    \]
    と定まる。

\end{document}
