\documentclass[a4paper, dvipdfmx]{jsarticle}
\usepackage{macros}
\setenumerate{label=(\roman*),itemsep=3pt,topsep=7pt}

\newcommand{\Diag}{\Delta}
\newcommand{\CFG}[1]{\cat{CFG}(\cat{#1})}
\begin{document}
\title{ゼミノート \#8 \\ Algebraic-ness of Spaces and Stacks}
\author{七条彰紀}
\maketitle
affine scheme, scheme. algebraic space, algebraic stackという貼り合わせの連なりを意識した定義をした後,
algebraic stackがschemeの貼り合わせとして定義できることを示す.
algebraic spaceとalgebraic stackの定義は全く平行に行われる.
そのことが分かりやすい記述を志向する.

\section{Fiber Product of Fibered Categories}
    $\cat{B}$ :: categoryとする.
    この時,$\FibBP{B}$は以下のような圏であった.
    \begin{description}[labelindent=1cm]
        \item[Objects:] fibered categories over $\cat{B}$.
        \item[Arrows:]  base-preserving natural transtormation.
    \end{description}
    新たに圏$\CFG{B}$を以下のように定義する.
    \begin{description}[labelindent=1cm]
        \item[Objects:] categories fibered in groupoids(CFG) over $\cat{B}$.
        \item[Arrows:] base-preserving natural transtormation.
    \end{description}

重要なのは次の存在命題である.
\begin{Prop}[\cite{Gomez} p.10]
    $\FibBP{B}$と$\CFG{B}$はfibered productを持つ.
\end{Prop}
%% {{{
\begin{proof}
    $\FibBP{B}$の射$F \colon \fibX \to \fibZ$と$F \colon \fibY \to \fibZ$をとり,
    $F, G$のfiber productを実際に構成する.

    \step{圏$\cat{P}$の構成}
    圏$\cat{P}$を以下のように定義する.
    \begin{description}
        \item[Objects:]
            以下の$4$つ組
            \begin{enumerate}
                \item $b \in \cat{B}$,
                \item $x \in \fibX(b)$,
                \item $y \in \fibY(b)$,
                \item $\fibZ$の恒等射上の同型射$\alpha \colon Fx \to Gy$.
            \end{enumerate}
        
        \item[Arrows:] \mnewline
            射$(b, x, y, \alpha) \to (b', x', y', \alpha')$は,
            二つの射$\phi_{\fibX} \colon x \to x', \phi_{\fibY} \colon y \to y'$であって以下を満たすもの:
            $\phi_{\fibX}, \phi_{\fibY}$は同じ射$b' \to b$上の射で,
            以下の可換図式を満たすもの.
            \[
            \begin{tikzcd}
                Fx \ar[d, red, "F\phi_{\fibX}"']\ar[r, "\alpha"]& Gy \ar[d, red, "G\phi_{\fibY}"]\\
                Fx' \ar[r, "\alpha'"']& Gy'
            \end{tikzcd}
            \]
    \end{description}

    \step{Cartesian Lifting in $\cat{P}$.}
    この圏は$\pi \colon (b,x,y,\alpha) \mapsto b$によってfibered categoryと成る.
    $f \colon b' \to b$の$\xi=(b, x, y, \alpha)$に関する
    Cartesian Lifting :: $f^*\xi \to \xi$は次のように定義される.
    \[
        \chi_{\xi}=(f^*x \xrightarrow{\chi_x} x, f^*y \xrightarrow{\chi_y} y)
        \colon
        f^*\xi=(b', f^*x, f^*y, \bar{\alpha}) \to \xi.
    \]
    ここで$\chi_x \colon f^*x \to x$は$f$の$x$に関するCartesian Liftingである.
    $\chi_y$も同様.
    さらに$\bar{\alpha}$は以下のTriangle Liftingで得られる射である
    \footnote
    {
        $f^*\alpha \colon f^*Fx \to f^*Gy$とは異なる.
        同型$Ff^*x \to F^*Fx, Gf^*y \to f^*Gy$と$f^*\alpha \colon Fx \to Gy$を
        合成しても$\bar{\alpha}$は得られる.
    }.
    \begin{center}
    \begin{tikzpicture}[mybox/.style={draw, inner sep=5pt}]
    \node[mybox] (X) at (0,3){%
        \begin{tikzcd}
            Ff^*x \ar[d, "F\chi_x"']\ar[r, red, "\bar{\alpha}"]& Gf^*y \ar[d, "G\chi_y"]\\
            Fx \ar[r, "\alpha"']& Gy
        \end{tikzcd}
    };
    \node[mybox] (B) at (0,0){%
        \begin{tikzcd}
            b' \ar[d, "f"']\ar[r, red, "\id"]& b' \ar[d, "f"]\\
            b \ar[r, "\id"]& b
        \end{tikzcd}
    };

    \node [above=5pt of X] {in $\fibZ$};
    \node [below=5pt of B] {in $\cat{B}$};
    \draw [->, line width=1.5pt] (X) edge (B);
    \node at (0.3,1.55) {$\pi_{\fibZ}$};
    \end{tikzpicture}
    \end{center}
    fibered categoryの間の射はcartesian arrowを保つので$F\chi_x, G\chi_y$もcartesian.
    したがってTriangle Liftingが出来る.
    $\bar{\alpha}$が同型であることはTriangle Liftingの一意性を用いて容易に証明できる.
    また,この可換図式から$\chi_{\xi}$が$\cat{P}$の射であることも分かる.

    \step{$\fibX, \fibY, \fibZ$がcategory fibered in groupoids(CFG)ならば$\cat{P}$もCFG}
    $\fibX, \fibY, \fibZ$がCFGならば$\cat{P}$もCFGとなる.
    実際,
    $\phi_{\fibX} \colon x \to x'$と$\phi_{\fibY} \colon y \to y'$の両方がcartesianならば
    $(\phi_{\fibX}, \phi_{\fibY}) \colon (b, x, y, \alpha) \to (b', x', y', \alpha')$はcartesianである.

    \step{$\cat{P}$からの射影写像.}
    定義から明らかに$\pr_1 \colon \cat{P} \to \fibX, \pr_2 \colon \cat{P} \to \fibY$が定義できる.
    射の定義にある可換図式は,
    以下の$A$がnatural transformationであることを意味している.
    \begin{defmap}
        A \colon & F\pr_1& \to& G\pr_2 \\
        {}& (F\pr_1)((b, x, y, \alpha))=Fx& \mapsto& \alpha(Fx)=\alpha((F\pr_1)((b, x, y, \alpha)))
    \end{defmap}
    $A$がbase-preservingであることは$\alpha$が恒等射上のもの(i.e. $\pi_{\fibZ}(\alpha)=\id$)であることから,
    isomorphismであることは$\alpha$が同型であることから示される.

    \step{$\cat{P}$ :: fiber product.}
    今,
    $\fibW \in \CFG{B}$と
    射$S \colon \fibW \to \fibX, T \colon \fibW \to \fibY$及び
    base-preserving isomorphism :: $\delta \colon FS \to GT$をとる.
    base-preservingなので,任意の$w \in \fibW$について$\pi_{\fibZ}(FS(w))=\pi_{\fibZ}(GT(w))$.
    そこで次のように関手が定義できる.
    \begin{defmap}
        H\colon & \fibW& \to& \cat{P} \\
        \mathbf{Object}& w& \mapsto& (\pi_{\fibZ}(FS(w)), Sw, Tw, \delta_{w}) \\
        \mathbf{Arrow}& [\phi \colon w \to w']& \mapsto& (S\phi \colon Sw \to Sw', T\phi \colon Tw \to Tw')
    \end{defmap}
    このように置くと,$S=\pr_1 H, T=\pr_2 H$となる.
    逆に$S \iso \pr_1 H', T \iso \pr_2 H'$となる
    関手$H' \colon \fibW \to \cat{P}$は$H$と同型に成ることが直ちに分かる.
\end{proof}
%% }}}
\begin{Remark}
    session4 命題4.5より,CFGの恒等射上の射は同型射である.
    したがって$\alpha \colon Fx \to Gy$に課せられた条件は,
    $\fibZ$がCFGならば一つしか無い.
\end{Remark}

\begin{Example}
    representable fibered categoryのfiber product.
\end{Example}

我々が扱うのはstackであるから,
stackという性質がfiber productで保たれていて欲しいが,果たしてそうなる.
\begin{Prop}[\cite{ASS} Prop 4.6.4]
    $\fibX, \fibY, \fibZ$ :: stack over $\cat{C}$とし,
    morphism of stacks :: $F \colon \fibX \to \fibZ, G \colon \fibY \to \fibZ$をとる.
    この時,$F, G$についてのfiber product :: $\fibX \times_{\fibZ} \fibY$はstackである.
\end{Prop}
%% {{{
\begin{proof}
    $\fibP=\fibX \times_{\fibZ} \fibY$とおく.
    $U \in \cat{C}, \covU=\{\phi_i \colon U_i \to U\} \in \Cov(U)$を任意にとり,
    $\epsilon_{\covU} \colon \fibP(U) \to \fibP(\covU)$を計算する.

    \step{$\epsilon_{\covU}(\xi)$.}
    $\xi=(b, x, y, \alpha)$をとり,$\epsilon_{\covU}(\xi)$を計算する.
    まず$\{\phi_i^*\xi\}_i$は既に詳しく説明した.
    注意が必要なのは同型$\sigma_{ij} \colon \pr_2^*\phi_i^*\xi \to \pr_1^*\phi_j^*\xi$である.
    可換性は以下の図式から分かる.
    \begin{center}
    \begin{tikzpicture}[mybox/.style={draw, inner sep=5pt}]
    \node[mybox] (X) at (0,5){%
        \begin{tikzcd}
            F\pr_2^*\phi_j^*x \ar[d, "\sigma_{ij}^{x}"']\ar[r, red, "\bar{\alpha}"]&
            G\pr_2^*\phi_j^*y \ar[d, "\sigma_{ij}^{y}"]\\
            F\pr_1^*\phi_i^*x \ar[d]\ar[r, red, "\bar{\alpha}"']& G\pr_1^*\phi_i^*y \ar[d]\\
            Fx \ar[r, "\alpha"']& Gy
        \end{tikzcd}
    };
    \node[mybox] (B) at (0,0){%
        \begin{tikzcd}
            b' \ar[r, red, equal]\ar[d, equal]& b' \ar[d, equal] \\
            b' \ar[r, red, equal]\ar[d, "\phi_j \circ\, \pr_2"']& b' \ar[d, "\phi_i \circ\, \pr_1"]\\
            b \ar[r, equal]& b
        \end{tikzcd}
    };

    \node [above=5pt of X] {in $\fibZ$};
    \node [below=5pt of B] {in $\cat{B}$};
    \draw [->, line width=1.5pt] (X) edge (B);
    \node at (0.3,2.45) {$\pi_{\fibZ}$};
    \end{tikzpicture}
    \end{center}
    
    \step{$\epsilon_{\covU}(\kappa)$.}
    (TODO)

\end{proof}
%% }}}
\section{Representable Morphism}
\begin{Remark}
    以下,$S$ :: schemeとし,$\Sch/S$上のsiteを$\cat{C}$と書く
    ($(\Sch/S)_{\tau}$といった表記も見かける).
    また,stackといえばstack in groupoidに限る.
\end{Remark}

\begin{Remark}
    scheme :: $S$は$\Sch/S$によってstackとみなす.
    また,sheaf:: $\shF$はGrothendiek construction :: $\int \shF$によってstackとみなす.
\end{Remark}

\begin{Def}[Representable by Scheme/Space]
    stack :: $\fibX$がrepresentable by scheme (resp. algebraic space)であるとは,
    あるscheme :: $X$ (resp. space $\shX$)が存在し,
    $\fibX \iso X=\Sch/X$ (resp. $\fibX \iso \shX=\int \shX$)であるということ.
\end{Def}

\begin{Def}[Representability of Morphism of Spaces/Stacks]
    \enumfix
\begin{enumerate}
\item
    morphism of spaces :: $f \colon \shX \to \shY$がrepresentable( by scheme)であるとは,
    任意の$S$-scheme :: $U$と$\cat{C}$の射$U \to \shY$について,
    fiber product :: $U \times_{\shY} \shX$(これはspace)がrepresentable by schemeであるということ.

\item
    morphism of stacks :: $f \colon \fibX \to \fibY$がrepresentable( by algebraic space)であるとは,
    任意の$S$-space :: $U$と$\cat{C}$の射$U \to \fibY$について,
    fiber product :: $U \times_{\fibY} \fibX$
    (これはstack)がrepresentable by algebraic spaceであるということ.
\end{enumerate}
\end{Def}

\begin{Lemma}
    morphism of stacks :: $f \colon \fibX \to \fibY$がrepresentable by algebraic spaceであることは,
    任意の\underline{$S$-scheme} :: $U$と射$U \to \fibY$について,
    fiber product :: $U \times_{\fibY} \fibX$
    (これはstack)がrepresentable by algebraic spaceであることと同値.
\end{Lemma}
(TODO: algebraic space定義の前に現れている.)

\section{Property of Representable Space/Stack/Morphism}

\begin{Def}
%    まずspaceとmorphism of spacesについて定義する.
%\begin{enumerate}
%\item
%    $\mathcal{P}$をschemeの性質とする.
%    この時,representable space :: $\shX$が性質$\mathcal{P}$を持つとは,
%    $\shX$をrepresentするschemeが性質$\mathcal{P}$を持つということである.

%\item
%    $\mathcal{P}$をmorphism of schemesの性質とする.
%    この時,representable morphism of spaces :: $f \colon \fibX \to \fibY$が性質$\mathcal{P}$を持つとは,
%    任意の$U \in \cat{C}$と射$U \to \fibY$について,
%    $\pr \colon \fibX \times_{\fibY} U \to U$
%    (に対応するmorphism of algebraic schemes)が性質$\mathcal{P}$を持つということである.
%\end{enumerate}

%    次にstackとmorphism of stacksについて定義する.
%    これらは上の定義を殆ど機械的に置換すれば得られる.
    \enumfix
\begin{enumerate}
\item
    $\mathcal{P}$をschemeの性質であって,local for etale topologyであるものとする.
    この時,\underline{representable} stack :: $\shX$が性質$\mathcal{P}$を持つとは,
    $\shX$をrepresentするalgebraic space (resp. scheme)が性質$\mathcal{P}$を持つということである.

\item
    $\mathcal{P}$をmorphism of schemeの性質であって,
    local on the targetかつstable under base changeであるものとする
    (ここの部分は\cite{ASS}と\cite{Gomez}\&\cite{IrrOfMg}で異なる).
    この時,\underline{representable morphism of algebraic stacks} :: $f \colon \fibX \to \fibY$が
    性質$\mathcal{P}$を持つとは,
    任意の$U \in \cat{C}$と射$U \to \fibY$について,
    $\pr \colon \fibX \times_{\fibY} U \to U$
    (に対応するmorphism of algebraic spaces)が性質$\mathcal{P}$を持つということである.
\end{enumerate}
\end{Def}

\section{Diagonal Map}
\begin{Def}[Diagonal Map]
    $\fibX/S$(すなわち射$\fibX \to S$)のdiagonal map :: $\Diag$とは,
    以下の可換図式に収まる射のことである.
    \[\begin{tikzcd}
            \fibX \ar[rrd, bend left, "\id"]\ar[rdd, bend right, "\id"']\ar[rd, "\Diag"]&
                                                            & \\
                                                            &
        \fibX \times \fibX \ar[r]\ar[d]\ar[rd, phantom, "\text{p.b.}"]& \fibX \ar[d]\\
          &\fibX \ar[r]& S
    \end{tikzcd}\]
\end{Def}

\begin{Prop}
    $\fibF$ :: stack on $\tau(S)$
    以下は同値である.
    \begin{enumerate}[label=(\roman*)]
        \item $\Diag \colon \fibX \to \fibX \to \fibX$ :: representable.
        \item 任意のscheme :: $U$について,$U \to \fibX$ :: representable.
        \item 任意のscheme :: $U, V$と射$U \to \fibX, V \to \fibX$について$U \times_{\fibX} V$ :: representable.
    \end{enumerate}
\end{Prop}
\begin{proof}
    (TODO)
\end{proof}

\section{Algebraic-ness}
\subsection{Definition}
\begin{Def}[Algebraic Space]
    $S$ :: schemeとし,$\shX$をspace over $S$(すなわちbig etale site $\Et(S)$上のsheaf)とする.
    $\shX$がalgebraicであるとは,次が成り立つということである.
\begin{enumerate}
    \item diagonal morphism :: $\Diag \colon \shX \to \shX \times_{S} \shX$がrepresentableである.
    \item scheme :: $U$からのetale surjective morphism :: $U \to \shX$が存在する.
\end{enumerate}
\end{Def}

\begin{Def}[Algebraic Stack][\cite{ASS}, \cite{IrrOfMg}]
    $S$ :: scheme, $\fibX$をstack in groupoid over $S$
    (すなわちbig etale site $\Et(S)$上のstack in groupoid)とする.
    $\fibX$がalgebraicであるとは,次が成り立つということである.
\begin{enumerate}
    \item diagonal morphism :: $\Diag \colon \fibX \to \fibX \times_{S} \fibX$がrepresentableである.
    \item algebraic space :: $U$からのetale surjective morphism :: $U \to \fibX$が存在する.
\end{enumerate}
    ここに現れる$U \to \fibX$は$\fibX$のatlasと呼ばれる.
\end{Def}

\begin{Remark}
    以上で定義したものはいわゆる``Deligne-Mumford stack"の直接の一般化である.
    通常は上記に加えて$\Diag$にquasi-compact, separatedという条件を課す.
    (ただし,実際にDeligneとMumfordがDM stackを導入したとされる\cite{IrrOfMg}での定義は上と全く同じである.)
    $\Diag$ :: quasi-compact, separatedかつ$U \to \fibX$にsmoothのみ要求するものは
    ``Artin stack"と呼ばれる.
\end{Remark}

\begin{Remark}
    stack :: $\fibX$へのalgebraic spaceからの射$U \to \fibX$が存在すれば,
    algebraic spaceの定義より,schemeから$\fibX$への射が存在する.
    surjective, etale, smoothなどの性質は合成について安定なので,
    algebraic stackの定義の二つ目の条件は「scheme :: $U$からの……」と書き換えられる.
\end{Remark}

\section{Property of Space/Stack/Morphism of Them}
\begin{Def}[\cite{IrrOfMg} p.100, Local Property for the topology.]
    $S$ :: schemeとし,$(\Sch/S)$上のsite :: $\cat{C}$を考える.
    $X, Y$ :: schemeとし,
    $\{\phi_i \colon X_i \to X\} \in \Cov(X), \{\psi_i \colon Y_i \to Y\} \in \Cov(Y)$を任意に取る.
    \begin{enumerate}
        \item 
            $P$をschemeの性質とする.
            $P$がlocal for the topologyであるとは,以下が成り立つということ: \mnewline
            $X$が$P$であることは,全ての$U_i$が$P$であることと同値.
        \item
            $P$をschemeの射の性質とする.
            $P$がlocal on the sourceであるとは,以下が成り立つということ:\mnewline
            $f \colon X \to Y$が$P$であることは,
                全ての$f \circ \phi_i$が$P$であることと同値.
        \item
            $P$をschemeの射の性質とする.
            $P$がlocal on the targetであるとは,以下が成り立つということ:\mnewline
            $f \colon X \to Y$が$P$であることは,
                全ての$\pr_2 \colon X \times_Y Y_i \to Y_i$が$P$であることと同値.
        \item
            (\cite{ASS} 5.1.3)
            $P$をschemeの射の性質とする.
            以下が全て成り立つ時,$P$はstableであると呼ばれる.
            \begin{itemize}
                \item 任意の同型は$P$.
                \item $P$は,射の合成で保たれる.
                \item $P$は,任意の$\cat{C}$の射によるbase changeで保たれる.
                \item local on the target.
            \end{itemize}
        \item
            (\cite{Gomez} 2.5)
            $P$をschemeの射の性質とする.
            以下が全て成り立つ時,$P$はlocal on the source and targetであると呼ばれる.:
            任意の以下の可換図式について,$f$が$P$であることは$f'$が$P$であることと同値.
            \[
            \begin{tikzcd}
                X' \ar[r] \ar[rd, "f'"']& Y' \times X \ar[r]\ar[d]& X \ar[d, "f"]\\
                {} & Y' \ar[r]& Y \ar[lu, phantom, "\text{p.b.}"]
            \end{tikzcd}
            \]
            ただし$X' \to Y' \times X, Y' \to Y$は,Artin (resp. DM) stackを考えているならば
            smooth (resp. etale) and surjectiveである.
    \end{enumerate}
\end{Def}

\begin{Remark}
    local on the source and targetは,
    $\ET(S)$を考えているならば次と同値:
    任意の以下の可換図式について,$f$が$P$であることは$f'$が$P$であることと同値.
    \[
    \begin{tikzcd}
        X' \ar[r]\ar[d, "f'"']&  X \ar[d, "f"]\\
        Y' \ar[r]& Y
    \end{tikzcd}
    \]
    ただし$X' \to Y' \times X, Y' \to Y$はetale, surjective morphismである.
\end{Remark}
\begin{Lemma}
    あるatlas :: $U$がlocal for etale topologyな性質を持つならば,
    任意のaltasがその性質を持つ.
\end{Lemma}

\begin{Def}[Property of Algebraic Stacks]
    \enumfix
\begin{enumerate}
\item
    $\mathcal{P}$をschemeの性質であって,
    local for etale topologyであるものとする.
    この時,algebraic stack :: $\fibX$が性質$\mathcal{P}$を持つとは,
    $\fibX$のatlasが性質$\mathcal{P}$を持つということである.

\item
    algebraic stack :: $\fibX$がquasi-compact
    \footnote{ 明らかに,これはlocal for etale topologyではない. }であるとは,
    $\fibX$のatlasが性質$\mathcal{P}$を持つということである.

\item
    $\mathcal{P}$をmorphism of schemeの性質であって,
    local on the source and targetであるものとする.
    この時,\underline{morphism of algebraic stacks} :: $f \colon \fibX \to \fibY$が
    性質$\mathcal{P}$を持つとは,
    以下の可換図式にある$f'$(に対応するmorphism of algebraic spaces)が
    性質$\mathcal{P}$を持つということである.
    \[
    \begin{tikzcd}
        X' \ar[r] \ar[rd, "f'"']& Y' \times \fibX \ar[r]\ar[d]& \fibX \ar[d, "f"]\\
        {} & Y' \ar[r]& \fibY \ar[lu, phantom, "\text{p.b.}"]
    \end{tikzcd}
    \]
    ただし$X' \to Y' \times \fibX, Y' \to \fibY$は,Artin (resp. DM) stackを考えているならば
    smooth (resp. etale) and surjectiveである.
\end{enumerate}
\end{Def}

\begin{Example}
    \begin{enumerate}
        \item local on the source and targetである性質の例:
            flat, smooth, etale, unramified, normal,
            locally of finite type, locally of finite presentation.
    \end{enumerate}
\end{Example}

\bibliographystyle{jplain}
\bibliography{reference}
\end{document}
