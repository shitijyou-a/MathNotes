\documentclass[a4paper]{jsarticle}
\usepackage{macros}

\begin{document}
\title{ゼミノート \#7 \\ Descent Theory}
\author{七条彰紀}
\maketitle

\section{Motivation}
    (TODO)

\section{Definition}
\begin{Def}
    関手$\epsilon_{\covU} \colon \shF(U) \to \shF(\covU)$を用いて以下のように定義する.
    \begin{enumerate}[label=(\roman*)]
        \item
            $\epsilon_{\covU}$ :: equivalenceとなる$\covU$をof effective descent for $\shF$と呼ぶ.
        \item
            $\epsilon_{\covU}$の像と同型である$\shF(\covU)$の対象を,effective dataという.
    \end{enumerate}
\end{Def}

\section{Criterion for fpqc Stacks}
\begin{Lemma}[\cite{ASS} Lemma 4.25]
    $S$ :: scheme,
    $\shF \to (\Sch/S)$ :: fibrationとする.
    以下が成り立つとする.
    \begin{enumerate}[label=(\alph*)]
        \item $\shF$はZariski topologyでのstackである.
        \item
            任意のflat surjective morphism of affine $S$-scheme :: $V \to U$について,\mnewline
            $\epsilon_{\{V \to U\}} \colon \shF(U) \to \shF(\{V \to U\})$は圏同値.
    \end{enumerate}
    この時,$\shF$はfpqc topologyでのstackである.
\end{Lemma}

証明のために段階を踏む.
\subsection{Step 1 / \tp{$\shF$}{F} :: fpqc prestack.}

\subsection{Step 2 / single morphism coverの場合に帰着させる.}

\subsection{Step 3 / affine schemeへのquasi-compact morphismの場合.}

\subsection{Step 4 / affine schemeへのflat surjective morphismの場合.}

\subsection{Step 5 / 一般の場合.}

\bibliographystyle{jplain}
\bibliography{reference}
\end{document}
