\documentclass[a4paper]{jsarticle}
\usepackage{macros}

\begin{document}
\title{ゼミノート \#4.5 \\ Fibered Categories, continued}
\author{七条彰紀}
\maketitle

\section{Grothendieck Construction} \label{sec:gro_const}
    今,fibered categoryからfiberとしてpsuedo-functorを構成した.
    実はこの逆が出来る.
    \begin{Def}[Grothendieck Construction]
        psuedo-functor :: $P \colon \cat{B} \to \Cat/\cat{B}$について,
        以下のように圏$\int P$を定義する.
        \begin{description}[labelindent=1cm]
            \item[Object.] $b \in \cat{B}$と$x \in P(b)$の組$(b, x)$.
            \item[Arrow.] $\phi \colon b \to b'$と$\Phi \colon P(\phi)(x) \to x'$の組$(\phi, \Phi)$.
        \end{description}
        射の合成は$(\psi, \Psi) \circ (\phi, \Phi)=(\psi \circ \phi, \Phi \circ P(\psi)(\Phi))$で与えられる.
        
        この圏によって以下の関手が定まる.
        \begin{defmap}
            \int \colon & \left\{ \parbox{2.3cm}{psuedo-functor \\ \quad \ $\cat{B} \to \Cat$} \right\}&
                \to& \Fib{B} \\
            {}& P& \mapsto& \int P
        \end{defmap}
    \end{Def}

    \begin{Example}
        $\ftor{S}$は$\Sch/S$に対応する.
        $F \colon \cat{C} \to \Sets$は$\bigsqcup_{c \in \cat{C}} F(c)$に対応する.
    \end{Example}

    \begin{Remark}
        David I. Spivak ``Category theory for scientists"によると,
        Grothendieck Constructionを最初に構成したのはGrothendieckではない.
        例えばMacLaneが以前から扱っている.
    \end{Remark}

    \begin{Thm}[Grothendieck Construction give Category Equivalence]
        Grothendieck Construction
        \[
            \int \colon  \left\{ \parbox{2.3cm}{psuedo-functor \\ \quad \ $\cat{B} \to \Cat$} \right\}
                \to \Fib{B}
        \]
        は圏同値である.
    \end{Thm}
    \begin{proof}
        P. T. Johnstone
            ``Sketches of an Elephant: A Topos Theory Compendium vol.1 (Oxford Logic Guides 43)"
        に証明がある
        (この文献で言うcloven fibered categoryが我々の定義するfibered categoryである).
    \end{proof}


\section{Category Fibered in Groupoids/Sets}
\subsection{Motivation}

\subsection{Definition}
    \begin{Def}[Groupoid, Category fibered in groupoids/sets]
    \end{Def}

    \begin{Def}[Category fibered in groupoid (Another Definition)]
    \end{Def}

\subsection{Propositions}
    \begin{Prop}
        $\Hom$がgroupoid.
    \end{Prop}

    \begin{Thm}[$2$-Yoneda Lemma]
    \end{Thm}

\section{Splittings of fibered categories}
    \begin{Def}
        
    \end{Def}

\section{Fiber Product of Category Fibered in Sets/Groupoids}
\subsection{Definition}

\subsection{Propositions}

\section{Equivalences of Fibered Categories}
\subsection{Definition}
    Equivalence, Fully Faithful.

\subsection{Propositions}

\bibliographystyle{jplain}
\bibliography{reference}
\end{document}
