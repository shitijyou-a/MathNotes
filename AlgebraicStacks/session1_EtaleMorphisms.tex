\documentclass[a4paper]{jsarticle}
\usepackage{macros, enumitem}
%\setlist[description]{style=nextline}
\newcommand{\diag}{\Delta}

\begin{document}
\title{ゼミノート \#1 \\ Etale Morphisms}
\author{七条彰紀}
\maketitle
\cite{HaMo}
\section{定義}
\begin{Def}[Infinitesimal Thickening, Formally Smooth/Unramified/Etale]
    \begin{enumerate}[label=(\roman*), leftmargin=*]
    \item
    $i \colon Y_0' \inclmap Y'$ :: closed embeddingについて,
    defining ideal :: $\ker i^{\#}$がnilpotent \footnote{i.e. $\Exists{n > 0} (\ker i^{\#})^n=0$}であるとき,
    $Y_0'$を$Y'$のinfinitesimal thickening(無限小肥大?)と呼ぶ.
    あるいは$i$をinfinitesimal thickeningと呼ぶ.

    \item
    $Y'$ :: affine $Y$-scheme, $Y_0' (\inclmap Y')$ :: infinitesimal thickening of $Y'$とする.
    $f \colon X \to Y$について,以下の図式を見よ.
    \[\xymatrix{
        Y_0' \ar@{_{(}->}[d]_-{\text{inf. thi.}} \ar@[red][r]& X \ar[d]^-{f} \\
        Y' \ar[r] \ar@[blue][ru]& Y
    }\]
    この時,次の写像が定まる.
    \begin{defmap}
        {}& \Hom_Y(Y', X)& \to& \Hom_Y(Y_0', X) \\
        {}& {\color{blue}\to}& \mapsto& {\color{red}\to}
    \end{defmap}
    この写像がsurjective injective,bijectiveであるとき,
    それぞれformally smooth, formally unramified, formally etaleという.
\end{enumerate}
\end{Def}

\begin{Def}[(Locally) Of Finite Presented Module/Algebra/Sheaf/Morphism]
    \begin{enumerate}[label=(\roman*), leftmargin=*] \hfill \vspace{-0.5cm}
    \item
        $R$-module :: $M$がfinitely presented moduleであるとは,
        次の完全列が存在すること.
        \[\xymatrix{
            A^{\oplus r} \ar[r]& A^{\oplus s} \ar[r]& M \ar[r]& 0
        }\]

    \item
        surjective ring homomorphism :: $\phi \colon R[x_1, \dots, x_s] \to A$が存在し,
        $\ker \phi$がfinitely generated idealであるとき,
        $A$ :: finitely presented $R$-algebra (of finite presentation over $R$)という.

    \item
        $\shF$ :: quasi-coherent sheaf on a scheme $X$とする.
        $\shF$ :: locally finitely presentedとは,
        任意のaffine open subscheme of $X$ :: $\Spec A \subseteq X$について,
        $\Gamma(\Spec B, \shF)$がfinitely presented $B$-\underline{module}であること.

    \item
        $f \colon X \to Y$ :: locally of finite presentationであるとは,
        任意の$\Spec B \subseteq Y$と$\Spec A \subseteq f^{-1}(\Spec B)$について,
        $A$ :: finitely presented $B$-algebraであるということ.
        あるいは(同値な条件として),
        affine open cover of $Y$ :: $Y=\bigcup_i \Spec B_i$が存在して,
        任意の$\Spec A_{ij} \subseteq f^{-1}(\Spec B_i)$について,
        $A_{ij}$ :: finitely presented $B_{i}$-algebraであるということ.

    \item
        $f \colon X \to Y$がquasi-compactであるとは,
        任意のaffine open subset of $X$ :: $\Spec A$について$f^{-1}(\Spec A)$ :: quasi-compactであること.
        あるいは(同値な条件として),
        affine open cover of $Y$ :: $Y=\bigcup_i \Spec B_i$が存在して,
        $f^{-1}(\Spec B_i)$ :: quasi-compactであること.
        
    \item
        $f \colon X \to Y$がquasi-separatedであるとは.
        またdiagonal morphism :: $\Delta \colon X \to X \times_{Y} X$
        \footnote
        {
            $\Delta$は以下のようにpullbackの普遍性から得られる射である.
            \[\xymatrix{
                    X \ar@/^5mm/[rr]\ar[rd]\ar@{-->}[r]^-{\Delta}& X \times X \ar[d]\ar[r] & X\ar[d]^-{f} \\
                {}& X\ar[r]_-{f} & Y \ar@{}[lu]|{\text{p.b.}}
            }\]
        }
        がquasi-compactであること.

    \item
        $f \colon X \to Y$がlocally of finite presentationかつquasi-compactかつquasi-separatedである時,
        $f$ :: finitely presentedという.
\end{enumerate}
\end{Def}
環$R$やscheme :: $Y$をnoetherianとすれば,
(locally) of finite presentationと(locally) of finite typeは同値に成る.
一般に(locally) of finite presentationの方が強い条件である(例を参照せよ).

\begin{Def}[Smooth/Unramified/Etale]
    morphism :: $f \colon X \to Y$は,
    formally smooth / unramified / etaleかつfinitely presentedならば
    smooth / unramified / etaleという.
\end{Def}
unramifiedについては,finite typeのみ要求する定義もある.
finitely presentedを要求するのはEGAからのもので,
我々が主に参照している\cite{ASS}もこの定義を取っている.

\section{定義に対する例}

\begin{Example}\label{example:not_qsep}
    locally of finite presentationかつquasi-compactだがNOT quasi-separatedである例を挙げる.

    以下のように設定する.
    \begin{itemize}
        \item $k$ :: field,
        \item $Y=\Spec k[x_1,x_2,\dots]$,
        \item $z=(x_1, x_2, \dots) \in Y$,
        \item $U=Y-\{z\}$.
    \end{itemize}
    この時,$U$はquasi-compactでない.
    これは$U$ :: quasi-compact $\iff$ $z$ :: finitely generatedからわかる
    \footnote
        {
            私のノート: \url{https://github.com/ShitijyouA/MathNotes/blob/master/Hartshorne_AG_Ch2/section2_ex.pdf}
            補題 Ex2.13.2 (II)に証明がある.
        }.

    $X$を,二つの$Y$のコピーを$U$で貼り合わせたものとし,
    $X_1, X_2 \subseteq X$をその$Y$のコピーとする.
    すなわち$X_1, X_2 \iso Y$.
    この同型を$\phi_i \colon X_i \to Y$と名付ける.
    このとき,$f \colon X \to Y$を$\phi_1, \phi_2$の$U$に沿った貼り合わせとする.
    こうすると$f|_{X_i}=\phi_i$となる.

    \paragraph{$f$ :: locally of finite presentation.}
    $Y$ :: affine schemeで,$f^{-1}(Y)=X_1 \cup X_2$であり,
    $X_1, X_2 \iso Y$であった.
    なので$f$ :: locally of finite presentation.

    \paragraph{$f$ :: quasi-compact.}
    同じく,$X_1, X_2$ :: quasi-compactなので$f^{-1}(Y)=X_1 \cup X_2$がquasi-compact.

    \paragraph{$f$ :: NOT quasi-separated.}
    $\basesp (X \times_Y X)$と$\diag \colon X \to X \times_Y X$を考えると次のように成る.
    \[ \diag \colon x \mapsto (\phi_1^{-1}(x), \phi_2^{-1}(x)). \]
    一方,$X_1 \times_{Y} X_2 (\subset X \times X)$は,$X_1, X_2 (\iso Y)$がaffineなのでaffine.
    そこで逆像$\diag^{-1}(X_1 \times_{Y} X_2)$を取ると,これは$U$である.
    既に述べたとおり,これはNOT quasi-compact.
\end{Example}

\begin{Example}
    \paragraph{Smooth (BUT NOT Etale) Morphism.}
    次のように定める.
    \begin{defmap}
        f\colon& \Spec k[x,y]& \to& \Spec k[t] \\
        {}& (x,y)& \mapsto& x^2+y^2
    \end{defmap}
    これはaffine schemeの間の射なのでquasi-separated.
    $f^{-1}(\Spec k[t])=\Spec k[x, y]$がnoetherian schemeなのでfinitely presented.
    あとはformally smoothであることを示せば良い.
    
    \paragraph{Unramified (BUT NOT Etale) Morphism.}
    次のように定める:
    \begin{alignat*}{2}
        g \colon \Spec \Q[x] \sqcup \Spec \Q[y]& \to \Spec \Q[t] \\
        x& \mapsto t && \text{ on } \Spec \Q[x]\\
        y& \mapsto t && \text{ on } \Spec \Q[y]
    \end{alignat*}
    $f$の場合と同様に,formally unramifiedだけ示せば良い.

    \paragraph{Etale Morphism.}
    \begin{defmap}
        h\colon& \Spec \Q[u, u^{-1}, y]/(y^d-u)& \to& \Spec \Q[t, t^{-1}] \\
        {}& (u, y)& \mapsto& u
    \end{defmap}
    $A=\Q[t, t^{-1}], B=\Q[u, u^{-1}, y]/(y^d-u)$とおくと.
    $h$に対応する環準同型は$h^{\#} \colon A \to B; t \mapsto ua \bmod (y^d-u)$.
    $f$の場合と同様に,formally etaleだけ示せば良い.
    
    以下の図式を考える.
    \[\xymatrix{
            B \ar@[red][r]^-{\alpha}\ar@[blue][rd]^-{\beta}& R/I \\
            A \ar[u]^-{h^{\#}} \ar[r]_-{\phi}& R \ar[u]_-{\pi}
    }\]
    ここで$I \subseteq R$はイデアルで,$I^N=0$となる整数$N>0$が存在する.
    与えられた$\alpha$から図式を可換にする$\beta$を構成し,
    このような$\beta$が$\alpha$に対し唯一つであることを示す.
    まず$\beta$は$t \in B$の像のみで定まることに注意する.
    図式が可換であることと,次が成立することは同値.
    \[ \beta h^{\#}(t)=\beta(u)=\phi(t), \qquad \pi \beta(u)=\alpha(u) \]
    よって$\beta(u)=\phi(t)$で$\beta$を定めれば良い.
    このように定めれば後者も成立する.
    また,この構成から明らかに$\beta$はただ一つ.

    \paragraph{Formally Etale BUT NOT Etale Morphism.}
    例(\ref{example:not_qsep})のmorphism :: $f \colon X \to Y$がそうである.
    このことを示すには,Formally etaleであることだけ確かめれば十分.

\end{Example}

\section{命題}

\begin{Prop}[\cite{ASS} Prop1.3.6 (i)]
    $f \colon X \to Y$をmorphism of schemesとする.
    この時$\shDer_{X/Y}$は次のように成る.
\begin{enumerate}[label=(\roman*)]
    \item $f$ :: smooth \quad \ \,$\implies \shDer_{X/Y}$ :: locally free sheaf of finite rank.
    \item $f$ :: unramified $\iff \shDer_{X/Y}=0$.
    \item $f$ :: etale $\implies \shDer_{X/Y}=0$.
\end{enumerate}
\end{Prop}
\begin{proof}
    証明は\cite{Mat} \S 25の内容を一部使う.
    特に\S 25始めからThm25.1の直前までがわかっていれば良い.

    主張はlocalなものだから,$X=\Spec B, Y=\Spec A$と仮定して良い.
    $f$ :: smoothより$B$ :: finitely presented $A$-algebra.
    $f$に対応する準同型を$\phi \colon A \to B$とする.

    (i)を示すために,$\modDer_{B/A}$ :: projective $B$-moduleを示す
    (projectiveならばlocally freeであることは\cite{StacksProj} section 10.84に証明がある).
    これはすなわち,$B$-moduleの以下の図式に対し,
    図式を可換にする$\tilde{D} \colon \modDer_{B/A} \to M$が存在するということである.
    \[\xymatrix{
        {} & \modDer_{B/A} \ar[d]^-{D}\\
        M \ar@{->>}[r]^-{t}& N
    }\]
    ここで$t$ :: surj.
    
    次の図式を考える.
    \[\xymatrix{
        B \ar[r]^-{f_D}& B[N] \\
        A \ar[r]\ar[u]^-{\phi}& B[M]\ar@{->>}[u]
    }\]
    ここで$B[M]$は\cite{Mat} \S 25でいう$B \ast M$である
    \footnote
    {
        これらは$B$-algebraで,加群としては$B \oplus M$で,
        乗法は$(b, m) \cdot (b', m')=(bb', bm'+b'm)$で定まる.
        重要な特性として,$\pi_M: B[M] \to B; (b, m) \mapsto b$のkernelはsquare-zeroで,
        $\pi_M$の$A$-algebra section (section which is $A$-albgebra morphism)と
        $A$-derivation $B \to M$が一対一に対応する.
    }.
    $B[N]$も同様.
    $f_D$は$A$-derivation :: $D$に対応する射$b \mapsto (b, D(b))$である.
    $B[M] \to B[N]$は$(b, m) \mapsto (b, t(m))$で与えられる射で,
    したがって全射であり核は$0 \oplus (\ker t)$.
    これはsquare-zero idealである.
    そして$\phi$ :: formally smoothであるから,
    図式を可換にする$B \to B[M]$が存在する.
    これに対応する$A$-derivationが所望の$\tilde{D}$である.

    (ii)を示す.
    $R$ :: ring, $I \subseteq R$ :: idealを$I^2=0$を満たすものとする.
    以下が可換図式だったとしよう.
    \[\xymatrix{
        B \ar[r]^-{\theta} \ar[rd]_-{\lambda}& R/I \\
        A \ar[r]\ar[u]^-{\phi}& R\ar[u]_-{\pi}
    }\]
    この時,$\lambda$をlifting of $\theta$と呼ぶ.
    \cite{Mat} \S 25より
    \footnote
    {
        あるいは私のノート
        \url{https://github.com/ShitijyouA/MathNotes/blob/master/Hartshorne_AG_Ch2/section8_ex.pdf}
        のEx8.6(a)の解答より.
    },
    \[
        \Hom_A(\modDer_{B/A}, I)
        =\Der_{A} (B, I)
        =\{\lambda−\lambda′ | \lambda, \lambda' \text{ :: lifting of $\theta$} \}
    \]
    となっている.
    $\phi$ :: formally unramifiedなので,lifting of $\theta$は一つしか無い.
    よって$\Hom_A(\modDer_{B/A}, I)=0$.
    任意の$R, I$についてこれが成立するので,これは$\modDer_{B/A}=0$と同値.

    formally etale $\implies$ formally unramifiedなので(ii)$\implies$(iii)は明らか.
\end{proof}

\begin{Prop}[\cite{ASS} Prop1.3.6 (iii)]
    exact seq
\end{Prop}

\begin{Prop}
    $f \colon X \to Y$を,
    locally of finite presentationとする.
    $f$ :: smoothと次の条件は同値である:
    
    任意の点$x \in X$について,
    $x$と$y=f(x) \in Y$の間にaffine neighborhood
    \[
        x \in \Spec A \subset X, \qquad y=f(x) \in \Spec B \subseteq Y
        \qquad (\text{with  }f(\Spec B) \subseteq \Spec A)
    \]
    が存在し,ある$n, s$と$f_1, \dots, f_s, g \in A[x_1, \dots, x_n]$について
    \[ B \iso \left( \frac{A[x_1, \dots, x_n]}{(f_1, \dots, f_s)} \right)[1/g]. \]
    さらに,Jacobian matrix ($n \times (n-r)$-matrix)
    \[ \mat{ \frac{\partial f_i}{\partial x_j} }_{i, j} \]
    の部分$(n-r)$正方行列は,いずれも可逆(行列式が$B$のunit element).

    さらに,$f$ :: etaleと,この条件で$n=r$であることは同値である.
\end{Prop}

\begin{Prop}[\cite{StacksProj}, Tag 02G7] \label{prop:unram_qfinite}
    $f \colon X \to Y$がunramified morphismならば,
    任意の$y \in Y$について,fiber of $f$ :: $X_y$は
    disjoint union of spectra of finite separable field extensions of $k(y)$.
\end{Prop}

\begin{Prop}[\cite{StacksProj}, Tag 04HM]
    $f \colon X \to Y$をseparated etale morphismとする.
    $y \in Y$に対し$f^{-1}(s)=\{x_1,\dots,x_n \}$とする
    (点が有限個であることは命題(\ref{prop:unram_qfinite})による).
    étale neighbourhood :: $\nu: (U, u) \to (Y, y)$が存在し,
    $X_U=X \times_{Y} U$のdisjoint union decomposition
    \[ X_U=\bigsqcup V_{i, j} \]
    について$V_{i, j} \iso U$.
\end{Prop}

\section{命題に対する例}

\section{演習問題}

\bibliographystyle{jplain}
\bibliography{reference}
\end{document}
