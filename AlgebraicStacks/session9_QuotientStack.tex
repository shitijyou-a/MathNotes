\documentclass[a4paper, dvipdfmx]{jsarticle}
\usepackage{macros}

\newcommand{\Diag}{\Delta}
\newcommand{\rest}{\vspace{5pt}}
\newcommand{\arpb}{\ar[lu, phantom, "\text{p.b.}"]}
\newcommand{\rep}{{\color{blue}\#}}

\begin{document}
\title{ゼミノート \#9 \\ Quotient Stacks}
\author{七条彰紀}
\maketitle
\tableofcontents
\vspace{10pt}

Algebraic stackの具体例としてQuotient stackを扱う.
この例を通じて特に,
「diagonal morphism $\Diag \colon \fibX \to \fibX \times_S \fibX$が表現可能とはどういうことか」
ということを考えたい.
参考文献として\cite{ChAlg} 1.3.2, \cite{IrrOfMg} Example 4.8, \cite{ASS}

\begin{Remark}
    以下,scheme $S$を固定し,big etale site :: $\ET(S)$上のstack in groupoidsのみ考える.
\end{Remark}

\section{Definition}
\begin{Def}[Group Scheme]
\end{Def}

\begin{Def}[Principal $G$-Bundle ($G$-torsor)]
\end{Def}

\begin{Def}[Quotient Stack]
\end{Def}

\section{Aim of This Session}
\begin{Thm}
    Quotient Stackはalgebraic stackである.
\end{Thm}

\section{準備}
\subsection{\tp{$\Diag$}{Diagram Morphism}の表現可能性のために.}

\section{証明}

\end{document}
