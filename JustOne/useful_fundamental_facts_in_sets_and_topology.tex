\documentclass[a4paper]{jarticle}
\usepackage{../math_note}

\title{集合論と位相空間論における有用で基礎的な事実のリスト}
\author{七条 彰紀}
\begin{document}
\maketitle

基礎的な集合論・位相空間論を一通り学び終えてより発展的な命題を扱っていても,
時々「アレが使えると思うんだけど,正確な命題を忘れてしまったな」
ということがある.
以下ではそういった事柄を列挙する.
いずれも,証明は集合論・位相空間論を一通り学び終えていれば簡単なものである.

\begin{screen}
    以下,$X, Y, \dots$等の大文字は集合あるいは位相空間とし,
    $f,g,\dots$等小文字はそれらの間の写像とする.
\end{screen}

\section{集合論}
    \subsection{射}
    写像$f \colon X \to Y, g \colon Y \to Z$を考える.
    \begin{enumerate}
        \item 合成$g \circ f$が全射ならば$g$も全射.
        \item 合成$g \circ f$が単射ならば$f$も単射.
    \end{enumerate}

    また,部分集合に関しては次も成立する.
    \begin{enumerate}
        \item $f$が単射ならば,$X$の任意の部分集合$S$について$f^{-1}(f(S))=S$.
        \item $f$が全射ならば,$Y$の任意の部分集合$T$について$f(f^{-1}(T))=T$.
    \end{enumerate}

\section{位相空間論}
    \subsection{射}
    写像$f \colon X \to Y, g \colon Y \to Z$を考える.
    \begin{enumerate}
        \item 写像$f$が連続であることと$f(\cl_X(A)) \subseteq \cl_Y(f(A))$が成立することは同値.
    \end{enumerate}

    \subsection{閉包$\cl$}
    位相空間の部分集合$A, B$と
    部分集合の族$\{S_{\lambda}\}_{\lambda \in \Lambda}$について次が成立する.
    \begin{gather}
        \cl(A \cap B) \subseteq \cl(A) \cap \cl(B), \qquad
        \cl \left(\bigcap S_{\lambda}\right) \subseteq \bigcap \cl \left( S_{\lambda} \right) \\
        \cl(A \cup B) = \cl(A) \cup \cl(B), \qquad
        \cl \left( \bigcup S_{\lambda} \right) \supseteq \bigcup \cl \left( S_{\lambda} \right)
    \end{gather}

\end{document}
