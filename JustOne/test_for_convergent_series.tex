\documentclass[a4paper]{jsarticle}
\usepackage{../math_note}

\title{冪級数の収束判定法}
\author{七条 彰紀}

\begin{document}
    \maketitle

    $B(r;z_0)=\{ z \in \C : |z-z_0|<r \} \subset \C$とする.
    これは$z_0$を中心とする半径$r$の$\C$における円盤である.
    冪級数についていくつかの収束判定法を紹介する.

    \section{Weierstrass M-test}
    収束判定法では次の事実が基本である.
    \begin{Thm}
        関数列$\{f_n: A \to \C\}_{n=0}^{\infty}$の級数$\sum_{n=0}^{\infty} f_n(x)$を考える.
        ある実数列$\{M_n\}_{n=0}^{\infty} \subset \R$について
        \[ \Forall{n=1,2,\dots} \sup_{x \in A} |f_n| \leq M_n \]
        が成り立ち,かつ$\sum_{n=0}^{\infty} M_n$が収束するとき,
        関数項級数$\sum_{n=0}^{\infty} f_n(x)$も収束する.
    \end{Thm}
    この定理を満たす関数項級数$\sum_{n=0}^{\infty} f_n(x)$は,
    特に正規収束すると言う.

    \section{Abel theorem}
        \begin{Thm}[Abel theorem]
            $\{a_n\}_{n=0}^{\infty} \subset \C$とする.
            \[
                z_0 \in \C,~ \sum_{n=0}^{\infty}{a_n z^n_0} < \infty
                \implies
                \Forall{z \in B(|z_0|; 0)} \sum_{n=0}^{\infty}{a_n z^n} < \infty
            \]
            特に,この時$\sum_{n=0}^{\infty}{a_n z^n}$は正規収束する.
        \end{Thm}
        \begin{proof}
            命題$\sum_{n=0}^{\infty}{a_n z^n_0} < \infty \implies \lim_{n \to \infty}{a_n z^n_0}=0$と,
            $|\frac{z}{z_0}|<1$を用いて,Weierstrass M-testへ帰着する.
        \end{proof}

    \section{Cauchy-Hadamard theorem}
        \begin{Thm}[Cauchy-Hadamard theorem]
        一複素変数$z$に関する,以下のような冪級数を考える.
        \[ f(z) = \sum_{n = 0}^{\infty} c_{n} (z-a)^{n}. \]
        ここで $a, c_n \in \C$ とする.このとき,$f$の収束半径は以下のように与えられる.

        \[
            \frac{1}{R}=\limsup_{n \to \infty} \big( | c_{n} |^\frac{1}{n} \big)
            :=\lim_{n\to\infty}\sup\{| c_{n} |^\frac{1}{n} : k\geq n\}.\]
        \end{Thm}

        \begin{proof}
            $\limsup_{n \to \infty} \big( | c_{n} |^\frac{1}{n} \big)=0$
            のとき冪級数の収束半径が$\infty$であることだけ示す.

            仮定より,任意の正数$\epsilon$に対し,
            十分大きい$n$について$0 \leq | c_{n} |^\frac{1}{n} < \epsilon$が成立する.
            そこで任意の$z \in \C$をとり,$\epsilon=\frac{1}{2|z|}$とする.
            すると,
            \[
                0 \leq | c_{n} |^\frac{1}{n} < \epsilon
                \implies
                0 \leq | c_{n} |^\frac{1}{n}|z| < 1/2
                \implies
                0 \leq | c_{n} ||z|^{n} < (1/2)^n
            \]
            よってWeierstrass M-testにより,任意の$z \in \C$について冪級数は収束する.
        \end{proof}

    \section{Ratio test}
        \begin{Thm}[Ratio test]
        べき級数$\sum_{n=0}^\infty{a_n}z^n$について,
        任意の$n$について$a_n \neq 0$,かつ
        \[
            \rho = \lim_{n \to \infty} \left| \frac{a_{n+1}}{a_n} \right|
        \]
        が存在するならば,$R=\frac{1}{\rho}$がべき級数の収束半径に等しい.
        \end{Thm}

        \begin{proof}
            今,考えている冪級数を二つに分けて,
            \[
                \sum_{n=0}^\infty{a_n z^n}=\sum_{n=0}^{N-1}{a_n z^n}+\sum_{n=N}^{\infty}{a_n z^n}
            \]
            としてみると,前半は有限級数であるから有限値に収束する.
            なので後半の収束だけを考える.
            $z=0$での収束は自明なので,以下では$z \neq 0$とする

            $0 \leq \rho < \infty$とする.極限の定義から,以下が成立.
            \[
                \Forall{\epsilon>0} \Exists{N \in \N} \Forall{n \in N}
                n>N \implies 0 \leq \left| \frac{a_{n+1}}{a_n} \right| < \rho + \epsilon
            \]
            ここで,任意の$n \in \mathbb{N}$について,
            \[
                |a_n|
                = \overbrace
                {
                    \left|\frac{a_{n}}{a_{n-1}}\right| \left|\frac{a_{n-1}}{a_{n-2}}\right| \cdots \left|\frac{a_{N}}{a_{N-1}}\right|
                }^{ (n-N+1)個 } |a_{N-1}|
            \]
            となる,したがって次のように命題がつながる.
            \begin{eqnarray*}
                \Forall{\epsilon>0} \Exists{N \in \N} \\
                \Forall{n \in \N} n>N &\implies& 0 \leq \left| \frac{a_{n+1}}{a_{n}} \right| < \rho + \epsilon \\
                \iff \Forall{n \in \N} n>N &\implies&
                    0 \leq
                        \left| \frac{a_{n+1}}{a_{n}} \right|,
                        \left| \frac{a_{n}}{a_{n-1}} \right|,
                        \dots,
                        \left| \frac{a_{N}}{a_{N-1}} \right|
                    < \rho + \epsilon \\
                &\implies&  0 \leq |a_n|< (\rho+\epsilon)^{n-N+1}|a_N| \\
                &\iff&      0 \leq |a_n||z^n|< (\rho+\epsilon)^{n-N+1}|a_N| |z^n| \\
                &\iff&      0 \leq |a_n||z^n|< \frac{|z|^n}{\frac{1}{(\rho+\epsilon)^{n-N+1}}}|a_N| \\
                &\iff&      0 \leq |a_n||z^n|< \left( \frac{|z|}{\frac{1}{\rho+\epsilon}} \right)^{n} (\rho+\epsilon)^{-N+1} |a_N| \\
            \end{eqnarray*}

            $0 < \rho<\infty$の場合を考える.
            まず,$|z|<1/\rho$となる任意の各$z$に対して適切に$\epsilon$をとれば
            \footnote{ $|z| \neq 0$ならば$0< \epsilon < \frac{1}{|z|} - \rho$の範囲にある$\epsilon$を,
            例えば$\frac{1}{2} \left(\frac{1}{|z|} - \rho \right)$をとる. },
            $\left( \frac{|z|}{\frac{1}{\rho+\epsilon}} \right)<1$となる.
            したがって,
            \begin{equation*}
                \sum_{n=N}^{\infty}{ \left( \frac{|z|}{\frac{1}{\rho+\epsilon}} \right)^{n} (\rho+\epsilon)^{-N+1} |a_N|} \\
                =(\rho+\epsilon)^{-N+1} |a_N| \sum_{n=N}^{\infty}{ \left( \frac{|z|}{\frac{1}{\rho+\epsilon}} \right)^{n}}
            \end{equation*}
            となる.$N$は各$\epsilon$に対して定まる有限値だから,これは収束する.
            Weierstrass M-testにより,冪級数の収束半径は$1/\rho$.

            $\rho=0$の場合を考える.
            このときは0でない任意の$z$について,同様に
            $ 0< \epsilon < \frac{1}{|z|}$の範囲の$\epsilon$をとれば
            $\left( \frac{|z|}{\frac{1}{\epsilon}} \right)<1$とすることができて,
            Weierstrass M-testにより収束半径は$\infty$となる.

            最後に,$\rho=\infty$の場合.
            ここまでほとんどこのケースの考察はしていないことに注意しておく.
            しかし同様の議論をすると$\lim_{n \to \infty}{|a_n|}=\infty$がわかる.
            したがって命題$\sum_{n=0}^{\infty}{a_n z^n} < \infty \implies \lim_{n \to \infty}{a_n z^n}=0 \implies \lim_{n \to \infty}{|a_n z^n|}=0$
            の対偶より,冪級数は発散する.
        \end{proof}

        これは$\sin z=z-\frac{z^3}{3!}+\frac{z^5}{5!}+\cdots$のように,
        十分先の項でも係数が0になることがあるとそのままでは使えない.
        しかし$\sin z$や$\cos z$には以下のように変形するとこの方法が使える.
        \begin{eqnarray*}
            \sin z \\
            &=& 1 \cdot \sum_{n=0}^{\infty}{\frac{(-1)^{n}}{(2n+1)!}z^{2n+1}} \\
            &=& (z \cdot 1/z) \cdot \sum_{n=0}^{\infty}{\frac{(-1)^{n}}{(2n+1)!}z^{2n+1}} \\
            &=& z \cdot \left(1/z \cdot \sum_{n=0}^{\infty}{\frac{(-1)^{n}}{(2n+1)!}z^{2n+1}} \right) \\
            &=& z \cdot \sum_{n=0}^{\infty}{\frac{(-1)^{n}}{(2n+1)!}z^{2n}} \\
%            &=& w^{\frac{1}{2}} \sum_{n=0}^{\infty}{\frac{(-1)^{n}}{(2n+1)!}w^n} ~~ (w := z^2)\\
        \end{eqnarray*}
        そこで,$S(z)=\sum_{n=0}^{\infty}{\frac{(-1)^{n}}{(2n+1)!}z^n}$の収束半径を考える.
        以下の計算から,これの収束半径は$\infty$と分かる.
        \[
            \lim_{n \to \infty} \left| \frac{a_{n+1}}{a_n} \right|
            = \lim_{n \to \infty} \frac{1}{2n(2n+1)}=0
        \]
        したがって$\sin z=zS(z^2)$の収束半径は$(\infty)^{1/2}=\infty$であることが示される.

\end{document}
