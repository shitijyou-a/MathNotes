\RequirePackage[]{silence}
\WarningFilter{latexfont}{Font shape}
\WarningFilter{latexfont}{Some font shapes were not available}
\documentclass[lualatex, ja=standard, a4paper]{bxjsarticle}

\usepackage{../math_note}
\usepackage[backend=biber, style=alphabetic]{biblatex}
\addbibresource{reference.bib}

\newcommand{\introword}[2]{{\bfseries #1} (#2) }
\newcommand{\ftor}[1]{\underline{#1}}

\newcommand{\ZAR}   {\opcat{\Ring}_{\mathrm{ZAR}}}
\newcommand{\ET}    {\opcat{\Ring}_{\mathrm{ET}}}
\newcommand{\SM}    {\opcat{\Ring}_{\mathrm{SM}}}
\newcommand{\FPPF}  {\opcat{\Ring}_{\mathrm{FPPF}}}
\newcommand{\FPQC}  {\opcat{\Ring}_{\mathrm{FPQC}}}

\newcommand{\stX}{\mathcal{X}}
\newcommand{\stY}{\mathcal{Y}}
\newcommand{\stZ}{\mathcal{Z}}

\newcommand{\Aff}{\cat{Aff}}

%% \allowbreak {{{
\makeatletter
\def\old@comma{,}
\catcode`\,=13
\def,{%
  \ifmmode%
    \old@comma\discretionary{}{}{}%
  \else%
    \old@comma%
  \fi%
}
\makeatother
%% }}}

\begin{document}
\title{スキーム・代数的空間・代数的スタックの別定義}
\author{七条 彰紀}
\maketitle

\begin{abstract}
    可換環の圏から出発して,
    スキーム,代数的空間,代数的スタックをチャート付き層あるいはチャート付きスタックとして定義する.
    「チャート付き対象」は多様体やスキームと言った対象の定義の様式を一般化したものである.
    それぞれチャート付き対象の様式での定義を述べた後,
    それらが通常使われる定義と同値であることを述べる.
    読者はすでにスキーム,代数的空間,代数的スタックの定義を熟知しているものとする.
\end{abstract}

\section{思想:チャート付き幾何的対象}
多様体は通常,以下のように定義される.

\begin{Def}[\introword{多様体}{manifold}の開被覆]
    位相空間$X$を考える.
    \begin{enumerate}
        \item
            位相空間の射$f \colon X \to Y$が\introword{開埋め込み}{open immersion}であるとは,
            \begin{itemize}
                \item $f(X)$が位相空間$Y$の開部分集合であり,
                \item かつ$f$から誘導される射$X \to f(X)$が同相射である,
            \end{itemize}
            ということ.

        \item
            位相空間の射の族$\{ u_i \colon U_i \to X \}_{i \in I}$が
            開被覆であるとは,
            任意の点$x \in X$に対して$x \in u_i(U_i)$なる$i \in I$が存在するということ.
    \end{enumerate}
\end{Def}

\begin{Def}[ユークリッド空間によるチャート付き位相空間としての実多様体]
    位相空間$X$が$m$次元実多様体であるとは,
    ユークリッド空間$\R^m$の開部分集合からの射からなる
    開被覆$\{ U_i \to X \}_{i \in I}$が存在するということ.
\end{Def}

一方,スキームはアフィンスキームによる開被覆を持つ局所環付き空間として定義されるのであった.

\begin{Def}[\introword{局所環付き空間}{locally ringed space}の開被覆]
    局所環付き空間$(X, \shO_{X})$を考える.
    \begin{enumerate}
        \item
            局所環付き空間の射$(f, f^{\#}) \colon (X, \shO_{X}) \to (Y, \shO_{Y})$が
            \introword{開埋め込み}{open immersion}であるとは,
            \begin{itemize}
                \item $f(X)$が位相空間$Y$の開部分集合であり,
                \item かつ$f$から誘導される射$X \to f(X)$が同相射であり,
                \item かつ層の射$f^{\#}|_{f(X)} \colon \shO_{Y}|_{f(X)}=\shO_{f(X)} \to \shO_{X}$が同型射である,
            \end{itemize}
            ということ.

        \item
            局所環付き空間の射の族$\{ (u_i, u_i^{\#}) \colon (U_i, \shO_{U_i}) \to (X, \shO_{X}) \}_{i \in I}$が
            開被覆であるとは,
            任意の点$x \in X$に対して$x \in u_i(U_i)$なる$i \in I$が存在するということ.
    \end{enumerate}
\end{Def}

\begin{Def}[アフィンスキームによるチャート付き局所環付き空間としてのスキーム]
    局所環付き空間$(X, \shO_{X})$がスキームであるとは,
    アフィンスキームからの射からなる開被覆$\{ \Spec R_i \to (X, \shO_{X}) \}_{i \in I}$が
    存在するということ.
\end{Def}

ここに共通して見られるのは,多様体とスキームはどちらも
「既によく知られている幾何的対象で被覆できる種類の幾何的対象」である,ということである.
このような幾何的対象は\cite{Lin16}で圏論的に取り扱われていて,
\introword{チャート付き対象}{charted object}と呼ばれている.

多様体は特別な位相空間,スキームは特別な局所環付き空間として定義されている.
一方,代数的空間と代数的スタックはそれぞれ特別な層,特別なスタックとして定義される.
スキームも特別な層として,可換環の圏から構成することが出来る.
この際に古典的な意味の位相空間は必要でない.
そして代数的空間と代数的スタックもチャート付き対象の形で定義することが出来る.

\section{環の景}

単位的可換環(以下,環)の圏を$\Ring$と書く.
もちろんこの圏には零環が属す.
非可換環は扱わない.

圏$\opcat{\Ring}$の射の性質として
開埋め込み,fppf 射,fpqc 射を定義する.
環と加群の定義,
モノ射,エピ射,
(忠実)平坦,(形式的に)滑らかな・不分岐・エタールな射,
有限表示射の定義は既知とする.

\begin{Def}[環の開埋め込み射]\label{def:open_imm_ring}
    環の射(準同型)$\phi \colon R' \to R$を考える.
    環の射$\phi \colon R' \to R$が平坦,モノ,有限表示であるとき
    $\phi$は開埋め込みであるという.
\end{Def}
通常の意味の開埋め込みとこの定義の関係は \cite{SP} 025G を参照せよ.

\begin{Def}[幾何的被覆]\label{def:geo_cov}
    $\Ring$の射の族$\{\phi_i \colon R \to S_i\}_{i \in I}$が環$R$の幾何的被覆であるとは,
    任意の体$k$と任意の射$S_i \to k$に対して,
    添字$i \in I$,体$k$と射$R \to k', k' \to k$が存在し,
    これらが以下のように可換図式を成すということ.
    \[
    \begin{tikzcd}
        R \ar[r, "\exists"]\ar[d, "\phi_i"']& k' \ar[d, "\exists"]\\
        S_i \ar[r, "\forall"']& k
    \end{tikzcd}
    \]

    一つの射からなる族$\{f \colon S \to R\}$が幾何的被覆であるとき,
    射$f$は幾何的全射であるという.
\end{Def}

\begin{Remark}
    この定義に有る図式は双対圏$\opcat{\Ring}$で描いたほうが分かりやすいかも知れない.
    \[
    \begin{tikzcd}
        k' \ar[r, "\exists"]\ar[d, "\exists"']& S_i \ar[d, "\phi"]\\
        k \ar[r, "\forall"']& R
    \end{tikzcd}
    \]
    この図式では$\Spec$が省略されていると考えれば,
    射$k \to R$は$R$の$k$有理点だと解釈できる.
\end{Remark}

\begin{Lemma}\label{lemma:geo_cov_is_stable_under_base_change}
    環$R$の幾何的被覆$\{ R \to S_i \}$と環の射$R \to R'$を任意にとる.
    これらから得られる射の族$\{ R' \to R' \otimes_{R} S_i \}$は
    $R'$の幾何的全射である.
\end{Lemma}
\begin{proof}
    テンソル積の普遍性を用いれば,圏論的な議論だけで証明できる.
\end{proof}

\begin{Def}[環のZariski / 平滑 / エタール / fppf 景]
    $\opcat{\Ring}$に次のように Grothendieck 位相を定義する.

    記号$\mu$を表(\ref{table:top_tau_and_mu})にあるいずれかの組とする.
    圏$\opcat{\Ring}$の対象$R$に対して,
    $\mu$である$\opcat{\Ring}$の射の集合$\{S_i \to R\}_i$であって
    合併的に全射であるものを全体のクラスを$\Cov(R)$とする.

    以上で定まる景の名前と記号は表(\ref{table:top_tau_and_mu})のとおりとする.
    $\Cov(R)$の元はこの景における$R$の被覆と呼ばれる.
\end{Def}

\begin{table}[htb]
\centering
\caption{環の景の名前,記号,対象の種類,被覆の種類}
\label{table:top_tau_and_mu}
\begin{tabular}{@{}llll@{}}
    \toprule
    名前 & 記号 & $\mu$ \\ \midrule
    Zariski 大景 & $\ZAR$ & 開埋め込み射 \\
    平滑 大景 & $\SM$ & 平滑 (smooth) 射 \\
    エタール大景 & $\ET$ & エタール射 \\
    fppf 大景 & $\FPPF$ & 平坦かつ局所有限表示な射 \\
    fpqc 大景 & $\FPQC$ & 平坦射 \\ \bottomrule
\end{tabular}
\end{table}

\begin{Remark}
    平坦な環の射$\phi \colon S \to R$について次が同値であることに注意.
    \begin{itemize}
        \item $\phi$は忠実平坦である.
        \item $\phi$から誘導される射$\Spec R \to \Spec S$は全射である.
    \end{itemize}
    したがって環$R \in \Ring$の$\FPQC$における被覆$\{S_i \to R\}_{i \in I}$について,
    ここから誘導される射$\prod_{i \in I} S_i \to R$は忠実平坦である.
    上で定義した景の被覆はいずれも平坦な射から成るので,
    いずれの景でも同様にして忠実平坦射が得られる.
\end{Remark}

\begin{Def}
    $\Ring$上の前層の圏を$\PShv(\opcat{\Ring})=\Set^{\opcat{\Ring}}$と書く.
    景$\mathcal{S}$上の層の圏を$\Shv(\mathcal{S})$と書く.
\end{Def}

\begin{Lemma}
    記号$\tau$を Zariski, ET, SM, FPPF, FPQC のいずれかとする.
    $\PShv(\opcat{\Ring}_{\tau})$と$\Shv(\opcat{\Ring}_{\tau})$は
    完備 (complete) かつ余完備 (cocomplete) である.
\end{Lemma}
\begin{proof}
    完備性は引き戻し (pullback) と終対象 (terminal object) の存在と同値であり,
    余完備性は押し出し (pushout) と始対象 (initinal object) の存在と同値である.
    %% TODO
\end{proof}

\section{代数幾何的な空間的対象の構成方法}
代数的空間や代数的スタックの文脈における「表現可能な射」や「表現可能な射による被覆」を一般化する.

次のような二つの圏$\cat{C}, \cat{S}$を考える.
\[ \opcat{\Ring} \subseteq \cat{C} \subseteq \cat{S} \]   
この包含関係は対象と射について単射的な関手によって与えるものとする.
例えば米田関手によって$\opcat{\Ring} \subseteq \PShv(\FPPF)$などが考えられる.

圏$\cat{S}$の射が成す任意の cospan (訳語不明) $x \rightarrow z \leftarrow y$について,
ファイバー積 $x \times_{z} y$が存在するものとする.

\begin{Def}[$\cat{C}$で表現可能; $\cat{C}$-representable]
    圏$\cat{S}$の対象が$\cat{C}$表現可能とは,
    その対象が$\cat{C}$(の適当な単射的関手による像)に属しているということ.

    圏$\cat{S}$の射$x \to y$が$\cat{C}$表現可能であるとは,
    $\cat{C}$の対象から$y$への任意の射$c \to y (c \in \mathrm{Ob}\cat{C})$について
    ファイバー積$x \times_{y} c$が$\cat{C}$表現可能である,
    ということ.
\end{Def}

\begin{Example}
    次の場合に「$\cat{C}$で表現可能」の定義を書き下してみる.
    \[ \cat{C}=\opcat{\Ring} \xrightarrow{\text{米田関手}} \PShv(\FPPF)=\cat{S} \]
    $\PShv(\FPPF)$の射$F \to G$が$\opcat{\Ring}$表現可能であるとは,
    任意の環からの射$\Hom_{\Ring}(-,A)=\ftor{A} \to G$について,
    ファイバー積$F \times_{G} \ftor{A}$が環で表現可能である
    (ある環$R$から得られる前層$\ftor{R}$と同型である)ということ.
\end{Example}

\begin{Def}[$\cat{S}$に於ける幾何的被覆]
    圏$\cat{S}$の射の族$\{ f_i \colon x_i \to x \}_{i \in I}$が
    $x \in \mathrm{Ob} \cat{S}$の幾何学的被覆であるとは,
    体$k \in \opcat{\Ring} \subset \cat{S}$から$x$への任意の射$k \to x$について,
    次の可換図式を成す添字$i \in I$,体$k'$と射$k' \to x_i, k' \to k$が存在する.
    \[
    \begin{tikzcd}
        k' \ar[r, "\exists"]\ar[d, "\exists"']& x_i \ar[d, "f_i"]\\
        k \ar[r, "\forall"']& x
    \end{tikzcd}
    \]
\end{Def}

\begin{Def}[$\cat{C}, E$-representable cover]
    $E$を圏$\cat{C}$の射の部分クラスとする.
    対象$x \in \mathrm{Ob}\cat{S}$の$\cat{C}, E$-representable coverとは,
    次の条件を満たす$\cat{S}$の射の族$\{ f_i \colon x_i \to x \}_{i \in I}$である.
    \begin{itemize}
        \item $\{ f_i \colon x_i \to x \}_{i \in I}$は幾何的被覆である.
        \item 任意の$i \in I$について$x_i$は$\cat{C}$の対象である.
        \item 任意の$i \in I$について$\cat{C}$で表現可能である.
        \item
            任意の$i \in I$と任意の$\cat{C}$の対象からの射$c \to x$について,
            引き戻しで得られる$\cat{C}$の射$x_i \times_{x} c \to c$は$E$に属す.
    \end{itemize}
\end{Def}

\begin{Def}
    $\opcat{\Ring} \subseteq \cat{C} \subseteq \cat{S}$と$\cat{C}$の射のクラス$E$について,
    $\cat{C}, E$-representable coverを持つ$\cat{S}$の対象を
    $\cat{S}$の$\cat{C}, E$チャート付き代数的空間対象 ($\cat{C}, E$-charted algebraically space object) という.
    $\cat{C}, E$チャート付き代数的空間対象の射は$\cat{S}$の対象としての射である.
\end{Def}

本当は「チャート付き代数的空間」と名付けたいが,これは特殊な代数的空間と紛らわしいため,
以上のように名付けた.

以上の定義を$\Ring$を使わずに行う場合,
問題と成るのは幾何的被覆 %%TODO

\begin{Def}[チャート付き代数的空間対象(の射)の代数的性質]
\end{Def}

\section{スキーム,代数的空間,代数的スタック}
    以下,fppf 大景$\FPPF$上の層を考える.

\subsection{アフィンスキーム}
アフィンスキームの圏は$\opcat{\Ring}$から米田関手を用いて構成される.

\begin{Def}
    $\Ring$上の前層の圏を$\PShv(\opcat{\Ring})=\Set^{\opcat{\Ring}}$と書く.
    景$\mathcal{S}$上の層の圏を$\Shv(\mathcal{S})$と書く.
\end{Def}

\begin{Def}[表現可能関手]
    環$R \in \opcat{\Ring}$について,関手$\ftor{R}$を次のように定義する.
    \begin{defmap}
        \ftor{R}\colon & \opcat{\Ring}& \to& \Set \\
        \textbf{\underline{対象}:}& S& \mapsto& \Hom_{\opcat{\Ring}}(S,R) \\
        \textbf{\underline{射}:}& \psi& \mapsto& (\circ \psi)
    \end{defmap}
    この関手$\ftor{R}$を環$R$で表現される関手という.
\end{Def}

\begin{Lemma}
    任意の環$A \in \opcat{\Ring}$について,
    関手$\ftor{A} \colon \opcat{\Ring} \to \Set$は
    景$\ZAR, \ET, \SM, \FPPF$上の層である.
\end{Lemma}
\begin{proof}
    \cite{SP} 023Pを参照せよ.
\end{proof}

\begin{Def}
    環で表現可能な fppf 大景$\FPPF$上の層をアフィンスキームと呼ぶ.
    アフィンスキームの射は層としての射とする.
    アフィンスキームの圏を$\Aff (\subset \Shv(\FPPF))$と書く.
\end{Def}

\subsection{スキーム}
    スキームは,アフィンスキームからの表現可能な開埋め込み射による幾何的被覆を持つ層である.

\begin{Def}[チャート付き層としてのスキーム]
    $\opcat{\Ring} \subseteq \Aff \subseteq \Shv(\FPPF)$を考える.
    ただし最初の包含関係は米田関手によって与えられる.
    $\mathbf{OpImm}$をアフィンスキームの開埋め込み射が成すクラスとする.
    層の圏$\Shv(\FPPF)$の$\Aff, \mathbf{OpImm}$チャート付き代数的空間対象をスキームという.
    スキームの圏を$\cat{Sch}$と書く.
\end{Def}

\begin{Prop}
    スキームの景は$\ZAR$と同型
\end{Prop}

\subsection{代数的空間}
\begin{Def}[チャート付き層としての代数的空間]
    $\opcat{\Ring} \subseteq \cat{Sch} \subseteq \Shv(\FPPF)$を考える.
    ただし最初の包含関係は米田関手によって与えられる.
    $\mathbf{Et}$をスキームのエタール射が成すクラスとする.
\end{Def}

\begin{Prop}
    代数的空間のdiagonal mapは表現可能
\end{Prop}

\subsection{代数的スタック}
\begin{Def}[代数的空間で表現可能なスタック]
\end{Def}

\begin{Def}[代数的空間で表現可能な射]
\end{Def}

\begin{Def}[代数的空間で表現可能な被覆]
\end{Def}

\begin{Def}[チャート付きスタックとしての代数的スタック]
    $\FPPF$上の亜群のスタック$\stX$が代数的スタックであるとは,
    代数的空間からの表現可能な射がなす被覆
    $\{ \phi_i \colon \ftor{S}_i \to \shX \}_{i \in I}$であって,
    全ての$i \in I$について$\phi_i$が\underline{エタール射}であるものが存在する,
    ということ.
%art. st. はschemeによるamooth coverを持つ$\FPPF(R)$上のスタック
\end{Def}

\begin{Prop}
    代数的スタックのdiagonal mapは代数的空間で表現可能.
\end{Prop}

\section{考えられる変種}
% stack charted by rep. et. mor. from scheme

\printbibliography[title=参考文献]
\end{document}
