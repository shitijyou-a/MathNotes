\documentclass[]{jsarticle}
\usepackage{../math_note, enumitem}

\newcommand{\Open}{\mathbf{Open}}
\newcommand{\Sets}{\mathbf{Sets}}
\newcommand{\covU}{\mathcal{U}}

\begin{document}
    \title{層(sheaf)の概観 \\ {\normalsize 代数幾何学に於いて}}
    \author{七条彰紀}
    \maketitle

    代数幾何学を始める人のために,またこれから学ぶ人のために書きます.ノートです.当然ですが.

    \section{代数幾何学に於けるsheafという概念}
    まず代数幾何学でのsheafの立ち位置について書いておきます.
    sheaf,より一般にstackは代数幾何学の中心的概念です.

    \paragraph{scheme theoryに於いて}
    scheme theory(スキーム論)では可換環からschemeという幾何学的な対象を構築します.
    この構築の際に,元と成っていた可換環を記憶しているのがsheafです.
    可換環が「切り刻まれ貼り合わされて」structure sheaf(構造層)となっているのです.
    他にscheme theoryで現れるsheafとしては,scheme上の(quasi-)coherent sheaf((準)連接層)が重要です.
    (quasi-)coherent sheafを調べることは,schemeについての情報を得るための基本的な手段です.
    またscheme上のsheaf cohomology(層係数コホモロジー)は,
    schemeに「変な感じの部分」がどれだけあるかを調べるための重要な道具となっています.
    (特にetale cohomologyは数論においても重要な位置を占めています.)

    \paragraph{scheme theoryをはみ出す}
    一方で,schemeだけでは用に足りないことがあります.
    例えばschemeの群による商を考えることがあります
    (これは普通の位相空間の群作用による商のようなものです).
    定義は圏論的に,普遍性を用いて定義されるのですが,
    条件を満たすschemeが無い,ということはしょっちゅうです.
    これに対する一つの解決方法として,
    schemeの概念を拡張するということが考えられます.
    圏論的に性質の良い,都合の良い対象まで研究対象に収めようというわけです.

    \paragraph{圏論ちょっとわかる,という人向け}
    schemeの概念を拡張するには,どのような方策を取るべきでしょうか.
    当座の目標は「schemeの圏を包含する圏を探す」ということです.
    全ての極限をschemeの圏に付け加える(pro-scheme),
    基礎を担う可換環論を非可換環論やモノイド論まで拡大する,
    などの手段があります.
    ですがまた別に,米田の補題を手がかりにする事が出来ます.
    米田の補題は,
    米田関手がschemeの圏からscheme上のpresheaf(前層)の圏への忠実充満関手
    となることをいっています.
    schemeの圏を包含する圏として「schemeの圏からscheme上のpresheafの圏」
    が使える,ということです.

    \paragraph{schemeの一般化に於けるsheaf}
    この,schemeの一般化(generalized scheme; 一般化スキーム)を考える方向では,
    sheafが中心概念です.
    実際にgeneralized schemeの代表であるalgebraic space(代数的空間)はsheafです.
    そしてalgebraic spaceの定義に
    algebraic spaceの位相空間の定義は含まれていません.
    
    ちなみに極端なことを言うと,
    schemeでさえ最初から位相空間無しに定義することが可能です.
    これは"functorial scheme"(関手的スキーム?)などと呼ばれます.
    もちろんこれは普通の意味のschemeではありませんが,
    "functorial scheme"から普通のschemeの体裁を整えることも,
    この逆も可能です.

    \paragraph{さらなる一般化}
    そしてsheafはstack(いわば,圏を値に取るsheaf.スタック)へ,
    algebraic spaceはalgebraic stack (Artin/DM stack)へと一般化されます.
    恐ろしいことにalgebraic spaceにもalgebraic stackに関しても
    (quasi-)coherent sheafやsheaf cohomologyといった理論が構築されています.

\section{sheafの思想}
    \paragraph{sheafの定義}
    位相空間上のsheafの定義の仕方は少なくとも$4$つ存在しますが,
    意味が分かりやすいのは
    ``identity axiom"(一致公理?)と
    ``gluability axiom"(接着性公理 or 貼り合わせ可能性公理?)を
    満たすpreaheaf(前層)として定義することだと思います.
%    (特定の完全列を満たすpresheafとして定義するのは圏論的な取扱いに向いていますし,
%    etale mapのsectionが成すpresheafとして定義するのはsheafの一歩進んだ理解を促します.)
    この定義を以下に述べます.
    \begin{Def}
        \hfill \vspace{-0.8cm}
        \begin{enumerate}[label=(\roman*), leftmargin=*]
        \item 
            位相空間$X$に対し,$X$の開集合と包含写像が成す圏を$\Open(X)$とする.

        \item
            $\Open(X)$から集合の圏$\Sets$への反変関手$\shF \colon \Open(X) \to \Sets$を
            $X$上のpresheafと呼ぶ.

        \item
            $X$上のpresheaf :: $\shF$と$U \in \Open(X)$について,
            $s \in \shF(U)$を$\shF$の$U$上のsectionと呼ぶ.
        
        \item
            $X$の開集合の間に有る包含射$\iota^{U}_{V} \colon U \inclmap V \in \operatorname{Arr} \Open(X)$を考える.
            $X$上のpresheaf :: $\shF$による$\iota^{U}_{V}$の像を$\res^{U}_{V}(:=\shF(\iota^{U}_{V}))$
            と表し,restriction map(制限射)と呼ぶ.
            $s \in \shF(U)$の$\res^{U}_{V}$の像はしばしば$s|_{V}(=\res^{U}_{V}(s))$と表記される.

        \item 
            さらにpresheaf :: $\shF$が以下の$2$条件を満たす時,$\shF$は(集合の)sheafと呼ばれる.
            ただし$U \in \Open(X)$を$X$の開集合とし,
            $\covU=\{ U_i \}_{i \in I}$を$U$の$X$の開集合による被覆
            \footnote{ すなわち,任意の点$x \in U$に対して,$x \in U_i$を満たす$U_i \in \covU$が存在する. }とする.
            \begin{description}[labelindent=0.5cm, leftmargin=1cm]
                \item[\underline{Identity Axiom}] \mbox{}\\
                    任意の$2$元$s ,t \in \shF(U)$が,
                    任意の$i$について$s|_{U_i}=t|_{U_i}$であるならば,$s=t$.

                \item[\underline{Gluability Axiom}] \mbox{}\\
                    $U$の開部分集合上のsectionsの組
                    $( s_i )_{i \in I} \in \prod_{i \in I} \shF(U_i)$が次を満たすとする.
                    \[ \Forall{i, j \in I} s_i|_{U_i \cap U_j}=s_j|_{U_i \cap U_j}. \]
                    この時,全ての$i \in I$について$s|_{U_i}=s_i$であるsection :: $s \in \shF(U)$が存在する.
                    (identity axiomより,これは一意に存在する.)
            \end{description}
        \end{enumerate}
    \end{Def}

    \paragraph{「局所的に調べ,大域的に知る」}
    identity axiomとgluability axiomはそれぞれ
    「$2$つのsectionは,断片(すなわち$s|_{U_i}, t|_{U_i}$)が等しければ,等しい」
    「sectionの断片(すなわち$s_i$)は貼り合わせられる」と読めます.
    そのため,この二つの条件は
    「局所的な情報から大域的な情報が決定される」という気持ちを表現したものだ,
    と筆者は感じています.

    \paragraph{もう一つの定義}
    上記定義の$(v)$,
    すなわちidentity axiomとgluability axiomを次のように述べることも出来ます.
    \begin{Def}
        位相空間$X$上のpresheaf :: $\shF$を考える.
        $U \in \Open(X)$を$X$の開集合とし,
        $\covU=\{ U_i \}_{i \in I}$を$U$の$X$の開集合による被覆とする.
        この$U, \covU$に対し,集合$\shF(\covU)$を以下通り定義する.
        \[
            \shF(\covU)=
            \left\{
                (s_i)_{i \in I} \ \middle|\  \Forall{i, j \in I} s_i|_{U_i \cap U_j}=s_j|_{U_i \cap U_j}
            \right\}
        \]
        $(s_i)_{i \in I}$に課された条件はGluability Axiomで述べられているものと全く同じである.
        この集合の元をdescent datum(降下データム?)と呼ぶ.

        さて,次のように写像を定める.
        \begin{defmap}
            \epsilon_{\covU}\colon & \shF(U)& \to& \shF(\covU) \\
            {}& s& \mapsto& (s|_{U_i})_{i \in I}
        \end{defmap}
        
        $\shF$がsheafであるとは,
        \underline{任意の$U, \covU$について$\epsilon_{\covU}$が全単射であるということ.}
    \end{Def}
    任意の$U, \covU$について,
    $\epsilon_{\covU}$が単射であることは$\shF$についてidentity axiomを満たすことと同値です.
    同じく,全射であることはgluability axiomを満たすことと同値です.

\section{Seafification(層化)}
    任意のpresheaf :: $\shF$について,対応するsheaf :: $\shF^+$
    \footnote{ associated sheafという意味で$\shF^a$と書く人も居ます. }が一意に存在します.
    この対応$\shF \mapsto \shF^+$はsheafification(層化)は呼ばれています.
    

\end{document}
