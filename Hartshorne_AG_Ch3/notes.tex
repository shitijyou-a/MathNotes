\documentclass[]{jsarticle}
\usepackage[]{../math_note}
\usepackage[]{enumitem}
\usepackage[all]{xy}

\newcommand{\catMod}{\mathfrak{Mod}}
\newcommand{\catAb}{\mathfrak{Ab}}

\title{Hartshorne ``Algebraic Geometry" Ch.3 \\のためのノート.}
\author{七条 彰紀}

\begin{document}

\section{Derived Functors}
    %% none

\section{Cohomology of Sheaves}

\subsection{Lemma2.4.}
Lemma2.4(p.207)で次の補題が使われている.
\begin{Lemma}
    $(X, \shO_X)$ :: ringed space,
    $\shM$ :: $\shO_X$-module
    とする.
    開集合$U \subseteq X$について次が成立する.
    \[ \Hom_{\shO_X}(\shO_U, \shM) \iso \shM(U). \]
    しかもこの同型は$U, \shM$について自然.
\end{Lemma}
\begin{proof}
    次のように写像を定める.
    \begin{defmap}
        \Phi:& \Hom_{\shO_X}(\shO_U, \shM)& \to& \shM(U) \\
        {}& \phi& \mapsto& \phi_U(1) \\
        {}& [\sect{V}{t} \mapsto \sect{V}{t \cdot (s|_V)}]& \mapedfrom& s
    \end{defmap}
\end{proof}

\subsection{Prop2.5.}
    Prop2.5の証明で
    long exact sequenceを使っている部分が
    行間に成っているので説明する.
    Thm1.11cより次の完全列が得られる.
    \[
    \xymatrix@R=20pt
    {
        0 \ar[r] & \Gamma(X, \shF) \ar[r]& \Gamma(X, \shI)\ar[r]^-{\phi}& \Gamma(X, \shG) \\
        \ar[r]^-{\delta^0} & H^1(X, \shF) \ar[r]& 0 \ar[r]& H^1(X, \shG) \\
        \ar[r]^-{\delta^1} & H^2(X, \shF) \ar[r]& 0 \ar[r]& H^2(X, \shG) \\
        \ar[r]^-{\delta^2} & \dots
    }
    \]

    もともとの完全列から$\phi$ :: surj.
    また$\Gamma(X, \shG) \xrightarrow{\delta^0} H^1(X, \shF) \to 0$ :: exactから
    $\delta^0$ :: surjも得られる.
    完全列はcomplexでもあるから,$\delta^0 \circ \phi=0$.
    以上より次が分かる.
    \[ H^1(X, \shF)=\im \delta^0=\delta^0(\phi(\Gamma(X, \shI)))=\im (\delta^0 \circ \phi)=0. \]

    上のlong exact sequenceの一部を取り出す.
    ここでは$i \geq 1$としている.
    \[\xymatrix{ 0 \ar[r]& H^i(X, \shG) \ar[r]^-{\delta^i} & H^{i+1}(X, \shF) \ar[r]& 0 }\]
    これがexactであるから,
    $H^i(X, \shG) \iso H^{i+1}(X, \shF)$.
    この同型と$\shG$もflasqueであることを用いると,
    $H^i(X,-)$の添字$i$についての帰納法でProp2.5の主張が証明される.

\subsection{Prop2.6.}
    Prop2.6の意味は次の通り.
    $(X, \shO_X)$ :: ringed space,
    $\shF \in \catMod(X)$とする.
    $\shO_X$-moduleの構造を忘却すれば,$\shF \in \catAb(X)$とみなせる.
    (constant sheaf :: $\Z_X$は$\catMod(X)$のinitial objectだから,
    $\shF$は$\Z_X$-moduleとみなすことが出来る.
    そして$\Z_X$-moduleはsheaf of abelian groupに他ならない.
    Cor2.3の証明も参照せよ.)
    なので$\shF$のinjective resolutionを
    $\catMod(X), \catAb(X)$の中でそれぞれ取る.
    \[
    \xymatrix{ \shF \ar[r]& \shI^0 \ar[r]& \shI^1 \ar[r]& \dots }
    \]
    すなわち$\shI^i$として$\catMod(X)$のobjectまたは$\catAb(X)$のobjectをとる.
    するとどちらの圏でresolutionをとるかによって,
    異なる完全列が得られるだろう.
    しかし得られるcohomology groupは同じである,
    というのがProp2.6の主張である.

    証明は次の様に行う:
    $\catMod(X)$における$\shF$のinjective resolution
    \[
    \xymatrix{ \shF \ar[r]& \shI^0 \ar[r]& \shI^1 \ar[r]& \dots }
    \]
    は,
    $\shF, \shI^i$が持つ$\shO_X$-moduleの構造を忘却して$\catAb(X)$の完全列とみなせば,
    $\Gamma(X, -): \catAb(X) \to \catAb$についてのacyclic resolutionである.
    したがってThm1.2より,
    このresolutionから得られるcohomology groupは
    $H^i(X, \shF)$に等しい.

\subsection{Proof of Thm2.7.}
p.210, proof of 2.7についてコメント.
証明ではStep $1$と類似の議論が何度も繰り返される.
この議論を一般化して述べると次のように成る.
\begin{Lemma}
    $U \subseteq X$を開集合とすると$Y=X-U$は閉集合.
    \begin{enumerate}[label=(\roman*), leftmargin=*]
        \item 
            \[ H^i(X, \shF_Y)=0 \mand H^i(X, \shF_{U})=0 \text{ for } i>n \eqno{(*)} \]
            が成立するならば,
            $H^i(X, \shF)=$ for $i>n$も成り立つ.
        \item
            $\dim Y < \dim X$ならば
            $(*)$の一方である$H^i(X, \shF_Y)=0 \ (i>n)$が成立する.
    \end{enumerate}
    (i)と(ii)を合わせると,
    \[ \dim Y < \dim X \mand H^i(X, \shF_U)=0 \ (i>n) \implies H^i(X, \shF)=0 \ (i>n) \]
    ということになる.
\end{Lemma}
\begin{proof}
    (i)を示すには完全列:
    \[\xymatrix{ 0 \ar[r]& \shF_U \ar[r]& \shF \ar[r]& \shF_Y \ar[r]& 0. }\]
    から誘導される長完全列を見れば良い.
    仮定を用いれば次のように成る.
    \[
    \xymatrix@R=10pt
    {
        \dots
        \ar[r]^-{\delta^{n-1}}  & H^{n}(X,   \shF_U)  \ar[r]& H^{n}(X, \shF)   \ar[r]& H^{n}(X,   \shF_Y) \\
        \ar[r]^-{\delta^{n}}    & 0  \ar[r]& H^{n+1}(X, \shF) \ar[r]& 0 \\
        \ar[r]^-{\delta^{n+1}}  & 0  \ar[r]& H^{n+2}(X, \shF) \ar[r]& 0 \\
        \ar[r]^-{\delta^{n+2}}  & \dots
    }
    \]
    よって$H^i(X, \Z_U)=0 \ (i>n)$.
    (ii)は帰納法の仮定である.
\end{proof}

Thm2.7の証明のうち,Step $3$が難しい.
$\alpha \in A$は,$\shF$の有限個のsectionの集合である.
そこで$\alpha=\{ \sect{U_i}{s_i} \}_{i=1}^r$とする.
$\shF_{\alpha}$は次のように書ける.
\[ \shF_{\alpha}=\sum_{i=1}^r (\iota_i)_* (s_i\Z_{U_i}) \]
ここで$\iota_i$は包含写像$U_i \inclmap X$であり,
$s_i\Z_{U_i}$は$s_i$で生成されるabelian groupのsheafである.
$x \in X$でのstalkをとると,
\[ (\shF_{\alpha})_x=\sum_{i | x \in U_i} (s_i)_x \Z. \]
右辺は$\{ (s_i)_x \}_{i | x \in U_i}$で生成されるabelian group.
$A$は包含関係によってdirected setとなり,
$\{ \shF_{\alpha} \}_{\alpha \in A}$も
sheafとしての包含関係(subsheafの関係)でdirect systemとなる.
そして各$\shF_{\alpha}$は$\shF$のsubsheafである.
したがってdirect limitの定義から,
次の図式が可換に成るような
$\varinjlim \shF_* \to \shF$がただひとつ存在する.
\[
\xymatrix
{
    \varinjlim \shF_* \ar@{-->}[r]^-{\phi}& \shF \\
    \{\ \shF_{\alpha'} \ar@{_(->}[r] \ar[u]\ar@/_10pt/[ru]& \shF_{\alpha} \ar@/^10pt/[lu]\ar[u] \ \}
}
\]
この図式全体をstalkの関手$(-)_x$で写す.
direct limit同士は交換できることを用いると次のように成る.
\[
\xymatrix
{
    \varinjlim (\shF_*)_x \ar[r]^-{\phi_x}& \shF_x \\
    \{\ (\shF_{\alpha'})_x \ar@{_(->}[r] \ar[u]\ar@/_10pt/[ru]& (\shF_{\alpha})_x \ar@/^10pt/[lu]\ar[u] \ \}
}
\]
$(\shF_{\alpha})_x$は上のとおりのもの.
なので$\varinjlim (\shF_*)_x=\shF_x$.
また下の行(direct system)から上の行への射は
いずれも包含写像だから,
二つの三角形は同じものである.
\[
\xymatrix
{
    \varinjlim (\shF_*)_x & {} \\
    \{\ (\shF_{\alpha'})_x \ar@{_(->}[r] \ar[u]& (\shF_{\alpha})_x \ar@/^10pt/[lu] \ \}
}
\quad \raisebox{-20pt}{=} \quad 
\xymatrix
{
    {} & \shF_x \\
    \{\ (\shF_{\alpha'})_x \ar@{_(->}[r] \ar@/_10pt/[ru]& (\shF_{\alpha})_x \ar[u] \ \}
}
\]
よって$\phi_x$ :: iso.
II, Prop1.1から$\phi$ :: iso, すなわち$\varinjlim \shF_*=\shF$.

$\alpha' \subseteq \alpha$としよう.
すると次の完全列が存在する.
\[\xymatrix{ 0 \ar[r]& \shF_{\alpha'} \ar[r]& \shF_{\alpha} \ar[r]& \shG \ar[r]& 0 }\]
$x \in X$におけるstalkをとると,
$\alpha' \subseteq \alpha$ゆえ
\[
    \shG_x
    =(\shF_{\alpha})_x/(\shF_{\alpha'})_x
    =\sum_{i \Big| \substack{x \in U_i \\ \sect{U_i}{s_i} \in \alpha-\alpha'}} (s_i)_x \Z
    =(\shF_{\alpha-\alpha'})_x.
\]
となる.
よって$\shG=\shF_{\alpha-\alpha'}$.
なので$\#\alpha$($\alpha$の元の個数)について
帰納的に考えていくと,
$\alpha$が1元集合の場合のみ考えれば十分だとと分かる.
\[
    H^i(X, \shF_{\alpha'})=0 \mand H^i(X, \shF_{\{\sect{U}{s}\}})=0
    \implies
    H^i(X, \shF_{\alpha' \cup \{\sect{U}{s}\}})=0.
\]
標準的な全射
$\Z_X \to \shF_{\{\sect{U}{s}\}}=j_*(s \Z_U) \ (j: U \inclmap X$)
が存在するため,
この全射の$\ker$を$\shR$とすれば
$\shF_{\{\sect{U}{s}\}} \iso \Z_X/\shR$.

\section{Cohomology of a Noetherian Affine Scheme}

\end{document}
