\RequirePackage[]{silence}
%%% Warning Filter {{{
\WarningFilter{latexfont}{Font shape}
\WarningFilter{latexfont}{Some font shapes were not available}
%%% }}}
\documentclass[lualatex, ja=standard, a4paper]{bxjsarticle}
\usepackage{../math_note}

\newcommand{\step}[1]{\paragraph{\underline{#1}}}
\newcommand{\xz}{\boldsymbol{x}}
\newcommand{\az}{\boldsymbol{a}}

\begin{document}
    \title{Hilbert の零点定理}
    \author{七条 彰紀}
    \maketitle
    このノートでは,Hilbert の零点定理(独: Hilbert's nullstellensatz)の,
    Zariski の補題を用いた証明を扱う.
   
    以下では文字$x, \xz$を一貫して不定元を表すために使う.
    $\xz=(x_1, \dots, x_d)$とする.

\section{導入}
    \subsection{定理}
    \begin{Thm}[Hilbertの零点定理, 弱形]
        $k$を代数閉体とする.
        多項式環$k[\xz]=k[x_1,\dots,x_d]$の極大イデアルが成す集合を$\Max(k[\xz])$と書く.
        この時,以下で定まる対応$\mu$は全単射である.
        \begin{defmap}
            \mu:& k^d& \to& \Max(k[\xz]) \\ 
            {}& (a_1, \dots, a_d)& \mapsto& (x_1-a_1, \dots, x_d-a_d)
        \end{defmap}
    \end{Thm}

    \begin{Def}
        $k$を代数閉体,$d$を正の整数とする.
        \begin{enumerate}
        \item 
            $k[\xz]=k[x_1,\dots,x_d]$のイデアル$\I{a} \subsetneq k[\xz]$に対して,
            \[ \zeros(\I{a})=\{ p \in k^d \mid \Forall{f \in \I{a}} f(p)=0 \} \]
            とおく.

        \item
            $k^d$の部分集合$Z$に対して,
            \[ \defs(Z)=\{ f \in k[\xz] \mid \Forall{p \in Z} f(p)=0 \} \]
            とおく.
            $\defs(Z)$は$k[\xz]$のイデアルに成っている.

        \item
            環$R$のイデアル$I$に対して,
            \[ \sqrt{I}=\{ r \in R \mid \Exists{n > 0} r^n \in I \} \]
            とおく.
            $\sqrt{I}$は$R$のイデアルに成っている.
        \end{enumerate}
    \end{Def}

    \begin{Thm}[Hilbertの零点定理, 強形]
        $k$を代数閉体,$d$を正の整数とする.
        $k[\xz]=k[x_1,\dots,x_d]$の任意のイデアル$\I{a} \subsetneq k[\xz]$に対して,
        \[ \defs(\zeros(\I{a}))=\sqrt{\I{a}} \]
        が成立する.
    \end{Thm}

    \subsection{別の定式化}
    以上の二つの定理が「弱形」「強形」と並べられる理由は今ひとつ理解りにくい.
    Terence Taoは自身のブログ``Whatの New"にHilbertの 零点定理を扱った記事を掲載している\cite{tao}.
    それによると,以上の二つのステートメントはそれぞれ次のように言い換えられる.

    \begin{Thm}[Hilbertの零点定理, 弱形 by T.Tao]
        $k$を代数閉体とし,多項式$P_1, \dots, P_m \in k[\xz]$をとる.
        この時,以下のちょうど一方が成立する.
        \begin{enumerate}[label=\arabic*.]
            \item 方程式系$P_1(\xz)=\ldots=P_m(\xz)=0$が解$\xz=(a_1, \dots, a_d) \in k^d$を持つ.
            \item$P_1 Q_1 + \ldots + P_m Q_m=1$を満たす多項式$Q_1,\ldots,Q_m \in k[x]$が存在する.
        \end{enumerate}
    \end{Thm}

    \begin{Thm}[Hilbertの零点定理, 強形 by T.Tao]
        $k$を代数閉体とし,多項式$P_1, \dots, P_m, R\in k[\xz]$をとる.
        この時,以下のちょうど一方が成立する.
        \begin{enumerate}[label=\arabic*.]
            \item 方程式系$P_1(\xz)=\ldots=P_m(\xz)=0, R(\xz) \neq 0$が解$\xz=(a_1, \dots, a_d) \in k^d$を持つ.
            \item$P_1 Q_1 + \ldots + P_m Q_m=R^r$を満たす多項式$Q_1,\ldots,Q_m \in k[\xz]$と非負整数$r$が存在する.
        \end{enumerate}
    \end{Thm}
    このように弱形は強形で$R=1$とした場合であることは明白である.
    したがって強形 $\implies$ 弱形が分かる.

\section{証明のための準備}
    
\begin{Lemma}[Zariski's Lemma]
体$k$上の有限生成代数$K$が体ならば,$K$は$k$の有限次代数拡大体である.
\end{Lemma}

\paragraph{Noether normalization theoremを使うもの.}
\begin{proof}
    Noether normalization theoremにより,
    有限生成代数$K$が$R:=k[y_1, \dots, y_m]$の整拡大となり,
    しかも$k$上代数独立であるような元$y_1, \dots, y_m$が存在する.
    
    $m>0$とする.
    $y_1 \in K$ (::field)なので$y_1^{-1} \in K$.
    したがって$y_1^{-1}$は$R$上整であるから,
    以下が成立するような$f \in R$と非負整数$n$が存在する.
    \[ (y_1^{-1})^n+f(y_1, \dots, y_m) (y_1^{-1})^{n-1}=0 \]
    この両辺に$y_1^{n}$を掛けると,
    \[ 1+f(y_1, \dots, y_m) y_1=0 \]
    となり,これは$y_1, \dots, y_m$が$k$上代数独立
    \footnote{$f(y_1, \dots, y_m)=0$となる0でない多項式$f \in k[x_1, \dots, x_m]$が存在しない} 
    であることに矛盾する.
    よって$m=0$.

    以上より,$K$は$k$の整拡大,すなわち代数拡大となる.
    再び$K$が$k$上有限生成代数な体であることから,$K$は$k$の有限次代数拡大体.
\end{proof}
%「$m>0$とする....よって$m=0$.」の部分は,
%「$m>0$とする.
%$R:=k[y_1, \dots, y_m]$のKrull次元は$m$以上.
%だからその整拡大$K$のKrull次元も$m$以上.
%しかし体$K$のKrull次元は0なので矛盾.」
%に置き換えることも出来る.
%ただしこの書き方ではGoing-Up Theoremを用いることになる.
%「$R$のKrull次元は$m$以上.」は,
%\[ 0 \subset (y_1) \subset \cdots \subset (y_1, \dots, y_m) \]
%という$R$のイデアル列を作り,これらの一つ一つが素イデアルであることを示せば良い.

\paragraph{整従属性を使うもの.}
(\cite{atimac}, Ex5.18と\cite{oneline}を参照)
\begin{proof}
    $k$代数としての$K$の生成元を$x_1,\dots,x_n$とする.
    $n=1$ならば定理の成立は自明なので$n>1$としよう.
    示したいことは$x_1,\dots,x_n$のすべてが$k$上代数的であること.
    なので帰納的に考えて,
    $x_2,\dots,x_n$が$k(x_1)$上代数的ならば$x_1,\dots,x_n$が$k$上代数的であることを示せば良い
    \footnote
    {
        言い換えれば
        $K=k(x_1,\dots,x_{n-2})(x_{n-1})[x_n] 
        \implies \dots \implies K=k(x_1)[x_2,\dots,x_n] \implies K=k[x_1][x_2\dots,x_n]$
    }.
    そこで,$x_1$が$k$上代数的でなく,同時に
    $x_2,\dots,x_n$が$k(x_1)$上代数的であると仮定し,背理法を用いる.
    この時,$K=k(x_1)[x_2,\dots,x_n]$となる.

    $x_2,\dots,x_n$が$k[x_1]_f(=k[x_1][1/f])$上代数的であるような$f \in k[x_1]$が存在する.
    実際,$x_2$が$k(x_1)$上代数的であることから,
    次の式を満たす$f_{i},g_{i} \in k[x_1]$が存在する.
    \[
        x_2^{d}+\left( \frac{g_{d-1}}{f_{d-1}} \right) x_2^{d-1}+\dots+\left(\frac{g_{0}}{f_{0}}\right)=0
        \mwhere d>0, f_{i},g_{i} \in k[x_1], g_{i} \neq 0.
    \]
    $\tilde{f}_2:=\prod_{i=0}^{d-1} f_{i}$とおく.
    $\frac{g_{d-1}}{f_{d-1}}$から$\frac{g_{0}}{f_{0}}$までを通分すると,
    $x_2$は$k[x_1][1/\tilde{f}_2]$上整であることが分かる.
    $x_3,\dots,x_n$も同様にして,結局$x_2,\dots,x_n$が
    $k[x_1]\left[1/\tilde{f}_2,\dots,1/\tilde{f}_n\right]$
    上代数的になるような$\tilde{f}_2,\dots,\tilde{f}_n \in k[x_1]$が存在することが分かる.
    さらに$f=\prod_{i=2}^n \tilde{f}_i$とおくと,$x_2,\dots,x_n$は
    $k[x_1]\left[1/f\right]=k[x_1]_f$
    上代数的であると言える.

    以上から,$K=k(x_1)[x_2,\dots,x_n]$は$k[x_1]_f$上整である.
    この整従属関係と$K$が体であることから$k[x_1]_f$も体(\cite{atimac}, Prop5.7).
    $k[x_1] \subseteq k[x_1]_f \subseteq k(x)$かつ$k(x_1)$が$k[x_1]$を含む最小の体(商体)であることから
    $k(x_1)=k[x_1]_f$.
    しかし実際は$k[x_1]_f \neq k(x_1)$となる
    \footnote
    {
    実際,仮定から$x_1$は$k$上超越的だから,
    $f$は$k[x_1]$の有限個の既約多項式の積に分解され,$k[x_1]$は無数の既約多項式を持つ.
    なので$f$と互いに素な既約多項式$g \in k[x_1]$が存在する.
    $1/g=h/f^n$となる$n>0, h \in k[x_1]$が存在すれば,$gh=f^n=f \cdot f^{n-1} \in (g)$.
    $g$は素元だから$f \in (g)$となり,$f,g$が互いに素であることに反する.
    よって$1/g \not \in k[x_1]_f$.
    }.
    よって矛盾が生じた.
\end{proof}


    \input{noether_normalization_thm}

    どうせなら使用したNoether normalization theoremと
    \cite{atimac} Prop5.7の証明も書いてself-containedにしたいところである.
    これはTODOとする.

\section{弱形の証明}
    \step{$(x_1-a_1, \dots, x_d-a_d)$は極大イデアル.}
    各$x_i$を$x_i \mapsto a_i$と写す写像を考える.
    明らかにこれは全射で,$\ker = (x_1-a_1, \dots, x_d-a_d)$.
    準同型定理から$k[\xz]/(x_1-a_1, \dots, x_d-a_d) \cong k$が得られる.
    剰余環が体になったので,$(x_1-a_1, \dots, x_d-a_d)$は$k[\xz]$の極大イデアル.

    \step{$\mu$は単射.}
    自明である.

    \step{$\mu$は全射.}
    多項式環$k[\xz]$の極大イデアル$\I{m} \in \Max(k[\xz])$を任意に取る.
    この時,剰余$L=k[\xz]/\I{m}$は体.
    しかも$\tilde{a}_i=x_i+\I{m}$とおけば$L=k[\{\tilde{a}_i\}_{i=1}^d]$と書けるから,$L$は有限生成$k$-代数.
    Zariskiの補題より,$L/k$は有限代数拡大である.
    $k$は代数的閉体であったから,$L \cong k$となり,よって各$\tilde{a}_i$は$k$の元$a_i$に対応する.
    こうして点$\az=(a_1,\dots,a_d)$が得られた.
    再び$x_i \mapsto a_i$という写像(像は$k[\{a_i\}_{i=1}^d]=k$)に準同型定理を用いれば,
    \[ k[\xz]/\mu(\az) \cong k[\{a_i\}_{i=1}^d] \cong k[\{\tilde{a}_i\}_{i=1}^d]=k[\xz]/\I{m} \]
    という同型が構成できる.
    したがって$\I{m}=\mu(\az)$.

\section{強形の証明}
    $\sqrt{\I{a}} \subseteq \defs(\zeros(\I{a}))$は明らか.
    逆に$f \not \in \sqrt{\I{a}}$として$f \not \in \defs(\zeros(\I{a}))$を示す.

    \step{素イデアル$\I{p}$の存在.}
    $\sqrt{\I{a}}$は$\I{a}$を含む素イデアル全体の共通部分であるから,
    この時$\I{a} \subseteq \I{p}, f \not \in \I{p}$なる素イデアル$\I{p}$が存在する.

    \step{体$L$の構成.}
    $\bar{f}=f+\I{p} (\neq 0)$とし,$C=(k[\xz]/\I{p})_f=(k[\xz]/\I{p})[1/\bar{f}]$とする.
    さらに$\I{m}$を$C$の極大イデアルとおく.
    すると体$L=C/\I{m}=(k[\xz]/\I{p})_f/\I{m}$は
    $\tilde{a}_i=\frac{x_i+\I{p}}{1}+\I{m}$で生成される有限生成$k$-代数.

    \step{$\az \in \zeros(\I{a})$かつ$f(\az) \neq 0$なる点$\az$を得る.}
    Zariskiの補題より,$L/k$は有限代数拡大である.
    $k$は代数的閉体であったから,$L \cong k$となり,
    よって各$\tilde{a}_i$は$k$の元$a_i$に対応する.
    こうして点$\az=(a_1,\dots,a_d)$が得られた.
    ここで以下の準同型を考える.
    \[
        \phi: k[\xz] \to k[\xz]/\I{p} \to (k[\xz]/\I{p})_f=C \to C/\I{m} \cong k;
    ~~ x_i \mapsto x_i+\I{p} \mapsto \frac{x_i+\I{p}}{1} \mapsto \frac{x_i+\I{p}}{1}+\I{m}=\tilde{a}_i \mapsto a_i.
    \]
    これは代入写像.
    繋いでいる写像はすべて準同型なので,
    $g \in \I{p}$は$C$の零元$\frac{0+\I{p}}{1}$へ写り,最終的に零元$0$へ写る.
    同様に,$f$は$C$の単元$\frac{f+\I{p}}{1}$へ写り,最終的に単元へ写る.
    つまり$g \in \I{p}$について$\phi(g)=g(\az)=0$で,$\phi(f)=f(\az)$は単元.
    よって$\az \in \zeros(\I{p}) \subset \zeros(\I{a})$かつ$f(\az) \neq 0$.

\section{その他の証明方針}
    重要定理だけあって,証明の方針はかなり多い.
    \begin{enumerate}
        \item 多項式環$k[\xz]$が Jacobson 環であることを用いる証明,
        \item 幾何的な Noether normalization theorem を用いる証明,
        \item Rabinowitsch's Trick を用いる証明,
        \item Artin-Tate の補題を用いる証明(\cite{atimac}, Prop7.9),
        \item 付置環を用いる証明(\cite{atimac}, Cor5.24),
        \item Chevalley's theorem about constructable set を用いる証明,
        \item Terence Tao と Enrique Arrondo による終結式を用いる証明,
        \item モデル理論的な証明.
    \end{enumerate}

\begin{thebibliography}{9}
    \bibitem{atimac}
        M.F.Atiyah, I.G.MacDonald
        ''Introduction to Commutative Algebra"

    \bibitem{tao}
        Terence Tao (2007/11/27)
        ''Hilbert’s nullstellensatz"
        \url{https://terrytao.wordpress.com/2007/11/26/hilberts-nullstellensatz/}

    \bibitem{oneline}
        Alborz Azarang
        ''A one-line undergraduate proof of Zariski's lemma and Hilbert's nullstellensatz"
        \url{http://arxiv.org/abs/1506.08376}
\end{thebibliography}
\end{document}
